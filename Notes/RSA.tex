\documentclass[12pt]{article}

\usepackage{amsthm}
\usepackage{amsmath}
\usepackage{amsfonts}
\usepackage{mathrsfs}
\usepackage{amssymb}
\usepackage{units}
\usepackage{graphicx}
\usepackage{tikz-cd}
\usepackage{nicefrac}
\usepackage{hyperref}
\usepackage{bbm}
\usepackage{color}
\usepackage{tensor}
\usepackage{tipa}
\usepackage{bussproofs}
\usepackage{ stmaryrd }
\usepackage{ textcomp }
\usepackage{leftidx}
\usepackage{afterpage}
\usepackage{varwidth}

\newcommand\blankpage{
    \null
    \thispagestyle{empty}
    \addtocounter{page}{-1}
    \newpage
    }

\graphicspath{ {images/} }

\theoremstyle{plain}
\newtheorem{thm}{Theorem}[subsection] % reset theorem numbering for each chapter
\newtheorem{proposition}[thm]{Proposition}
\newtheorem{lemma}[thm]{Lemma}
\newtheorem{fact}[thm]{Fact}
\newtheorem{cor}[thm]{Corollary}

\theoremstyle{definition}
\newtheorem{defn}[thm]{Definition} % definition numbers are dependent on theorem numbers
\newtheorem{exmp}[thm]{Example} % same for example numbers
\newtheorem{notation}[thm]{Notation}
\newtheorem{remark}[thm]{Remark}
\newtheorem{condition}[thm]{Condition}
\newtheorem{question}[thm]{Question}
\newtheorem{construction}[thm]{Construction}
\newtheorem{exercise}[thm]{Exercise}
\newtheorem{example}[thm]{Example}

\newcommand{\bb}[1]{\mathbb{#1}}
\newcommand{\scr}[1]{\mathscr{#1}}
\newcommand{\call}[1]{\mathcal{#1}}
\newcommand{\psheaf}{\text{\underline{Set}}^{\scr{C}^{\text{op}}}}
\newcommand{\und}[1]{\underline{\hspace{#1 cm}}}
\newcommand{\adj}[1]{\text{\textopencorner}{#1}\text{\textcorner}}
\newcommand{\comment}[1]{}
\newcommand{\lto}{\longrightarrow}

\usepackage[margin=1cm]{geometry}

\title{RSA Cryptography}
\author{Will Troiani}
\date{August 2020}

\begin{document}

\maketitle
\tableofcontents

\section{Relevant number theory}
First we define a function
\begin{defn}
Given a positive integer $n$, let $d_n$ denote the number of integers $m$ such that $\operatorname{gcd}(n,m) = 1$ where $m \leq n$. The function
\begin{align*}
\phi: \bb{Z}_{\geq 0} &\lto \bb{Z}_{\geq 0}\\
n &\longmapsto d_n
\end{align*}
is \textbf{Euler's Totient function}.
\end{defn}
This function admits a surprising form:
\begin{proposition}
Let $n \geq 0$ and write $n = p_1^{k_1}\cdot\hdots\cdotp_n^{k_n}$. Then
\begin{equation}\label{eq:explicit}
\phi(n) = p_1^{k_1 - 1}(p_1 - 1)\cdot\hdots\cdot p_n^{k_n-1}(p_n-1)
\end{equation}
\end{proposition}
\begin{example}
Applying \eqref{eq:explicit} we find:
\begin{equation}
\phi(10) = \phi(2.5) = 2^{1-1}(2-1)5^{1-1}(5-1) = 1.1.1.4 = 4
\end{equation}
and indeed
\begin{equation}
\operatorname{gcd}(1,10) = \operatorname{gcd}(3,10) = \operatorname{gcd}(7,10) = \operatorname{gcd}(9,10)
\end{equation}
\end{example}
RSA cryptography will hinge crucially on the following Theorem due to Euler:
\begin{thm}[Euler's Theorem]\label{thm:euler}
Let $a$ be a number coprime to $n$. Then
\begin{equation}
a^{\phi(n)} = 1\text{ }(\operatorname{mod}n)
\end{equation}
\end{thm}
\section{RSA Cryptography}
A significant feature of RSA cryptography is that if $A$ wants to send a message to $B$, then in fact only $B$ needs to know the private key. That is, the private key can remain hidden \emph{even from $A$}.

Say $A$ is sending $B$ a message $m$.
\begin{enumerate}
\item First, $B$ picks two prime numbers, $p,q$ say.
\item $B$ makes the following calculations/choices:
\begin{enumerate}
\item Calculate $n = pq$,
\item Pick an integer $e$ such that $e$ is coprime to $\phi(n)$,
\item Calculate an integer $d$ such that
\begin{equation}
ed = 1\text{ }\operatorname{mod}\phi(n)
\end{equation}
\item Then $B$ sends $A$ the integer $e$ (these are the public key).
\item It is important that $B$ keeps $d$ to themselves (this is the private key).
\end{enumerate}
\end{enumerate}
Then, $A$ write a message and encodes it in an integer in some way, for instance, by translating each letter into ASCII and then taking the integer which is given by the concatenation of these numbers, call this number $m$. Note, it is important that $n$ is taken large enough so that $n > m$. Then $A$ sends $c := m^e$ to $B$.
\begin{enumerate}
\item $B$ receives $c$ and performs the following calculation:
\begin{align}
c^d &= (m^e)^d\\
&= m^{ed}\label{eq:intimediate_calc}
\end{align}
Now, we have chosen $d$ such that $ed = 1\text{ }\operatorname{mod}\phi(n)$, so there exists $k > 0$ such that
\begin{equation}
ed = \phi(n)k + 1
\end{equation}
Continuing with Calculation \eqref{eq:intimediate_calc}:
\begin{align}
m^{ed} &= m^{\phi(n)k + 1}\\
&= (m^{\phi(n)})^k\cdot m
\end{align}
taking this equation modulo $n$, we now have (by Euler's Theorem (Theorem \ref{thm:euler})):
\begin{align}
(m^{\phi(n)})^k\cdot m &= 1^k\cdot m\text{ }\operatorname{mod}n\\
&= m\text{ }\operatorname{mod}n\
\end{align}
\end{enumerate}
and so $B$ has uncovered the message.





























\end{document}