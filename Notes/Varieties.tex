\documentclass[12pt]{article}

\usepackage{algorithm}
\usepackage{algpseudocode}
\usepackage{amsthm}
\usepackage{amsmath}
\usepackage{amsfonts}
\usepackage{mathrsfs}
\usepackage{amssymb}
\usepackage{units}
\usepackage{graphicx}
\usepackage{tikz-cd}
\usepackage{nicefrac}
\usepackage{hyperref}
\usepackage{bbm}
\usepackage{color}
\usepackage{tensor}
\usepackage{tipa}
\usepackage{bussproofs}
\usepackage{ stmaryrd }
\usepackage{ textcomp }
\usepackage{leftidx}
\usepackage{afterpage}
\usepackage{varwidth}

\newcommand\blankpage{
    \null
    \thispagestyle{empty}
    \addtocounter{page}{-1}
    \newpage
    }

\graphicspath{ {images/} }

\theoremstyle{plain}
\newtheorem{thm}{Theorem}[subsection] % reset theorem numbering for each chapter
\newtheorem{proposition}[thm]{Proposition}
\newtheorem{lemma}[thm]{Lemma}
\newtheorem{fact}[thm]{Fact}
\newtheorem{cor}[thm]{Corollary}
\newtheorem{claim}[thm]{Claim}

\theoremstyle{definition}
\newtheorem{defn}[thm]{Definition} % definition numbers are dependent on theorem numbers
\newtheorem{exmp}[thm]{Example} % same for example numbers
\newtheorem{notation}[thm]{Notation}
\newtheorem{remark}[thm]{Remark}
\newtheorem{condition}[thm]{Condition}
\newtheorem{question}[thm]{Question}
\newtheorem{construction}[thm]{Construction}
\newtheorem{exercise}[thm]{Exercise}
\newtheorem{example}[thm]{Example}
\newtheorem{aside}[thm]{Aside}
%\newtheorem{algorithm}[thm]{Algorithm}

\newcommand{\bb}[1]{\mathbb{#1}}
\newcommand{\scr}[1]{\mathscr{#1}}
\newcommand{\call}[1]{\mathcal{#1}}
\newcommand{\psheaf}{\text{\underline{Set}}^{\scr{C}^{\text{op}}}}
\newcommand{\und}[1]{\underline{\hspace{#1 cm}}}
\newcommand{\adj}[1]{\text{\textopencorner}{#1}\text{\textcorner}}
\newcommand{\comment}[1]{}
\newcommand{\lto}{\longrightarrow}

\usepackage[margin=1cm]{geometry}

\title{Varieties}
\author{Will Troiani}
\date{December 2020}

\begin{document}

\maketitle
\tableofcontents
\section{Introduction}
The standard text to read before studying Algebraic Geometry is Hartshorne \cite{hartshorne}. However, the chapter titled \emph{Varieties} leaves many important theorems unproved (although references are provided), many important concepts arrive without motivation (eg, regular functions) and lots of crucial content is pushed into the exercises. The present notes fill these blanks, and should be read alongside the notes Commutative Algebra \cite{algebra} and Hartshorne Exercise Solutions \cite{hartshorne_solutions}. It is recommended that this is taken as the main resource, with the others used for reference.
\section{Affine Varieties}
\subsection{Algebraic sets and the ideal of a set}
Loosely speaking, a zero $a \in A$ in some $k$-algebra $A$ of a polynomial $p$ is a generalisation of an arithmetic equation, for instance the equation $1 + 2 = 3$ says ``the pair $(1,2) \in \bb{C}^2$ is a zero to the polynomial $x + y - 3 \in \bb{C}[x,y]$". Thus, to study zero sets of polynomials is to study the structure of algebraic equations. Furthermore, polynomials as \emph{functions} are themselves a primitive object of study as they approximate continuous functions arbitrarily well, a statement made formal by the \emph{Stone-Weierstrass Theorem}. Algebraic Geometry takes polynomials along with the functions they induce as the fundamental object of study. Throughout, we work with an algebraically closed field $k$.
\begin{defn}
Let $\frak{a} \subseteq k[x_1,...,x_n]$ be an idea. Define the \textbf{zero set of $\frak{a}$}:
\begin{equation}
    Z(\frak{a}) := \lbrace (p_1,...,p_n) \in k^n \mid \forall q \in \frak{a}, q(p_1,...,p_n) = 0\rbrace
\end{equation}
A subset $T \subseteq k^n$ for which there exists an ideal $\frak{a} \subseteq k[x_1,...,x_n]$ so that $T = Z(\frak{a})$ is an \textbf{algebraic set}.

Given an arbitrary set of points $T \subseteq k^n$ we define the \textbf{ideal of $T$}:
\begin{equation}
    I(T) := \lbrace q \in k[x_1,...,x_n] \mid \forall p = (p_1,...,p_n) \in T, q(p_1,...,p_n) = 0 \rbrace
\end{equation}
which is in fact an ideal.
\end{defn}
Two non-equal ideals may have equal zero sets, for instance $Z(x^2) = Z(x)$. This particular equality is easy to see because $k$ being a field admits no nilpotent elements, so for any $x \in k$ such that $x^2 = 0$ is necessarily such that $x = 0$. Our first goal is to answer the question:
\begin{question}\label{question:zero_set}
When do two ideals yield the same zero set?
\end{question}
This will be answered by \emph{Hilbert's Nullstellensatz}, which we state on the level of generality where $k$ is not necessarily algebraically closed.
\begin{defn}
\label{def:algzero}
Let $F$ be a field (not necessarily algebraically closed). An \textbf{algebraic zero} of a subset $\Phi \subseteq F[x_1,...,x_n]$ is a sequence $(\alpha_1,...,\alpha_n)$ of elements in an algebraic closure $\bar{F}$ such that $f(\alpha_1,...,\alpha_n) = 0$ for all $f \in \Phi$.
\end{defn}
Notice that if a root exists in any algebraic closure it exists in them all, so it makes sense to talk about an algebraic zero in absence of a particular algebraic closure.
\begin{thm}
\label{thm:hilbertsnullstellensatz}
Let $\Phi \subseteq F[x_1,...,x_n]$, and write $(\Phi)$ for the ideal generated by $\Phi$,
\begin{enumerate}
    \item\label{generates} if $\Phi$ admits no algebraic zeros, then $(\Phi) = F[x_1,...,x_n]$.
    \item\label{power} let $f \in F[x_1,...,x_n]$ be such that $f(\alpha_1,...,\alpha_n) = 0$ for all algebraic zeros $(\alpha_1,...,\alpha_n)$ of $\Phi$, then there exists $r > 0$ such that $f^r \in (\Phi)$.
\end{enumerate}
\end{thm}
\begin{proof}
See \cite{algebra}.
\end{proof}
Thus we have:
\begin{lemma}
Let $\frak{a} \subseteq k[x_1,...,x_n]$ be an ideal. Then
\begin{equation}\label{eq:IZ}
    IZ(\frak{a}) = \sqrt{\frak{a}}
\end{equation}
\end{lemma}
We thus have $ZIZ(\frak{a}) = Z(\sqrt{\frak{a}})$, our next goal is to show that $ZIZ(\frak{a}) = Z(\frak{a})$ provides an answer to Question \ref{question:zero_set}:
\begin{lemma}
Let $\frak{a},\frak{b} \subseteq k[x_1,...,x_n]$ be ideals. Then
\begin{equation}
    Z(\frak{a}) = Z(\frak{b}) \Longleftrightarrow \sqrt{\frak{a}} = \sqrt{\frak{b}}
\end{equation}
\end{lemma}
So we must establish $ZIZ(\frak{a}) = Z(\frak{a})$. We take this as an invitation to develop some more general theory. First notice the algebraic sets form a topology on $k^n$:
\begin{lemma}
The algebraic sets form a collection of closed sets, in fact:
\begin{enumerate}
    \item $\varnothing = Z(1)$, and $k^n = Z(0)$,
    \item $Z(\frak{a}) \cup Z(\frak{b}) = Z(\frak{a}\frak{b})$,
    \item $\bigcap_{i \in I}Z(\frak{a}_i) = Z(\frak{b})$, where $\frak{b}$ is the ideal generated by $\bigcup_{i \in I}\frak{a}_i$.
\end{enumerate}
\end{lemma}
\begin{proof}
Simple calculations.
\end{proof}
From now on, we consider $k^n$ as a topological space whose closed sets are given by the algebraic sets:
\begin{defn}
The topology on $k^n$ defined by taking all algebraic sets to be closed sets is the \textbf{Zariski topology}. The set $k^n$ along with the Zariski topology is \textbf{Affine $n$-space} and is denoted $\bb{A}^n$.
\end{defn}
We now have the language required to state the key result:
\begin{lemma}
Let $T \subseteq \bb{A}^n$ be a subset, then
\begin{equation}
    ZI(T) = \overline{T}
\end{equation}
\end{lemma}
\begin{proof}
We clearly have $T \subseteq ZI(T)$ and that $ZI(T)$ is closed so it remains to show that if $W = Z(\frak{a}) \supseteq T$ is a closed set containing $T$ then $ZI(T) \subseteq Z(\frak{a})$.  Since $T \subseteq Z(\frak{a})$ we have $IZ(\frak{a}) \subseteq I(T)$. Also, $\frak{a} \subseteq IZ(\frak{a})$ and so $Z(\frak{a}) \supseteq ZIZ(\frak{A}) \supseteq ZI(T)$. Thus $\overline{T} = ZI(T)$.
\end{proof}
We thus have $ZIZ(\frak{a}) = \overline{Z(\frak{a})} = Z(\frak{a})$.

The central objects of study are the algebraic sets, these decompose uniquely (Proposition \ref{prop:decomposition}) into a finite union of \emph{irreducible} algebraic sets
\begin{defn}
A non-empty topological space $X$ is \textbf{irreducible} if it cannot be written as the union of two non-empty, proper, closed subsets. A subset $Y \subseteq X$ is \textbf{irreducible} if it is an irreducible space when endowed with the subspace topology.
\end{defn}
So, we define:
\begin{defn}
An irreducible, algebraic set is an \textbf{affine variety}. An open subset of an affine variety is a \textbf{quasi-affine variety}.
\end{defn}
The theory of Algebraic Geometry is built so that both sides, the algebra and the geometry, complement one another. As what we have seen so far suggests, the first example of this is the following connection between the algebraic sets and radical ideals:
\begin{proposition}\label{prop:alg_rad}
There is an order reversing bijection
\begin{align}
    \Phi: \lbrace \text{Algebraic subsets of }\bb{A}^n\rbrace &\stackrel{\sim}{\lto}\lbrace \text{Radical ideals of }k[x_1,...,x_n]\rbrace\\
    T &\longmapsto I(T)
\end{align}
with inverse given by $\Phi^{-1}(\frak{a}) = Z(\frak{a})$. An algebraic subset $T \subseteq \bb{A}^n$ is irreducible if and only if $\Phi(T)$ is prime.
\end{proposition}
\begin{proof}
A matter of unwinding definitions, see \cite[\S I 1.4]{hartshorne} for details.
\end{proof}
We know by Hilbert's Basis Theorem (see \cite{algebra}) that $k[x_1,...,x_n]$ is a Noetherian ring. This along with Proposition \ref{prop:alg_rad} motivate the following definition:
\begin{defn}
A topological space $X$ is \textbf{Noetherian} if every strictly decreasing chain of closed subsets is finite. In other words, for every strictly descending chain of closed sets:
\begin{equation}
    X_1 \supsetneq X_2 \supsetneq \hdots
\end{equation}
there exists $N > 0$ such that for all $n > N$ we have $X_n = X_{n+1}$.
\end{defn}
\begin{proposition}\label{prop:decomposition}
Let $X$ be a Noetherian topological space. Every non-empty, closed subset $Y$ can be written as the union $Y = Y_1 \cup \hdots \cup Y_n$ of irreducible, closed subsets. Moreover, if for all $i \neq j$ we have that $X_i$ is not contained in $X_j$ then this sequence is unique.
\end{proposition}
\begin{proof}
Let $\scr{X}$ be the set of non-empty, closed subsets of $X$ which \emph{cannot} be written as the finite union of irreducible closed subsets. Since $X$ is Noetherian this set contains a minimal element, say $Y$. Then since $Y$ is not irreducible we have that $Y = Y_1 \cup Y_2$ which are both closed and non-empty. By minimality of $Y$ we have that $Y_1,Y_2$ both can be written as a finite union of irreducible subsets and thus, so can $Y$, a contradiction.

For the uniqueness claim, see \cite[\S I 1.5]{hartshorne}
\end{proof}
%
%
%
\subsection{Irreducible sets}\label{sec:irreducible}
In Section \ref{sec:dimension} the \emph{dimension} of a vector space will be generalised to make sense for arbitrary affine varieties. This will be done by looking at the \emph{irreducible subsets} of an affine variety. This Section establishes some elementary properties of such sets and is possibly better skipped and then referred back to as questions naturally arise in the reader's mind. The following provides an alternative check for irreducibility:
\begin{lemma}\label{lem:irred_criterion}
A subsets $Y \subseteq X$ of a topological space $X$, then the following are equivalent:
\begin{enumerate}
    \item\label{lem:irred_irred} $Y$ is irreducible,
    \item\label{lem:irred_dense_open} every non-empty open subset $U \subseteq Y$ of $Y$ is dense,
    \item\label{lem:irred_pair} every pair of non-empty open subsets $U_1,U_2 \subseteq Y$ have non-empty intersection.
\end{enumerate}
\end{lemma}
\begin{proof}
\ref{lem:irred_irred} $\Rightarrow$ \ref{lem:irred_dense_open}: Say $U \subseteq Y$ is non-empty, open, and not dense. Then $Y = \overline{U} \cup U^c$.

\ref{lem:irred_dense_open} $\Rightarrow$ \ref{lem:irred_pair}: Say $U_1,U_2$ were non-empty, open, and had empty intersection. Then $U_2^c$ is a closed set containing $U_1$ and so $U_2^c \supseteq \overline{U_1}$. It follows that $\overline{U_1} \neq Y$ and so $U_1$ is not dense.

\ref{lem:irred_dense_open} $\Rightarrow$ \ref{lem:irred_irred}: Say $Y = Y_1 \cup Y_2$ with $Y_1,Y_2 \subseteq Y$ both non-empty, proper subsets of $Y$. Then $Y_1^c \cap Y$ and $Y_2^c \cap Y$ are non-empty, open subsets of $Y$ with emtpy intersection.
\end{proof}
Given that a space $X$ is irreducible if every open subset is dense, it may seem like every subset of an irreducible set is irreducible, however this is not true:
\begin{lemma}
A closed subset of an irreducible set need not be irreducible.
\end{lemma}
\begin{proof}
The closed set $\lbrace 0, 1 \rbrace \subseteq \bb{A}^1$ provides an example of a closed subset of an irreducible space which is not irreducible.
\end{proof}
A helpful property of irreducibility is that it is transitive:
\begin{lemma}\label{lem:irred_subset_irred}
If $Y$ is an irreducible subset of a toplogical space $X$ and $Z\subseteq X$ is an irreducible subset of $X$, then $Z$ is irreducible as a subset of $Y$.
\end{lemma}
\begin{proof}
Let $Z = U \cup V$ where $U = U' \cap Z$, $V = V' \cap Z$ with $U',V' \subseteq X$ closed. Then
\[Y = Z\cup Y = \big( (U' \cap Z) \cup (V' \cap Z)\big) \cup Y = (U' \cap Y) \cup (V' \cap Y) = U' \cup V'\]
which implies that $U'= Y$, say. Thus $Z = Z \cap U' = U$ which shows that $Z$ is irreducible.
\end{proof}
Irreducibility is preserved by closure, (a helpful corollary of this is: the closure of a quasi-affine variety is an affine variety):
\begin{lemma}
The closure of an irreducible set is irreducible.
\end{lemma}
\begin{proof}
Let $Y$ be an irreducible subset of a topological space $X$. Say $\bar{Y} = U \cup V$ with $U,V \subseteq \bar{Y}$ closed. Then $\bar{Y} = (\bar{Y} \cap U') \cup (\bar{Y} \cap V')$ so taking the intersection with $Y$ we obtain $Y = (Y \cap U') \cup (Y \cap V')$ which by irreducibility of $Y$ implies $Y = Y \cap U'$, say. Taking the closure of both sides we obtain 
\[\bar{Y} = \overline{Y \cap U'} = \bar{Y} \cap \bar{U'} = \bar{Y} \cap U'\]
where the last equality holds as $U'$ is closed.
\end{proof}
%\begin{lemma}
%\label{lem:irred_trans}
%If $U \subset X$ is an irreducible subset and $V \subseteq U$ is irreducible then $V$ is an irreducible subset of $X$.
%\end{lemma}
%\begin{proof}
%Say $V = V_1 \cup V_2$ with $V_i \subseteq X$ closed in $V$, that is there exists $V_1',V_2' \subseteq X$ closed such that $V_i = V_i' \cap V$. Assume these are proper subsets. Then
%\[V = U \cap V = U \cap \big((V \cap V_1') \cup (V \cap V_2')\big) = (V \cap (U \cap V_1')) \cup (V \cap (U \cap V_2'))\]
%that is, $V$ can be written as the union of two proper subsets of $U$ both of which are closed in $V$ (as $U \cap V_i'$ are closed subsets of $U$), thus $V$ is a reducible subset of $U$.
%\end{proof}
\subsection{Dimension}\label{sec:dimension}
In linear algebra, \emph{dimension} is defined to be the number of basis elements necessary to describe the space. Due to the linearity of this setting, any choice of (ordered) basis establishes an isomorphism to Euclidean space and so this definition fits an intuitive idea of what dimension ought to mean. Thinking of $\bb{A}^n$ as a $k$-vector space with standard basis vectors $e_1,...,e_n$ we can define the descending sequence of subspaces
\begin{equation}
    \bb{A}^n = \operatorname{Span}(e_1,...,e_n) \supsetneq \operatorname{Span}(e_1,...,e_{n-1}) \supsetneq \hdots \supsetneq \operatorname{Span}(e_1) \supsetneq 0
\end{equation}
using the language of varieties as:
\begin{equation}\label{eq:dimension_chain}
    \bb{A}^n = Z(\varnothing) \supsetneq Z(x_n) \supsetneq \hdots \supsetneq Z(x_2,...,x_n) \supsetneq Z(x_1,...,x_n)
\end{equation}
So we see the number of basis vectors is equal to the length of the chain \eqref{eq:dimension_chain}, moreover, this chain is of maximal length, a fact which will be proved later. The situation is the same for any \emph{linear affine variety}:
\begin{defn}
An affine variety $X \subseteq \bb{A}^n$ is \textbf{linear} if it is equal to the zero set of a collection of linear polynomials:
\begin{equation}
    X = Z(f_1,...,f_n)
\end{equation}
all $f_i$ linear.
\end{defn}
Notice that since each $f_i$ is linear if $P = (P_1,...,P_n) \in X$ is a point, then $f_i(P_1,...,P_n) = 0$ for each $i$ and so $f_i(\lambda P) = \lambda f_i(P) = 0$. So $P \in X \Longrightarrow\lambda P \in X$ for all $\lambda \in k$. Simiplarly, points of $X$ are closed under addition and so $X$ is a vector space. There thus exists a basis $P^1,...,P^m$ and from these we can construct the following strictly descending chain of irreducible, closed sets:
\begin{align}
    X &\supsetneq X \cap Z(x_1 + \hdots + x_n - P^1_1 + \hdots + P_n^1) \\
    &\supsetneq X \cap Z(x_1 + \hdots + x_n - P^1_1 + \hdots + P_n^1, x_1 + \hdots + x_n - P^2_1 + \hdots + P_n^2) \\
    &\supsetneq \hdots\\
    &\supsetneq Z(\lbrace x_1 + \hdots + x_n - P^i_1 + \hdots + P_n^i\rbrace_{i = 1,...,m})
\end{align}
We can now make a general definition:
\begin{defn}\label{def:dimension_var}
The \textbf{dimension} of a (quasi)-affine variety is the supremum (possibly infinite) of the lengths of all strictly descending chains of irreducible, closed subsets.
\end{defn}
With Definition \ref{def:dimension_var} in mind, a corollary to Proposition \ref{prop:alg_rad} can now be given:
\begin{cor}
Let $X$ be an affine variety. Then
\begin{equation}
    \operatorname{dim}X = \operatorname{dim}A(X)
\end{equation}
\end{cor}\label{cor:dimension}
\begin{proof}
Immediate from Proposition \ref{prop:alg_rad} and the correspondence Theorem.
\end{proof}
We can now set in stone a few loose claims made earlier
\begin{cor}
$$\operatorname{dim}\bb{A}^n = n$$
\end{cor}
\begin{proof}
\begin{equation}
    \operatorname{dim}\bb{A}^n = \operatorname{dim}k[x_1,...,x_n] = n
\end{equation}
\end{proof}
Thus the chain \eqref{eq:dimension_chain} is indeed maximal.

How can this theory be pushed to quasi-affine varieties?
\begin{question}\label{question:quasi}
How does the dimension of a quasi-affine variety relate to the affine variety it is a subset of?
\end{question}
This has a satisfying answer (Corollary \ref{lem:quasi_dimension}). We will use:
\begin{lemma}\label{lem:the_lemma_never_stated}
Let $X$ be any topological space and $U \subseteq X$ any open subset. Then there is a bijection
\begin{align}
    \Phi: \lbrace \text{Irreducible, closed subsets }C \subseteq X, C \cap U \neq \varnothing \rbrace &\lto \lbrace \text{Irreducible, closed subsets }D \subseteq U\rbrace\\
    C &\longmapsto C \cap U
\end{align}
with inverse given by $\Phi^{-1}(D) = \overline{D}$. Moreover, this bijection preserves strict inclusion, that is,
\begin{equation}
    C_1 \subsetneq C_2 \Longleftrightarrow \Phi(C_1) \subsetneq \Phi(C_2)
\end{equation}
\end{lemma}
\begin{proof}
The set $C \cap U$ is an open subset of irreducible $C$ and thus is irreducible (Exercise $1.6$ \cite{hartshorne_solutions}). By the same exercise, the closure of an irreducible set is irreducible, thus we have well defined maps.

We must show $\overline{C \cap U} = C$. Let $W \subseteq X$ be closed and assume $C \cap U \subseteq W$, we show $C \subseteq W$. Since $C \cap U \subseteq W$ we have $C \cap U \subseteq C \cap W$ so taking closures (in $C$) we get $\overline{C \cap U} \subseteq \overline{C \cap W}$. Since $C \cap U$ is an open subset of irreducible $C$, it is thus dense. It follows that $\overline{C \cap U} = C$. Moreover, since $C \cap W$ is closed, we have $\overline{C \cap W} = W$. Thus $C \subseteq W$.

Now we show $\overline{D} \cap U = D$ (where $\overline{D}$ means the closure in $X$). Since $D$ is closed in $U$ there exists $D' \subseteq X$ closed (in $X$) such that $D = D' \cap U$. So we must show $\overline{D' \cap U} \cap U = D' \cap U$. If $x \in \overline{D' \cap U} \cap U$ then $x \in \overline{D'} \cap U$ and so $x \in D'$ but also $x \in U$, so $x \in D' \cap U$. Thus $\overline{D' \cap U} \cap U \subseteq D' \cap U$, the other inclusion is obvious.

For the final claim, say $C_1 \subsetneq C_2$ but $C_1 \cap U = C_2 \cap U$. Then $C_2\setminus C_1 \cap U = \varnothing$. Thus $C_2 = C_1 \cup (U^c \cap C_2)$, contradicting irreducibility of $C_2$.

Conversely, say $D_1 \subsetneq D_2$ but $\overline{D_1} = \overline{D_2}$ (closure taken in $X$). Then $\overline{D_1} \cap U = \overline{D_2} \cap U$. For $i=1,2$ we have $D_i = D_i' \cap U$ for some closed subsets $D_i' \subseteq X$ and so $\overline{D_1' \cap U} \cap U = \overline{D_2' \cap U} \cap U$ and as already seen $\overline{D_i' \cap U} \cap U = D_i' \cap U$ and so $D_1 = D_2$.
\end{proof}
An application of Lemma \ref{lem:the_lemma_never_stated} is to relate the dimension of a space to its open subsets.
\begin{cor}
Let $X$ be any topological space which is covered by open subsets $\lbrace U_i\rbrace_{i \in I}$. Then
\begin{equation}
    \operatorname{dim}X = \operatorname{sup}\operatorname{dim}U_i
\end{equation}
\end{cor}
\begin{proof}
The map $\Phi^{-1}$ of Lemma \ref{lem:the_lemma_never_stated} can be used to relate strictly increasing sequences of irreducible, closed subsets of $U_i$ to such chains in $X$, thus $\operatorname{dim}U_i \leq \operatorname{dim}X$ for all $i$. Furthermore, for any chain $C_0 \subsetneq \hdots \subsetneq C_n$ of irreducible, closed subsets of $X$ there exists some $U_i$ such that $U_i \cap C_0 \neq \varnothing$, and so we get an associated chain in $U_i$ and so $\operatorname{dim}X \leq \operatorname{sup}\operatorname{dim}U_i$.
\end{proof}
We now begin answering Question \ref{question:quasi}, we need the following:
\begin{thm}
\label{thm:onepointeight}
Let $A$ be a finitely generated $k$-integral domain, with $k$ a field. Then
\begin{enumerate}
    \item\label{thm:onepointeighta} $\operatorname{tr.deg}_kA = \operatorname{dim}A$,
    \item\label{thm:onepointeightb} if $\frak{p}$ is any prime of $A$ then $\operatorname{ht.}\frak{p} + \operatorname{dim}A/\frak{p} = \operatorname{dim}A$
\end{enumerate}
\end{thm}
\begin{proof}
See \cite{dimension}
\end{proof}
\begin{cor}\label{lem:quasi_dimension}
Let $Y$ be a quasi-affine variety. Then
\begin{equation}
    \operatorname{dim}Y = \operatorname{dim}\overline{Y}
\end{equation}
\end{cor}
\begin{proof}
Since $Y \subseteq \overline{Y}$ we have $\operatorname{dim}Y \leq \operatorname{dim}\overline{Y}$ (see the solution to Exercise $1.10a$ \cite{hartshorne_solutions} for details), it remains to show the reverse inclusion. Let $C_0 \subsetneq \hdots \subsetneq C_n$ be a chain of irreducible, closed subsets of $\operatorname{dim}Y$, since $\operatorname{dim}Y \leq \operatorname{dim}\overline{Y}$ we know that $\operatorname{dim}Y$ is finite, so assume this chain is of maximal length. By Lemma \ref{lem:the_lemma_never_stated} we have a corresponding chain
\begin{equation}\label{eq:induced_chain}
\overline{C_0} \subsetneq \hdots \subsetneq \overline{C_n}    
\end{equation}
in $\overline{Y}$ which is of maximal length. NB, we do not at this point know that $\operatorname{dim}\overline{Y} = n$ because what if there is another chain in $\overline{Y}$ which has empty intersection with $Y$? To proceed, we use Theorem \ref{thm:onepointeight} (part \ref{thm:onepointeightb}). Since $C_0$ is a singleton set $\lbrace P\rbrace$ say, and $\overline{C_0} = C_0$ the chain \eqref{eq:induced_chain} corresponds to a maximal length sequence of prime ideals of $A(\overline{Y})$ contained in the maximal ideal $\frak{m}_P$ corresponding to the point $P$. Thus, $\operatorname{ht}.\frak{m}_P = n$. Applying the Theorem we now have
\begin{equation}
    n = \operatorname{ht}\frak{m}_p + \operatorname{dim}A(Y)/\frak{m}_P = \operatorname{dim}A(\overline{Y}) 
\end{equation}
The proof will be complete once we have shown $A(Y)/\frak{m}_P \cong k$. This can be seen by considering the map
\begin{align}
    \varphi: A(Y) &\lto k\\
    \overline{x}_i &\longmapsto P_i
\end{align}
where $\overline{x}_i$ is the image of $x_i$ under the map $k[x_1,...,x_n] \lto A(Y)$. The map $\varphi$ is surjective and so induces the required isomorphism.
\end{proof}
\section{Projective varieties}
Projective varieties kill two birds with one stone: any two distinct linear equations will have a point of intersection (even parallel ones) and we can talk rigorously about a concept which is present in ambience when dealing with affine varieties: the different directions toward infinity (which is much more sophisticated than concepts such $\pm \infty$). Hence, projective space plays a roll in Algebraic Geometry similar to what the one point compactification does in topology.
%
\begin{defn}
Consider the set $\bb{A}^{n+1}\setminus\lbrace 0 \rbrace$ modded out by the equivalence relation $x \sim \lambda x$ for any $\lambda \in k\setminus \lbrace 0 \rbrace$. We denote this set $\bb{P}^n$. This is endowed with a topological space but \emph{not} the quotient space topology, instead the closed sets are defined to be such zero sets of homogeneous polynomials (polynomials where all monomials have the same degree). \textbf{Projective $n$-space} is $\bb{P}^n$ along with this topology. (See \cite{hartshorne} for more details).
\end{defn}
Projective space is Noetherian (see solution to Exercise $2.5$ \cite{hartshorne_solutions}) so by Proposition \ref{prop:decomposition} every closed set is a union of irreducible closed sets, thus, just as we did for affine varieties, we define:
\begin{defn}
A \textbf{Projective variety} is an irreducible, closed subset of $\bb{P}^n$. A \textbf{quasi-projective variety} is an open subset of a projective variety.
\end{defn}
\begin{notation}\label{notation:affine_proj}
If extra clarity is needed, given an ideal $\frak{a} \subseteq k[x_0,...,k_n]$, the notation $Z_{\bb{P}^n}(\frak{a})$ will be used for the zero set of $\frak{a}$ in $\bb{P}^n$ and $Z_{\bb{A}^{n+1}}(\frak{a})$ for the zero set $Z(\hat{\frak{a}})$ in $\bb{A}^{n+1}$ where $\hat{\frak{a}} \subseteq \bb{A}^{n+1}$ is defined by
\begin{equation}
    f(x_1,...,x_{n+1}) \in \hat{\frak{a}} \subseteq k[x_1,...,x_{n+1}] \Longleftrightarrow f(x_0,...,x_n) \in \frak{a} \subseteq k[x_0,...,x_n]
\end{equation}
\end{notation}
In practice, projective varieties are dealt with by considering a related affine variety, one way of doing this is the following: given a a non-empty projective variety $\varnothing \neq Z_{\bb{P}^n}(\frak{a}) \subseteq \bb{P}^n$, consider $Z_{\bb{A}^{n+1}}(\frak{a})\subseteq \bb{A}^{n+1}$ (as per Notation \ref{notation:affine_proj}), which consists of all representations of equivalence classes in $Z(\frak{a}) \subseteq \bb{P}^n$ along with, by homogeneity of $\frak{a}$, a new point $0$. One then deals with these two situations separately. See the solution to Exercise $2.1,2.2,2.3$ \cite{hartshorne_solutions} for two examples of this.

Another way of relating a projective variety to a related affine variety is by using the fact that projective varieties are covered homeomorphically by affine varieties, a result the next Section \ref{sec:homog_proj} is dedicated to proving.
\subsection{Homogenisation and projection}\label{sec:homog_proj}
Projective space is covered homeomorphically by affine varieties:
\begin{thm}\label{thm:proj_covered_aff}
\label{projaff}
Let $U_l$ denote the $\bb{P}^n\setminus Z(x_l)$. Then the map
\begin{align*}
    \varphi_l: U_l &\to \bb{A}^n\\
    [(a_1,...,a_{n+1})] &\mapsto \Big(\frac{a_1}{a_i},...,\hat{\frac{a_l}{a_l}},...,\frac{a_{n+1}}{a_i}\Big)
\end{align*}
is a homeomorphism.
\end{thm}
We delay the proof for now. Any non-homogeneous polynomial can be made so as we describe in Definition \ref{def:homogenisation}. This process has an \emph{inverse} (as made precises by Lemma \ref{lem:poly_homog_bij}) called \emph{projection} (Definition \ref{def:projection}). Theorem \ref{thm:proj_covered_aff} can be thought of as the geometric repercussions of these transformations of polynomials.
%
\begin{defn}\label{def:projection}
Let $f \in k[x_1,...,x_n]$ be a polynomial. The \textbf{$i^{\text{th}}$ projection} of $f$, denoted $\operatorname{Proj}_l(f)$ is the polynomial given by setting $x_{i} = 1$ and substituting $x_j$ for $x_{j-1}$ for each $j > i$.

For any ideal $\frak{b} \subseteq k[x_1,...,x_{n+1}]$ let $\operatorname{Proj}_l(\frak{b})$ be the ideal of $k[x_1,...,x_n]$ be given by $f \in \frak{b} \Leftrightarrow \operatorname{Proj}_l(f) \in \operatorname{Proj}_l(\frak{b})$.
\end{defn}
\begin{example}
The $2^{\text{nd}}$ projection of $x_1x_2 + 3x_3^2 \in \bb{C}[x_1,x_2,x_3]$:
\begin{equation}
    \operatorname{Proj}_2(x_1x_2 + 3x_3^2) = x_1 + 3x_2^2
\end{equation}
\end{example}
\begin{defn}
Let $f \in k[x_1,...,x_n]$ be a polynomial of degree $d$. The \textbf{$i^{\text{th}}$ homogenisation} of $f$ is the polynoial in $k[x_0,...,x_n]$ given by
\begin{equation}
    x_i^{\operatorname{deg}f}\big(f(x_0/x_i,...,x_n/x_i)\big)
\end{equation}
where we leave out $x_i/x_i$.
\end{defn}
%
\begin{example}
The $2^{\text{nd}}$ homogenisation of $x_1 + 3x_2^2 \in \bb{C}[x_1,x_2]$:
\begin{equation}
    \operatorname{Hgn_2}(x_1 + 3x_2^2) = x_2^2(x_0/x_2 + 3) = x_0x_2 + 3x_2^2
\end{equation}
\end{example}
\begin{lemma}\label{lem:poly_homog_bij}
The $0^{\text{th}}$ projection and the $0^{\text{th}}$ homogenisation establish a bijection:
\begin{align}
    \Phi: \lbrace \text{Polynomials }p \in k[x_1,...,x_n]\rbrace &\stackrel{\sim}{\lto} \lbrace \text{Homogeneous polynomials }p \in k[x_0,...,x_{n}]\rbrace\\
    f &\longmapsto \operatorname{Hgn}_0(f)
\end{align}
with inverse given by $\Phi^{-1}(f) = \operatorname{Proj}_0(f)$.
\end{lemma}
Clearly, the other projection and homogenisations also establish such a bijection, but for simplicity we work with $0$.
\begin{lemma}
\label{closedsets}
Where $\varphi_l$ is as in Theorem \ref{projaff}, we have for ideals $\frak{a} \subseteq k[x_1,...,x_n]$ and $\frak{b} \subseteq k[x_1,...,x_{n+1}]$:
\begin{itemize}
    \item $\varphi_l(Z(\frak{b})) = Z(\operatorname{Proj}_l(\frak{b}))$
    \item $\varphi_l^{-1}(Z(\frak{a})) = Z(\operatorname{Hgn}_l(\frak{a}))$
\end{itemize}
\end{lemma}
\begin{proof}[Proof of Theorem \ref{projaff}]
$\varphi$ is clearly bijective, and by Lemma \ref{closedsets} is a closed map such that the inverse image of every closed set is closed.
\end{proof}
Notice that $\operatorname{Proj}_0$ and $\operatorname{Hgn}_0$ are respectively $\alpha$ and $\beta$ in \cite[\S 1 Prop 2.2]{hartshorne}. In fact, more can be said, the map $\varphi_l$ of Theorem \ref{thm:proj_covered_aff} is an \emph{isomorphism}, however we have not defined a morphism of varieties yet. In practice, only the fact that $\varphi_l$ is a homeomorphism is needed, for examples, see the solutions to Exercises $2.6,2.7$ \cite{hartshorne_solutions}.
%
%
%
%
%
%
\subsection{Gr\"{o}bner Basis}
Given a set of polynomials $f_1,...,f_m \in k[x_1,...,x_n]$, it was shown in Section \ref{sec:homog_proj} how to obtain a corresponding set of homogeneous polynomials $\beta(f_1),...,\beta(f_m)$. There is a critical subtlety here: if $I$ is the ideal generated by $f_1,...,f_n$ and $\beta(I)$ represents the set of polynomials
\begin{equation}\label{eq:beta_I}
    \beta(I) := \lbrace \beta(f) \mid f \in I\rbrace
\end{equation}
then the ideal generated by $\beta(I)$ is not necessarily equal to $(\beta(f_1),...,\beta(f_m))$, as the following example demonstrates.
\begin{example}\label{ex:beta_induced}
Let $f_1 = x^2 - y$ and $f_2 = x^3 - z$. Then
\begin{equation}
    -x f_1 + f_2 = z - xy
\end{equation}
and so $\beta(z - xy) = zw - xy$ is in the ideal generated by $\beta(I)$ but $zw - xy$ is not in $(\beta(f_1),\beta(f_2)) = (x^2 - wy, x^3 - w^2z)$.
\end{example}
\begin{defn}
A \textbf{monomial order} is a total order $>$ on $\bb{Z}_{\geq 0}^n$ subject to:
\begin{itemize}
    \item if $\alpha, \beta, \gamma \in \bb{Z}_{\geq 0}^n$ and $\alpha > \beta$ then $\alpha + \gamma > \beta + \gamma$,
    \item $>$ is a well ordering, that is, every non-empty subset of $\bb{Z}_{\geq 0}^n$ has a least element.
\end{itemize}
Hence we have the following question:
\begin{question}\label{question:why_grobner}
	How can we find a generating set for the ideal generated by $\beta(I)$?
	\end{question}
\end{defn}\label{def:leading_term}
The monomial order which will be used to resolve the motivating problem given at the start of this section will be \emph{graded lexicographic ordering}, given in Definition \ref{def:graded_lexico}. This definition requires the preliminary notion of \emph{lexicographic ordering}.
\begin{defn}\label{def:lexico}
	Define $(m_1,\hdots, m_n) <_{\operatorname{lex}} (k_1,\hdots, k_n)$ if
	\begin{equation}
		\exists r \leq n, m_r < k_r\text{ and }t < r \Longrightarrow m_t = k_t
		\end{equation}
	\end{defn}
\begin{defn}\label{def:graded_lexico}
	Define $(m_1,\hdots, m_n) <_g (k_1,\hdots, k_n)$ if
	\begin{equation}
		\sum_{i=1}^n m_i < \sum_{i = 1}^n k_i\text{ or } \Big(\sum_{i=1}^r n_i = \sum_{i = 1}^t m_i \text{ and }(m_1, \hdots, m_n) <_{\operatorname{lex}}(k_1,\hdots, k_n)\Big)
		\end{equation}
	\end{defn}
Monomial orders induce total orders on the set of monomials in polynomial rings in the following way: say $<$ is a monomial order on $\bb{Z}_{\geq 0}^n$ and consider the polynomial ring $k[x_1,...,x_n]$. Writing $x_1^{m_1}\hdots x_n^{m_n}$ as $\underline{x}^{(m_1,...,m_n)}$ we define
\begin{equation}
	\underline{x}^{(m_1,..,m_n)} < \underline{x}^{(k_1,...,k_n)} \text{ if }(m_1,\hdots,m_n) < (k_1,\hdots,k_m)
	\end{equation}

\begin{defn}
Fix a monomial order on $k[x_1,...,x_n]$. Denote the \textbf{leading term} (with respect to this monomial order) of a polynomial $f \in k[x_1,...,x_n]$ by $\operatorname{LT}(f)$.

Let $f_1,...,f_n \subseteq k[x_1,...,x_n]$ be a set of polynomials and $I$ the ideal generated by them. The \textbf{ideal of leading terms}, denoted $(\operatorname{LT}(I))$ is the ideal generated by the set
\begin{equation}
    \operatorname{LT}(I) := \lbrace \operatorname{LT}(f) \mid f \in I\rbrace
\end{equation}
\end{defn}
The key observation is if $f_1,...,f_n  \in k[x_1,...,x_n]$ are given and $I$ is the ideal generated by $f_1,...,f_n$, then $\operatorname{LT}(I)$ and $(\operatorname{LT}(f_1),...,\operatorname{LT}(f_n))$ need not be equal. The theory of \emph{Gr\"{o}bner bases} solves this subtlety.
\begin{defn}
Let $f_1,...,f_m$ be a given set of polynomials in $k[x_1,...,x_n]$ and $I$ the ideal they generate. A \textbf{Gr\"{o}bner basis} for $I$ is a set of polynomials $g_1,...,g_r \in k[x_1,...,x_n]$ such that, if $J$ is the ideal generated by $g_1,...,g_r$:
\begin{equation}
    (\operatorname{LT}(J)) = (\operatorname{LT}(g_1),...,\operatorname{LT}(g_r))
\end{equation}
\end{defn}
A question arises immediately: is a Gr\"{o}bner basis for an ideal $I$ necessarily a generating set of $I$? The answer is yes (Lemma \ref{lem:Grobner_is_basis}), the proof of which will use the \emph{division algorithm for polynomials of many variables}, see \cite[\S 2. 3 Theorem 3]{grobner}.
\begin{lemma}\label{lem:Grobner_is_basis}
Let $f_1,...,f_m$ be a Gr\"{o}bner basis for and ideal $I \subseteq k[x_1,...,x_n]$. Then $f_1,...,f_m$ generated $I$.
\end{lemma}
\begin{proof}
By the division algorithm we can write any $f \in I$ as $\sum_{i = 1}^m \alpha_i f_i + r$ for polynomails $\alpha_i,r \in k[x_1,...,x_n]$, where $f$ is a remainder. It follows that $r = f - \sum_{i = 1}^m \alpha_i f_i$ so in particular, $r \in I$. Thus, $\operatorname{LT}r \in (\operatorname{LT}f_1,...,\operatorname{LT}f_m)$ as $f_1,...,f_m$ form a Gr\"{o}bner basis, so $r \neq 0$ contradicts the property of $r$ being a remainder. Thus, $r = 0$.
\end{proof}
Lemma \ref{lem:Dickson} will be used to prove that every ideal $I \subseteq k[x_1,...,x_n]$ admits a Gr\"{o}bner basis.
\begin{notation}
A monomial in $k[x_1,...,x_n]$ will be denoted $x^\alpha, x^\beta,\hdots$ where $\alpha,\beta,\hdots$ are elements of $\bb{Z}_{\geq 0}^n$.
\end{notation}
\begin{defn}
An ideal $I \subseteq k[x_1,...,x_n]$ is a \textbf{monomial ideal} if it is generated by monomials.
\end{defn}
The ring $k[x_1,...,x_n]$ is Noetherian and so every ideal is finitely generated, Dickson's Lemma states that every monomial ideal is generated by finitely many monomials:
\begin{lemma}[Dickson's Lemma]\label{lem:Dickson}
For every monomial ideal $I \subseteq k[x_1,...,x_n]$, there exists finitely many monomials $x^{\alpha_1},...,x^{\alpha_n}$ which generate $I$.
\end{lemma}
\begin{proof}
We proceed by induction on $n$. If $n = 1$ then $k[x_1]$ is a PID so $I = (f)$ for some $f$ which must be a monomial as $f$ divides each of the monomials which generate $I$ and so $f$ is a monomial itself.

Now say the result holds true for all $k[x_1,...,x_{n-1}]$ where $n > 1$. Let $I$ be generated by a set of monomials $\lbrace x^\alpha \mid \alpha \in A \rbrace$. Consider the following collection of ideals indexed by the natural numbers $m \geq 0$:
\begin{equation}
    J_m := \lbrace f \in k[x_1,...,x_{n-1}] \mid \exists \alpha \in A, x^\alpha x_n^{m} = f\rbrace
\end{equation}
The ideals $J_m$ are generated by the monomials $x^\alpha$ for which $x^\alpha x_{n}^m \in I$ and so are themselves monomial ideals. By the inductive hypothesis, for each $m$ there exists finitely many monomials $x^{\alpha_1^m},...,x^{\alpha_{r(m)}^m} \in k[x_1,...,x_{n-1}]$ which generate $J_m$. We finish the proof by showing that the ideal $I$ is generated by the following finite set of monomials:
\begin{equation}
\scr{S} := \lbrace x^{\alpha_1^m}x_n^m,...,x^{\alpha_{r(m)}^m}x_n^m \mid m \geq 0\rbrace
\end{equation}
By construction, $\scr{S} \subseteq I$. Conversely, every $f \in I$ is a linear combination of monomials in $\lbrace x^\alpha \mid \alpha \in A \rbrace$. Each of the monomials in this linear combination must be divisible for some $x^\alpha$ in some $J_m$ by $x^\alpha x_n^m$, again by construction of $\scr{S}$.
\end{proof}
\begin{cor}\label{cor:grobner_basis_existence}
Every monomial ideal $I$ admits a Gr\"{o}bner basis.
\end{cor}
\begin{proof}
The proof is simply the observation that $(\operatorname{LT}(I))$ is a monomial ideal, now apply Dickson's Lemma \ref{lem:Dickson}.
\end{proof}
\begin{question}\label{question:Grobner_check}
Given a set of generators $f_1,...,f_m$ of an ideal $I$, how do we check if $f_1,...,f_m$ form a Gr\"{o}bner basis for $I$?
\end{question}
This question is answered by performing an analysis on the possible obstructions which cause a generating set to not be a Gr\"{o}bner basis. This analysis is written out very well in \cite[\S 2]{grobner} and so we do not repeat the discussion here. The answer to Question \ref{question:Grobner_check} which uses polynomial long division of multiple variables (with respect to a monomial ordering) is given by \cite[\S 2.3]{grobner}.
\begin{defn}
Fix a monomial ordering. Let $f_1,f_2 \in k[x_1,...,x_n]$ be two polynomials and let $\alpha = (\alpha_1,...,\alpha_n)$ and $\beta = (\beta_1,...,\beta_n)$ be their respective degrees. Let $\gamma = (\operatorname{max}\lbrace \alpha_1,\alpha_2\rbrace,\hdots,\operatorname{max}\lbrace \alpha_n,\beta_n\rbrace)$, the \textbf{least common multiple} of $f$ and $g$ is $x^\gamma$.

The \textbf{S-polynomial} of $f,g$ is
\begin{equation}
    S(f,g) = (x^\gamma/\operatorname{LT}(f))f - (x^\gamma/\operatorname{LT}(g))g
\end{equation}
\end{defn}
\begin{lemma}\label{lem:suffices_S_poly}
Let $I \subseteq k[x_1,...,x_n]$ be an ideal. A generating set $f_1,...,f_m$ of $I$ is a Gr\"{o}bner basis for $I$ if and only if for all pairs $i \neq j$, the remainder on division of $S(f_i,f_j)$ by $f_1,...,f_m$ is zero.
\end{lemma}
\begin{proof}
See \cite[\S 2.6 Theorem 6]{grobner}.
\end{proof}
\begin{defn}
	The polynomial ring $k[x_1,...,x_n]$ is graded, that is, if $S_d$ is the abelian group generated by all weight $d$ monomials in $k[x_1,...,x_n ]$, then
	\begin{equation}
		k[x_1,...,x_n] \cong \bigoplus_{d \geq 0}S_d
		\end{equation}
	A polynomial $f \in k[x_1,...,x_n]$ has \textbf{degree $d$} if the maximum of all the weights of all the terms in $f$ is equal to $d$.
	\end{defn}

\begin{algorithm}\label{alg:division_alg}
	\caption{The division algorithm}\label{alg:division}
	\begin{algorithmic}
		\Require $(f_1,\hdots,f_s),f$, returns the result of dividing $f$ by $(f_1,\hdots,f_s)$.
		\State $q_1,\hdots, q_s \gets 0,\hdots 0$
		\State $p \gets f$
		\While{$p \neq 0$}
		\State $\text{DivOcc} \gets \texttt{False}$
		\State $i \gets 1$
		\While{$i \leq s \text{ and } \text{DivOcc} = \texttt{false}$}
		\If{$\operatorname{LT}f_i | \operatorname{LT}p$}
		\State $q_i \gets \operatorname{LT}f_i/\operatorname{LT}p$
		\State $p \gets p - (\operatorname{LT}f_i/\operatorname{LT}p)f_i$
		\State $\text{DivOcc} \gets \texttt{True}$
		\Else 
		\State $i \gets i + 1$
		\EndIf
		\If{$\text{DivOcc} = \texttt{False}$}
		\State $r \gets r + \operatorname{LT}p$
		\State $p \gets p - \operatorname{LT}p$
		\EndIf
		\EndWhile
		\EndWhile\\
		\Return{$(q_1,...,q_s, r)$}
	\end{algorithmic}
\end{algorithm}
Corollary \ref{cor:grobner_basis_existence} is relevant to Example \ref{ex:beta_induced} by way of Corollary \ref{cor:relate_beta_grobner} to the following lemma:
\begin{lemma}\label{lem:grobern_betagrobner}
	Fix an ordering on the set of indeterminants $x_1 > \hdots > x_n$ and consider the graded lexicographic monomial ordering (Definition \ref{def:graded_lexico}) induced by this. We know this induces a total order on the set of monomials of the polynomial ring $k[x_1,...,x_n]$. Similarly, we consider the order on the set of monomials of $k[x_1,...,x_n,x_0]$ induced by $x_1 > \hdots > x_n > x_0$. If $\{f_1,\hdots,f_m\} \subseteq k[x_1,\hdots,x_n]$ are a Gr\"{o}bner basis for the ideal generated by the set $\{f_1, \hdots, f_m\}$, then $\{\beta f_1, \hdots, \beta f_m\} \subseteq k[x_1,\hdots, x_n, x_0]$ is a Gr\"{o}bner basis for the ideal generated by the set $\{ \beta f_1, \hdots, \beta f_m\}$.
\end{lemma}
\begin{proof}
	The general idea of the proof is to run two instances of the division algorithm ``side by side" and observe the relationships between the two. The first of these will be the division of $f$ by the sequence $(f_1,...,f_m)$ and the second of these will be the division of $\beta f$ by the sequence $(\beta f_1, ..., \beta f_m)$. We will see that if $i \leq m$ is such that division by $f_i$ occurrs at step $j$ of the former instance of the division algorithm, then division by $\beta f_i$ will occurr also at step $j$ in the later instance of the division algorithm.
	
	Lastly, we will show that if the division at step $j$ in the former instance results in a remainder of $0$, then the same is true for the later instance.
	
	Now we proceed with the formal proof. To avoid exploding parentheses we write $\beta f_i$ for $\beta(f_i)$ and $\operatorname{LT}f$ for $\operatorname{LT}(f)$.
	
	Since $(f_1,\hdots,f_m)$ form a Gr\"{o}bner basis for the ideal that they generate, we know that for any $i,j \leq m$, division of $S(f_i,f_j)$ by $(f_1,...,f_m)$ results in a remainder of $0$. That is, division occurrs at every step of the division algorithm.
	
	We establish some notation for the two instances of the Division Algorithm which this proof concerns.
	
	The status of the polynomial labelled $p$ in Algorithm \ref{alg:division_alg} at step $j$ by $p_j$, where Algorithm \ref{alg:division_alg} is performed with input $(f_1,...,f_m,f)$. The previous comment implies that at step $j$ of the division algorithm we have the following assignment of the polynomial $p_j$, where $i_j$ is the least integer such that $\operatorname{LT}f_{i_j} \mid \operatorname{LT}p_j$:
	\begin{equation}
		p_{j + 1} = 
		\begin{cases}
			S(f_k, f_l) & j = 0\\
			p_j - (\operatorname{LT}p_j/\operatorname{LT}f_{i_j})f_{i_j}& j > 0
			\end{cases}
		\end{equation}
	The graded lexicographic orderings given in the statement of the Lemma is subject to the property that any polynomial $g \in k[x_1,...,x_n]$ satisfies
	\begin{equation}\label{eq:graded_feature}
		\operatorname{LT}\beta g = \operatorname{LT}g
		\end{equation}	
	This is because the leading monomial $\operatorname{LM}g$ has total weight equal to the degree of $g$. We remark that this is where \emph{graded} lexicographic ordering must be used, a similar statement to \eqref{eq:graded_feature} does \emph{not} hold for lexicographic ordering.
	
	Now we introduce a new instance of the division algorithm. The status of $p$ at step $j$ of Algorithm \ref{alg:division_alg} when run on the input $(\beta f_1,\hdots, \beta f_m, \beta f)$ will be denoted $\hat{p}_j$.
	
	 We can use \eqref{eq:graded_feature} can now be used to simplify the defining expression for $\hat{p}_j$, in what follows, $v_j$ is the least integer such that $\operatorname{LT}f_{v_j} \mid \operatorname{p_j}$ (in fact, we will see later in this proof that $v_j = i_j$ for all $j$, but for now we cannot assume this):
	\begin{equation}
		\hat{p}_{j+1} =
		\begin{cases}
			S(\beta f_k, \beta f_l) & j = 0\\
			\hat{p}_j - (\operatorname{LT}\hat{p}_j/\operatorname{LT}f_{v_j})\beta f_{v_j} & j > 0
		\end{cases}
		\end{equation}
	The most important claim for the entire proof is the following: for each $j$, define the integer $s_j$ to be such that
	\begin{equation}
		x_0^{s_j} = \operatorname{LM}\hat{p}^j/\operatorname{LM}p_j
		\end{equation}
	then the following holds
	\begin{equation}
		\exists r_j \geq 0,\text{ }x_0^{r_j} \beta p_j = \hat{p}_j
		\end{equation}
	We prove the claim by induction on $j$. If $j = 0$ then we must show
	\begin{equation}
		\exists r_0 \geq 0,\text{ }x_0^{r_0}\beta S(f_k,f_l) = S(\beta f_k, \beta f_l)
		\end{equation}
	Let $\gamma = \operatorname{LCM}(\operatorname{LM}\beta f_k, \operatorname{LM}\beta f_l)$. Also by \eqref{eq:graded_feature} we have $\operatorname{LT}\beta f_k = \operatorname{LT} f_k$ and similarly with $l$ in place of $k$. Hence we have the following.
	\begin{equation}\label{eq:first_step}
		S(\beta f_k, \beta f_l) = (x^\gamma/ \operatorname{LT}f_k)\beta f_k - (x^\gamma/\operatorname{LT}f_l)\beta f_l
		\end{equation}
	For any indeterminant $z \in \{x_1,\hdots,x_n,x_0\}$ and any polynomial $g \in k[x_1,\hdots, x_n, x_0]$ we have
	\begin{equation}
		z\beta g = \beta (zg)
		\end{equation}
	Hence \eqref{eq:first_step} becomes
	\begin{equation}\label{eq:second_step}
		\beta\big((x^\gamma / \operatorname{LT}f_k)f_k\big) - \beta\big((x^\gamma/\operatorname{LT}f_l) f_l\big)
		\end{equation}
	For any pair of polynomials $g,h \in k[x_1,\hdots, x_n, x_0]$ of the same degree we have that there exists $u \geq 0$ such that
	\begin{equation}
		\beta g - \beta h = x_0^u \beta(g - h)
		\end{equation}
	Hence \eqref{eq:second_step} implies there exists $r_0 \geq 0$ such that
	\begin{equation}\label{eq:third_step}
		S(\beta f_k, \beta f_l) = x_0^{r_0}\beta\big((x^\gamma/\operatorname{LT}f_k)f_k) - (x^\gamma/\operatorname{LT}f_l)f_l\big)
		\end{equation}
	Finally, we observe that by \eqref{eq:graded_feature} we have $\operatorname{LCM}(\operatorname{LM}\beta f_k, \operatorname{LM}\beta f_l) = \operatorname{LCM}(\operatorname{LM} f_k, \operatorname{LM} f_l)$ and so
	\begin{equation}
		(x^\gamma/\operatorname{LT}f_k)f_k) - (x^\gamma/\operatorname{LT}f_l)f_l = S(f_k, f_l)
		\end{equation}
	Combining this with \eqref{eq:third_step} establishes the base case.
	
	Now say $j > 0$. Notice that the inductive hypothesis implies that $i_j = v_j$. Then we have the following calculation.
	\begin{align}
		\hat{p}_{j+1} &= \hat{p}_j - (\operatorname{LT}\hat{p}_j/\operatorname{LT}f_{i_j})\beta f_{i_j}\\
		&= x_0^{r_j}\beta p_j - \beta\big((\operatorname{LT}\hat{p}_j/\operatorname{LT}f_{i_j}) f_{i_j}\big)\\
		&= x_0^{r_j}\beta p_j - x_0^{r_j}\beta\big((\operatorname{LT}p_j/\operatorname{LT}f_{i_j}) f_{i_j}\big)\\
		&= x_0^{r' + r_j}\beta\big(p_j - (\operatorname{LT}p_j/\operatorname{LT}f_{i_j})f_{i_j}\big),\text{ for some }r'\geq 0\\
		&= x_0^{r_{j+1}}\beta p_{j+1},\text{ where }r_{j + 1} := r' + r_j
		\end{align}
	Establishing the claim.
	
	It now follows directly from the claim that $\hat{p}_j = 0$ if and only if $p_j = 0$, this proves the Lemma.
\end{proof}
\begin{cor}\label{cor:relate_beta_grobner}
Let $f_1,...,f_m$ be generators of an ideal $I \subseteq k[x_1,...,x_n]$. If $f_1,...,f_m$ is a Gr\"{o}bner basis, then $\beta f_1,...,\beta f_m$ generate the ideal generated by $\beta(I)$.
\end{cor}
\begin{proof}
By Lemma \ref{lem:grobern_betagrobner} and Lemma \ref{lem:Grobner_is_basis}.
\end{proof}
As an application we can solve Question \ref{question:why_grobner}: if $(f_1,...,f_m)$ is a generating set for $I$, then extend this to a Gr\"{o}bner basis $(f_1,...,f_{m'})$ with respect to graded lexicographic ordering induced by $x_1 > \hdots > x_n$. Then $\beta f_1, \hdots, \beta f_{m'}$ is a generating set for the ideal generated by $\beta(I)$.
\subsection{Segre embedding}
We define the \textbf{Segre embedding}:
\begin{align*}
    \psi: \bb{P}^r \times \bb{P}^s &\lto \bb{P}^{r + s + rs}\\
    (P,Q) &\longmapsto [P_0Q_0:...:P_0Q_s:......:P_rQ_0:...:P_rQ_s]
\end{align*}
\begin{defn}
We define a function $\theta: k[\lbrace z_{ij}\rbrace_{0 \leq i \leq r, 0 \leq j \leq s}] \lto k[x_0,...,x_r,y_0,...,y_s]$ which maps $z_{ij} \longmapsto x_iy_j$
\end{defn}
\begin{lemma}\label{lem:theta_unique_existence}
Let $R \in \bb{P}^{r + s + rs}$. Then
\begin{equation}
    P \in Z(\operatorname{ker}\theta) \Longrightarrow \exists ! (Q,R) \in \bb{P}^r \times \bb{P}^s\text{ st }\psi(Q,R) = P
\end{equation}
\end{lemma}
\begin{proof}
First we show existence. In particular, $P$ is a root of every polynomial of the form $z_{ij}z_{kl} - z_{il}z_{kj}$, where $0\leq i,k\leq r$ and $0\leq j,l\leq s$. Let $\lbrace P_{ij}\rbrace$ be a set of homogeneous coordinates for $P$ and now fix a pair of integers $(a,b)$ such that $P_{ab} \neq 0$. For all $0 \leq k\leq r$ and all $0 \leq j \leq s$ we have $P_{aj}/P_{ab} = P_{kj}/P_{kb}$ which implies:
\[\frac{P_{aj}}{P_{ab}}P_{kb} = P_{kj}\]
Thus we can recover all $P_{kj}$ from the set $\lbrace P_{a0},...,P_{as},P_{0b},...,P_{rb}\rbrace$. We write $P$ as
\[P = \Big[\frac{P_{aj}}{P_{ab}}P_{kb}\Big]_{0 \leq k \leq r, 0 \leq j \leq s} = \psi\Big(\big[P_{0b}:...:P_{rb}\big], \big[\frac{P_{a0}}{P_{ab}}:...:\frac{P_{as}}{P_{ab}}\big]\Big)\]

For uniqueness, say $(P,Q),(P',Q') \in \bb{P}^r \times \bb{P}^s$ were such that $\psi(P,Q) = \psi(P',Q')$. Write
\begin{align*}
    \psi(P,Q) &= [P_0Q_0:...:P_0Q_s:......:P_rQ_0:...:P_rQ_s]\\
    &= [P_0'Q_0':...:P_0'Q_s':......:P_r'Q_0':...:P_r'Q_s'] = \psi(P',Q')
\end{align*}
and let $\lambda \neq 0$ be such that 
\begin{equation}\label{eq:scaler}
(P_0Q_0:...:P_0Q_s:......:P_rQ_0:...:P_rQ_s) = \lambda (P_0'Q_0':...:P_0'Q_s':......:P_r'Q_0':...:P_r'Q_s')
\end{equation}
From the above, there exists pairs of integers $(a,b),(a',b')$ such that
\begin{equation}\label{eq:image_type}
    \frac{P_{a}Q_j}{P_{a}Q_b}P_kQ_b = P_kQ_j\qquad\text{and}\qquad \frac{P_{a'}'Q_j'}{P_{a'}'Q_b'}P_k'Q_b' = P_k'Q_j'
\end{equation}
Thus for all $0 \leq k \leq r, 0 \leq j \leq s$:
\begin{align*}
    P_kQ_j &= \frac{P_{a}Q_j}{P_{a}Q_b}P_kQ_b & \text{by }\eqref{eq:image_type}\\
    &= \frac{\lambda P'_{a'}Q'_j}{P_{a}Q_b}\lambda P'_kQ_{b'}' & \text{by }\eqref{eq:scaler}\\
    &= \lambda^2 \frac{P_{a'}'Q_{b'}'}{P_{a}Q_{b}} \Big(\frac{P_{a'}'Q_j'}{P_{a'}'Q_{b'}'}P_k'Q_{b'}'\Big)\\
    &= \lambda^2 \frac{P_{a'}'Q_{b'}'}{P_{a}Q_{b}} P_k'Q_j' & \text{by }\eqref{eq:image_type}
\end{align*}
proving $(P,Q) = (P',Q')$.
\end{proof}
\begin{cor}\label{cor:segre_image_kernel}
$\operatorname{im}\psi = Z(\operatorname{ker}\theta)$.
\end{cor}
\begin{proof}
It remains to show $Z(\operatorname{ker}\theta) \subseteq \operatorname{im}\psi$ which is trivial. Notice that this does not use the uniqueness claim of Lemma \ref{lem:theta_unique_existence}.
\end{proof}
\begin{cor}\label{cor:segre_injectivity}
The Segre embedding $\psi$ is injective.
\end{cor}
\begin{proof}
By the uniqueness claim of Lemma \ref{lem:theta_unique_existence} and Corollary \ref{cor:segre_image_kernel}.
\end{proof}
Thus, for any $R \in \operatorname{im}(\psi)$ we can choose a lift $(P,Q) \in \bb{P}^r \times \bb{P}^s$ along $\psi$ and integers $(a,b)$ such that $P_aQ_b \neq 0$ such that
\begin{equation}
    ([P_{0}Q_b:...:P_{r}Q_b],[\frac{P_aQ_0}{P_aQ_b}:...:\frac{P_aQ_s}{P_aQ_b}]) = (P,Q)
\end{equation}
which by Lemma \ref{cor:segre:injectivity} defines a well defined function $\psi^{-1}: \operatorname{im}\psi \lto \bb{P}^r \times \bb{P}^s$, which indeed is inverse to $\psi$.
\begin{lemma}\label{lem:segre_continuous}
The Segre embedding is continuous and closed.
\end{lemma}
\begin{proof}
Since $\psi$ is a bijection, we have that $\psi(\cap_{i \in I}U_i) = \cap_{i \in I}\psi(U_i)$ for all collections of closed sets $U_i \subseteq \bb{P}^r \times \bb{P}^s$, thus to prove that the Segre embedding is closed it suffices to show that the image of closed sets in a basis of closed sets for $\bb{P}^r \times \bb{P}^s$ are closed.

Towards this end, let $Z(\frak{a}) \times Z(\frak{b}) \subseteq \bb{P}^r \times \bb{P}^s$ be closed. Write $\frak{a} = (f_1,...,f_n), \frak{b} = (g_1,...,g_m)$ and $\operatorname{deg}f_i = d_i, \operatorname{deg}g_i = e_i$. Consider the following for $i = 1,...,r, j = 1,..., m$:
\begin{equation}\label{eq:first_homog}
    x_i^{e_j}g_j(y_0,...,y_s)
\end{equation}
which, since $g_j$ is homogeneous, can be written as
\begin{equation}
    h_{ij}(x_0y_0,...,x_ry_s)
\end{equation}
for some homogeneous polynomial $h_{ij}$. Similarly, for $i' = 1,...,m, j' = 1,...,n$ we write
\begin{equation}\label{eq:second_homog}
    y_{i'}^{e_j'}f_{j'}(x_0,...,x_r) = l_{i'j'}(x_0y_0,...,x_ry_s)
\end{equation}
for some homogeneous polynomial $l_{i'j'}$. We now let
\begin{equation}
    \frak{c} := \Big(\lbrace h_{ij}(z_{00},...,z_{rs})\rbrace_{i = 1,...,r, j = 1,...,m}, \lbrace l_{i'j'}(z_{00},...,z_{rs})\rbrace_{i' = 1,...,n, j' = 1,...,s}\Big) \subseteq k[z_{00},...,z_{rs}]
\end{equation}
We claim:
\begin{equation}
    \psi\big(Z(\frak{a}) \times Z(\frak{b})\big) = Z(\frak{c}) \cap \operatorname{im}\psi
\end{equation}
The inclusion $\subseteq$ is clear. For the converse, if $R \in Z(\frak{c}) \cap \operatorname{im}\psi$ then in particular, $R \in \operatorname{im}\psi$ and so there exists $(P,Q) \in \bb{P}^r \times \bb{P}^s$ such that $\psi(P,Q) =R$. Moreover, since $R \in Z(\frak{c})$ we have that $\psi(P,Q)$ satisfies the polynomials \eqref{eq:first_homog}, \eqref{eq:second_homog} and we cannot have that all $x_i = 0$ nor that all $y_{i'} = 0$ which means all $g_j$ vanish on $Q$ and all $f_{j'}$ vanish on $P$. Thus $(P,Q) \in Z(\frak{a}) \times Z(\frak{b})$, which proves the Segre embedding is closed.


\end{proof}
\begin{remark}
The main point in the proof of Lemma \ref{lem:segre_continuous} is that we want to relate the relations on $(P,Q)$ to the relations of all products of the components of $P$ and $Q$. It is a simple idea: if $f(P) = 0$ for some polynomial $f$, then $(y_jf)(P,Q_j) = 0$, now raise $y_j$ to an appropriate power so that $y_jf$ can be rewritten in terms of $z_{ij}$.
\end{remark}
\begin{cor}\label{cor:homeo}
The Segre embedding is a homeomorphism onto its image.
\end{cor}
\begin{proof}
By nature of being a continuous, closed, injective map.
\end{proof}
\begin{cor}
If $X,Y$ are quasi-projective varieties then $\psi(X, Y)$ is quasi-projective.
\end{cor}
\begin{proof}
By Corollary \ref{cor:homeo} the map $\psi$ is injective, and by Lemma \ref{lem:segre_image_kernel} the image of an open set is a subset of a projective variety.
\end{proof}
\begin{defn}
Let $X$ be a topological space which is a topological product of two varieties, write $X = X_1 \times X_2$. Let $V \subseteq X$ be an open subset and $\varphi: V \lto k$ a function. The map $\varphi$ is \textbf{``regular"} at a point $P \in V$ if there exists an open neighbourhood $U_P \subseteq V$ of $P$ and polynomials $f_P,g_P$ such that $\varphi\restriction_{U_P} = f_P/g_P$.

If $Y$ is a variety and $\psi: X \lto Y$ a continuous function, then $\psi$ is a \textbf{``morphism"} if for every open subset $V \subseteq Y$ and ``regular" function $\varphi: V \lto k$ the function $(\varphi \psi)\restriction_{\psi^{-1}(V)}$ is ``regular".
\end{defn}
\begin{lemma}
The Segre embedding $\psi$ is a ``morphism".
\end{lemma}
\begin{proof}
The key point is that the Segre embedding multiplies entries and so composes with a polynomial to yield a polynomial.
\end{proof}


\section{Morphisms}
To establish a \emph{category of varieties} we need to define a \emph{morphism} of varities. These will be continuous functions which preserve \emph{regular functions}.
\subsection{Regular functions}
Later, when we study singularity theory, the following definition will be central:
\begin{defn}
Let $X = Z(f_1,...,f_m)$ be an affine variety of dimension $n$ and $P \in X$ a point on $X$. Then $X$ is \textbf{nonsingular} at $P$ if the \emph{jacobian matrix} $||(\partial f_i/\partial x_j)P ||$ has rank $n - m$, otherwise $X$ is \textbf{singular} at $P$. The affine variety $X$ is \textbf{nonsingular} if it is nonsingular at all of its points, otherwise it is \textbf{singular}.
\end{defn}
It will be shown later (Lemma \ref{}) that this definition is independent of the choice of $f_i$ taken as generators of $X$, however it is not clear that this definition is invariant under isomorphism. In fact, a notion of singularity at a point $P$ can (and will) be defined independent of the embedding of $X$, and in the special case of algebraic curves a definition will be given in terms of the \emph{local ring} $\call{O}_P$. Thus the notion of regular functions defined on quasi-affine varieties is central, so we need an appropriate definition.
\begin{defn}
Let $X$ be a quasi-affine variety. Then a function $f: X \lto \bb{A}^1$ is \textbf{regular} at $P \in X$ if there exists an open neighbourhood $U$ containing $P$ and polynomials $g,h$ with $h$ nowhere zero on $U$ such that $f\restriction_U = g/h$. The function $f$ is \textbf{regular} if it is regular at every point $X$.
\end{defn}
Any two regular functions which are equal on an open subset of $X$ are equal on all of $X$, however, it is \emph{not} the case that all regular functions are rational functions. Say $f$ is regular and $f\restriction_U = g_1/h_1$ and $f\restriction_V = g_2/h_2$. For $f$ to be well defined it must be that $f\restriction_{U \cap V} = g_1/h_1 = g_2/h_2$, but it may be the case that $h_1$ is undefined in $V$ and $h_2$ is undefined in $U$, and that there is no other rational function which is defined in both of $U$ and $V$ which equals $f$ on $U \cap V$.
\begin{remark}
In fact it may be the case that one has a regular function $f$ and rational functions $g_i/h_i$ ($i = 1,2$) with $g_i/h_i = f\restriction_{U_i}$ for some open sets $U_i$, with $h_1$ not defined on $U_2$, $h_2$ not defined on $U_1$ but \emph{still} there is a global, rational representation of $f$. For example, consider the affine variety defined by $x^2 + y^2 = 1$. We have that
\begin{align*}
    x^2 + y^2 = 1 &\Longrightarrow x^2 = 1 - y^2\\
    &\Longrightarrow x^2 = (1+ y)(1 - y)\\
    &\Longrightarrow x/(1-y) = (1+y)/x
\end{align*}
thus the function $f$ given by $x/(1-y)$ on $y \neq 1$ and $(1+y)/x$ on $x \neq 0$ is a well defined regular function, but is so on an affine variety, which we will see later necessarily admits a polynomial representative. Notice this shows something stronger than the claim, that not only can a global, rational representation of $f$ be found, but a \emph{polynomial} representation may be.
\end{remark}
Considering now an affine variety $X \subseteq \bb{A}^n$, the algebraic object $A(X)$ consists of all polynomials $k[x_1,...,x_n]$ modulo the equivalence relation $\sim$ which identifies polynomials whose induced functions on $X$ are equal. Thus we have an injective function:
\begin{equation}\label{eq:alpha}
    \alpha: A(X) \lto \call{O}(X)
\end{equation}
which in fact we will show is surjective, showing that all regular functions defined on affine varieties admit a global, polynomial representative! First, notice that if $f \in k[x_1,...,x_n]$ is a polynomial, where we denote its image in $A(X)$ by $\bar{f}$, and $P = (P_1,...,P_n) \in X$ is a point such that $\bar{f}(P) \neq 0$ then $\alpha(\bar{f})$ is a unit with inverse $1/(\alpha(\bar{f})\restriction_{Z(f)^c})$. Thus, we obtain a map
\begin{equation}\label{eq:isolocalring}
    A(X)_{(x_1 - P_1,...,x_n - P_n)} \lto \call{O}_{P}
\end{equation}
which, since $\alpha$ is injective and $A(X)$ is an integral domain, is injective. Moreover, this is surjective by definition of a regular function.

More can be said. For each $P \in X$ let $\frak{m}_P$ denote the maximal ideal of $\call{O}_P$, which is consists of the germs of regular functions which vanish at $P$. There is a family of homomorphisms
\[A(X) \stackrel{\alpha}{\rightarrowtail} \call{O}(X) \rightarrowtail \call{O}_P \to \call{O}_P/\frak{m}_P\]
which are injective as labelled. The kernel of this map is $I:= (x_1 - P_1,...,x_n - P_n)$. Thus
\begin{equation}\label{eq:string}
    A(X)/I \cong \call{O}_P/{\frak{m}_P}
\end{equation}
Moreover, by considering the evaluation function $\call{O}_P \lto k$ it is easy to see that $\call{O}/\frak{m}_P \cong k$. Combining this and \eqref{eq:string} we have:
\begin{align*}
    \operatorname{dim}\call{O}_P &= \operatorname{ht}.\frak{m}_P\\
    &= \operatorname{ht}.I\\
    &= \operatorname{ht}.I + \operatorname{dim}k\\
    &= \operatorname{ht}.I + \operatorname{dim}\big(A(X)/I\big)\\
    &= \operatorname{dim}A(X)
\end{align*}
Now, any nonzero element $f \in A(X)$ maps under $\alpha$ to a unit with inverse $1/f\restriction_{Z(f)^c}$. Thus we obtain an injective map
\[\operatorname{Frac}A(X) \rightarrowtail K(X)\]
In fact, this map is also surjective: for each nonzero $[(U,f)] \in K(X)$ we have that $[(U,f)] \in \call{O}_P$ for some $P$. This follows by the already established isomorphism \eqref{eq:isolocalring} and the fact that the following diagram commutes:
\[
\begin{tikzcd}
A(X)_{I}\arrow[r,"{\sim}"]\arrow[d,rightarrowtail] & \call{O}_P\arrow[d,rightarrowtail]\\
\operatorname{Frac}A(X)\arrow[r] & K(X)
\end{tikzcd}
\]
Again, more can be said. We recall the general facts that if $B$ is an integral domain which is also a $k$-algebra then $\operatorname{dim}B$ and $\operatorname{tr.deg}_kB$ yield the same integer, and that $\operatorname{dim}X = \operatorname{dim}A(X)$. Hence $K(X)$ is an algebraic extension of $k$ with transcendence degree equal to $\operatorname{dim}X$. Furthermore, $A(X)$ is a finitely generated $k$-algebra, and so the extension $k/K(X)$ is finitely generated.

Finally, we show that $\alpha$ of \eqref{eq:alpha} is surjective. We have that
\begin{align}
    A(X) &\subseteq \call{O}(X)\\
    &\subseteq \bigcap_{P \in X}\call{O}_P\\
    &\subseteq \bigcap_{\frak{m}}A(X)_\frak{m}
\end{align}
where the last subset inequality is ranging over all maximal ideals $\frak{m}$ of $A(Y)$ (here we have also used the fact that points of $X$ are in one to one correspondence with maximal ideals of $A(X)$, this fact is easily proven and shown explicitly in Hartshorne).

Surjectivity now follows from the general fact that for an integral domain $B$ we have $\bigcap_{\frak{m}}B_\frak{m} = B$ (where both are considered inside the field of fractions). In summary:

\begin{proposition}
\label{prop:affine_basics}
Let $X$ be an affine variety, then
\begin{enumerate}
    \item $A(X) \cong \call{O}(X)$,
    \item for any point $P = (P_1,...,P_n) \in X$ we have $A(X)_{(x_1 - P_1,...,x_n - P_n)} \cong \call{O}_P$ and $\operatorname{dim}\call{O}_P = \operatorname{dim}X$,
    \item $\operatorname{Frac}A(X) \cong K(X)$ and the transcendence degree of the finitely generated extension $K(X)/k$ is equal to $\operatorname{dim}(X)$.
\end{enumerate}
\end{proposition}
\subsection{Morphisms}
We now work in the generality of arbitrary varieties:
\begin{defn}
A \textbf{variety} is one of an affine variety, a quasi-affine variety, a projective variety, or a quasi-projective variety. A \textbf{morphism of varieties} $f: X \lto Y$ is a continuous function such that for all $P \in X$ there exists an open neighbourhood $U$ of $f(p)$ such that for all regular functions $\varphi: U \lto k$ the function $\varphi \circ f$ is regular. The category of varieties and morphisms of varieties is denoted $\underline{\operatorname{Var}}$.
\end{defn}
Let $f: X \lto Y$ be a morphism with $X$ an arbitrary variety and $Y$ an affine variety. Then there is an induced map
\[\hat{f}: \call{O}(Y) \lto \call{O}(X)\]
given by precomposing with $f$. We have also already seen that $\call{O}(Y) \cong A(Y)$ (Proposition \ref{prop:affine_basics}), the resulting map $A(Y) \lto \call{O}(Y)$ is a $k$-algebra homomorphism, and so we have described a function
\begin{equation}
    \alpha: \operatorname{Hom}_{\operatorname{Var}}(X,Y) \lto \operatorname{Hom}_{k\operatorname{Alg}}(A(Y),X)
\end{equation}
We now describe an inverse to $\alpha$. Let $f: A(Y) \lto X$ be a $k$-algebra homomorphism. Let $\bar{x}_1,...,\bar{x}_n$ denote the image of $x_1,...,x_n \in k[x_1,...,x_n]$ in $A(Y)$. Consider the images $f(\bar{x}_1),...,f(\bar{x}_n)$ of the elements $\bar{x}_1,...,\bar{x}_n \in A(Y)$ under $f$ (a step which could not have been done had we not assumed $Y$ was affine). These are all regular functions $f(x_i): X \lto k$ and so for each $P \in X$ we obtain an element $(f(\bar{x}_1)(P),...,f(\bar{x}_n)(P)) \in \bb{A}^n$, we now show that this is an element of $Y$. For any $g \in I(Y)$ the polynomial $g(\bar{x}_1,...,\bar{x}_n)$ is equal to zero in $A(Y)$. Thus for any $P \in X$ we have
\[f(g(\bar{x}_1,...,\bar{x}_n))(P) = 0\]
However, $g$ is a polynomial and $f$ is a $k$-algebra homomorphism, so
\[f(g(\bar{x}_1,...,\bar{x}_n))(P) = g\big(f(\bar{x}_1)(P),...,\bar{x}_n(P)\big) = 0\]
and so $(f(\bar{x}_1)(P),...,f(\bar{x}_n)(P)) \in ZI(Y) = Y$.  In fact, this is a morphism, this follows from Lemma \ref{lem:variable_functions} below. We have described a function
\begin{equation}
    \beta: \operatorname{Hom}_{k\operatorname{Alg}}(A(Y),\call{O}(X)) \lto \operatorname{Hom}_{\operatorname{Var}}(X,Y)
\end{equation}
Finally, we show that $\alpha$ and $\beta$ are mutually inverse to each other. Let $f: X \lto Y$ be a morphism of varieties and let $P \in X$, we compute $(\beta \alpha f)(P)$, where $\pi_i: Y \lto k$ is the $i^{\text{th}}$ projection map:
\begin{align*}
    (\beta \alpha f)(P) &= \big((\hat{f} \circ \varphi)(\bar{x}_1(P)),\hdots,(\hat{f}\circ \varphi)(\bar{x}_n(P))\big)\\
    &= \big((\pi_1 \circ f)(P),\hdots,(\pi_n \circ f)(P)\big)\\
    &= f(P_1,...,P_n)\\
    &= f(P)
\end{align*}
Similarly, given $f: A(Y) \lto \call{O}_X$ and $q \in A(Y)$ we compute $(\alpha \beta f)(P)$, where $\varphi: A(Y) \lto \call{O}_Y$ is the isomorphism \eqref{prop:affine_basics}, and $h: X \lto Y$ is the function given by $x \mapsto f(\bar{x}_1)(x),...,f(\bar{x}_n)(x)$:
\begin{align*}
    (\alpha \beta f)(q) &= \alpha(h)(q)\\
    &= \hat{h}(\varphi(q))
\end{align*}
Let $P \in X$ be arbitrary, we claim $\hat{h}(\varphi(q))(P) = f(q)(p)$:
\begin{align*}
    \hat{h}(\varphi(q))(P) &= \varphi(q)(h(P))\\
    &= q(h(P))\\
    &= f(q(P))\\
    &= f(q)(P)
\end{align*}
where the penultimate equality follows from the definition of $h$, that $f$ is a $k$-algebra homomorphism, and $q$ is a polynomial.
\begin{remark}
The final calculation can be written in different notation:
\begin{align*}
    (\alpha\beta f)(q)(P) &= \alpha(f(\bar{x}_1)(\und{0.2}),...,f(\bar{x}_n)(\und{0.2}))(P)\\
    &= (f(\bar{x}_1),...,f(\bar{x}_n))^{\wedge{ }})(\varphi(q))(P)\\
    &= \varphi(q)\big(f(\bar{x}_1)(P),...,f(\bar{x}_n)(P)\big)\\
    &= q\big(f(\bar{x}_1)(P),...,f(\bar{x}_n)(P)\big)\\
    &= f(q(P))\\
    &= f(q)(P)
\end{align*}
\end{remark}
\begin{lemma}\label{lem:check_variables}
Let $X$ be a variety, and let $Y \subseteq \bb{A}^n$ be an affine variety. A map of sets $\psi: X \lto Y$ is a morphism if and only if $x_i \circ \psi$ is a regular function on $X$ for each $i$, where $x_1,...,x_n$ are the coordinate functions on $\bb{A}^n$.
\end{lemma}
\begin{proof}
First, that $\psi$ being a morphism implies $x_i \circ \psi$ is regular follows from the definition of morphism.

Next, the key observation is that $x_i \circ \psi$ being regular implies that $f \circ \psi$ is regular for any polynomial $f: Y \lto k$. We then have for any zero set $Z(f_1,...,f_n) \subseteq Y$ that
\begin{equation}
\psi^{-1}Z(f_1,...,f_n) = \psi^{-1}\big(Z(f_1)\cap...\cap Z(f_n)\big) = \psi^{-1}Z(f_1)\cap ... \cap \psi^{-1}Z(f_n)
\end{equation}
and so we can assume $n = 1$.

We notice that $\psi^{-1}Z(f) = (f\circ \psi)^{-1}\lbrace 0 \rbrace$ which is closed as $f\circ\psi$ is regular (from the key observation) and thus continuous.

Moreover, if $g: Y \lto k$ is regular, then there exists open $U \subset Y$ and polynomials $g_1,g_2$ such that $g\restriction_{U} = g_1/g_2$. Thus, for any $P \in X$:
\begin{equation}
g\restriction_{U}(\psi(P)) = g_1(\psi(P))/g_2(\psi(P))
\end{equation}
and we know $g_i \circ \psi$ is regular for $i = 1,2$, thus there are open sets $W_1,W_2 \subseteq Y$ both containing $\psi(P)$ such that $g_i\restriction_{W_i}$ is a quotient of polynomials, for $i = 1,2$. It follows that $g \circ \psi$ is regular.
\end{proof}



In summary:
\begin{proposition}
\label{prop:adjunction}
Let $X$ be a variety and $Y$ an affine variety. Then there is a natural bijection
\[\alpha: \operatorname{Hom}_{\operatorname{Var}}(X,Y) \lto \operatorname{Hom}_{k\operatorname{Alg}}(A(Y),\call{O}(X))\]
Moreover, a morphism $X \lto Y$ is \emph{dominant} (that is, it has dense image) if and only if the corresponding homomorphism $A(Y) \lto \call{O}(X)$ is injective.
\end{proposition}
\begin{proof}
We only need to prove the last claim. \textcolor{red}{How do we get injective implies dominant?}
\end{proof}
%
%
%
%
%
\subsection{Products}
The category of Varieties $\underline{\operatorname{Var}}$ admits products, and this product is \emph{not} equal to the topological product. For instance, we will see that $\bb{A}^2$ is a product of two copies of $\bb{A}^1$ but $\bb{A}^2 \not\cong \bb{A}^1 \times \bb{A}^1$ (see solution to Sol. $1.4$ \cite{hartshorne_solutions}). Arbitrary products will require some work, but the product of two affine varieties is simpler.
\begin{proposition}\label{prop:prod_affine_variety}
Let $X \subseteq \bb{A}^n$ and $Y \subseteq \bb{A}^m$ be affine varieties. Then the product $X \times Y \subseteq \bb{A}^{n+m}$ is a product in the category of varieties.
\end{proposition}
\begin{proof}
The main observation is that $A(X \times Y) \cong A(X) \otimes A(Y)$, then we use Proposition \ref{prop:adjunction}. For details see Sol $3.15$ \cite{hartshorne_solutions}.
\end{proof}
For the projective case we need to use the \emph{Segre embedding}:
\begin{lemma}
Define the following function
\begin{align}
    \Psi: \bb{P}^r \times \bb{P}^s &\lto \bb{P}^{rs + r + s}\\
    \big([P_0:\hdots:P_r],[Q_0:\hdots:Q_s]\big) &\longmapsto \big([\hdots, P_iQ_j,\hdots]\big)
\end{align}
where elements in the image appear in lexicographic order. This map is well defined, injective, the image is a projective variety, and this map is an isomorphism onto its image.
\end{lemma}
\begin{proof}
See Sol $2.14$ \cite{hartshorne_solutions}.
\end{proof}
\begin{proposition}
Let $X \subseteq \bb{P}^n$ and $Y \subseteq \bb{P}^m$ be projective varieties. Then the image of $X \times Y \subseteq \bb{P}^r \times \bb{P}^s$ under the Segre embedding is a project in the category of varieties.
\end{proposition}
\begin{proof}
This is partly worked out in Sol $3.16$ of \cite{hartshorne_solutions}. However, \textcolor{red}{there is a hole here, how do we know we have a product}?
\end{proof}




\subsection{Rational maps}
In Algebraic Topology, the notion of homeomorphism is relaxed to homotopy equivalence which leads to significant theorems (Whitehead's Theorem, etc) relating topology to algebra. Similarly, rational maps are a relaxation of morphisms of varieties and in this Section we explore how this relaxed notion interacts with algebra.
\begin{defn}
A \emph{rational map} of varieties $X \lto Y$ is an equivalence class of pairs $(U,\varphi)$ where $U \subseteq X$ is an open subset and $\varphi: U \lto Y$ a morphism, where any two such pairs are equivalent if the morphisms agree on their intersection. If the image of $\varphi$ is dense then by Fact \ref{fact:dominant_rat_map} all elements of the corresponding rational map are, and so we say the rational map is \textbf{dominant}.
\end{defn}
\begin{fact}
\label{fact:dominant_rat_map}
If one representative of a rational map is dominant, they all are.
\end{fact}
%
We need:
\begin{lemma}
\label{lem:inv_img_closure}
If $f:X \lto Y$ is continuous and $U \subseteq Y$ then $\overline{f^{-1}(U)} \subseteq f^{-1}(\overline{U})$.
\end{lemma}
\begin{proof}
Since $f$ being continuous and $f^{-1}(U) \subseteq f^{-1}(\overline{U})$.
\end{proof}
\begin{proof}[Proof of Fact \ref{fact:dominant_rat_map}]
Let $U \subseteq V \subseteq X$ both be open (and hence dense) and assume the image of $V$ is dense in $Y$. We have
\[X = \overline{U} \subseteq \overline{f^{-1}(f(U))} \subseteq f^{-1}(\overline{f(U)})\]
using Lemma \ref{lem:inv_img_closure} for the second inclusion. Thus, $\operatorname{im} f \subseteq \overline{f(U)}$, taking the closure of both sides gives $\overline{\operatorname{im} f} \subseteq \overline{f(U)}$ the result then follows from $\overline{\operatorname{im} f} \subseteq \overline{f(V)} = Y$.

Thus if $\varphi_U$ and $\varphi_V$ are rational maps which agree on $U \cap V$ with $\varphi_V$ dominant, we have that $\varphi_V\restriction_{U \cap V} = \varphi_U\restriction_{U \cap V}$ is dominant and thus $\varphi_U$ is dominant.
\end{proof}
For the next result, we need the following:
\begin{proposition}\label{prop:base}
The set of open, affine varieties forms a basis for the topology of any variety.
\end{proposition}
\begin{proof}
Let $Y$ be a variety. We need to show that for any point $P \in Y$ and open subset $U$ containing $P$, there exists an open, affine variety $Z$ such that $P \in Z \subseteq U$. We know that each variety can be covered by quasi-affine varieties and so we can assume $U = Y$, and that $Y$ is quasi-affine. 

We consider the set $Z := \overline{Y} - Y$ which is closed and does not contain $P$. Thus there exists a polynomial $f$ such that $f(P) \neq 0$ and $f \in IZ$. We let $H$ be the hypersurface $f = 0$, and the proof will be complete once we show $Y - Y \cap H$ is an open, affine variety.

Openness is clear, so it remains to show that it is affine. For this we appeal to Lemma \ref{lem:ambient_hypersurface_affine} below, which states that $\bb{A}^n - G$ is affine for any hypersurface $G$. 

Since $Y - Y \cap H$ is an open subset of irreducible $Y$, we have that $Y - Y \cap H$ is irreducible as a subspace of $Y$.  Since irreducible subspaces of irreducible subspaces remain irreducible (Lemma \ref{lem:irred_subset_irred}) the set $Y - Y \cap H$ is an irreducible subset of $\bb{A}^n - H$, for some $n$. It now remains to prove the following claim:
\begin{claim}
Let $Y$ be a variety. $Y - Y \cap H$ is a closed subset of $\bb{A}^n - H$.
\end{claim}
We show $Y - Y \cap H = \overline{Y} \cap (\bb{A}^n - H)$. First notice $\overline{Y} \cap (\bb{A}^n - H) = \overline{Y} - \overline{Y} \cap H$. We have:
\begin{align*}
    \overline{Y} - Y \subseteq H &\Longrightarrow (\overline{Y} - Y) \cap \overline{Y} \subseteq H \cap \overline{Y}\\
    &\Longrightarrow \overline{Y} - Y \subseteq H \cap \overline{Y}\\
    &\Longrightarrow \overline{Y} - \overline{Y} \cap H \subseteq \overline{Y} - (\overline{Y} - Y)\\
    &\Longrightarrow Y - \overline{Y} \cap H \subseteq Y
\end{align*}
Subtracting $Y \cap H$ from both sides and noting that $Y \cap H \subseteq \overline{Y} \cap H$ yields:
\begin{equation}
\label{eq:closed_sub}
\overline{Y} - \overline{Y} \cap H \subseteq Y - Y \cap H
\end{equation}
For the reverse direction, we have that $Y - Y \cap H \subseteq \overline{Y}$ and $\overline{Y} \cap H \subseteq \overline{Y}$ and so
\[(Y - Y \cap H) \cup (\overline{Y} \cap H) \subseteq \overline{Y}\]
subtracting $\overline{Y} \cap H$ from both sides gives:
\begin{align*}
    \big((Y - Y \cap H) \cup (\overline{Y} \cap H)\big) - (\overline{Y} \cap H) &\subseteq  \overline{Y} - (\overline{Y} \cap H)\\
    \Longrightarrow (Y - Y \cap H) - (\overline{Y} \cap H) &\subseteq \overline{Y} - (\overline{Y} \cap H)
\end{align*}
so since $(Y - Y \cap H) - (\overline{Y} \cap H) = (Y - Y \cap H) - (Y \cap \overline{Y} \cap H)$ we have
\begin{align*}
    (Y - Y \cap H) - (Y \cap \overline{Y} \cap H) &\subseteq \overline{Y}  - (\overline{Y} \cap H)\\
    \Longrightarrow (Y - Y \cap H) - (Y \cap H) &\subseteq \overline{Y}  - (\overline{Y} \cap H)\\
    \Longrightarrow Y - Y \cap H &\subseteq \overline{Y}  - \overline{Y} \cap H
\end{align*}
proving the result.
\end{proof}
\begin{lemma}
Let $H$ be a hypersurface $f(x_1,...,x_n) = 0$ and assume $H \subseteq \bb{A}^n$. Then $\bb{A}^n - H$ is affine.
\end{lemma}
\begin{proof}
We consider the hypersurface $J = Z(x_{n+1}f - 1) \subseteq \bb{A}^{n+1}$. There is a map $\varphi: J \lto \bb{A}^n - H$ given by the projection $(P_1,...,P_{n+1}) \longmapsto (P_1,...,P_n)$ (which, by the way, corresponds to the map $A \lto A_f$ where $A = k[x_1,...,x_n]$). This map is bijective with inverse given by $(P_1,...,P_n) \longmapsto (P_1,...,P_n,1/f(P_1,...,P_n))$. That these are both morphisms is easy, for the inverse map use \ref{lem:check_variables}.
\end{proof}
\begin{example}
As an easy example, $\bb{A}^1 - Z(x)$ is isomorphic to the hypersurface $x_2 = 1/x_1$ in $\bb{A}^2$.
\end{example}



In light on Proposition \ref{prop:adjunction} we ask ``what relaxations are needed to turn the adjunction \ref{prop:adjunction} into an equivalence of ƒcategories"? This is answered by the following:
\begin{thm}
\label{thm:equivalence_of_cats}
The category of varieties and dominant, rational maps is equivalent to the category of finitely generated field extensions of $k$.
\end{thm}
The proof will begin by consider two varieties $X,Y$ and a $k$-algebra homomorphism $K(Y) \lto K(X)$. Since $Y$ is covered by affine varieties (Proposition \ref{prop:base}) we take an affine variety $Y' \subseteq Y$ which since $K(Y) \cong K(Y')$ means we have a $k$-algebra homomorphism $K(Y') \lto K(X)$. We then use this to obtain an open subset $U \subseteq X$ such that we have the following commutative diagram:
\[
\begin{tikzcd}
K(Y)\arrow[d,swap,"{\cong}"]\\
K(Y')\arrow[rr] & & K(X)\\
A(Y')\arrow[u,rightarrowtail]\arrow[rr] & & \call{O}(U)\arrow[u,rightarrowtail]\\
& \wr &\text{ \eqref{prop:adjunction}}\\
U\arrow[rr]\arrow[d,rightarrowtail] & & Y'\arrow[d,rightarrowtail]\\
X\arrow[rr,dashed] & & Y
\end{tikzcd}
\]
where the dashed arrow is a rational map induced by the map $U \lto Y'$.
\begin{proof}
Let $X,Y$ be arbitrary varieties. Let \underline{DVar} denote the category of varieties and dominant ration maps, and let \underline{$k$alg} the category of $k$-algebras. We will show that there is a natural bijection
\begin{equation}
    \alpha: \text{\underline{DVar}}(X,Y) \lto \underline{k\text{alg}}(K(Y),K(X))
\end{equation}
Let $(U,\varphi)$ be a representative of a rational map $X \lto Y$, and say we have an element $[(V,\psi)] \in K(Y)$. Notice that $V$ is non-empty (by definition of elements of the field of functions) and since the image of $\varphi$ dense that $\varphi^{-1}(V)$ is non-empty. Moreover, since $\varphi$ is a morphism, the pair $(\varphi^{-1}(V),\varphi \circ \psi\restriction_{\varphi^{-1}(V)})$ yields a well defined element of $K(X)$. This map is independent of the representative $(U,\varphi)$ taken and so we have a well defined map $K(Y) \lto K(X)$.

We now define an inverse to $\alpha$. Let $\varphi: K(Y) \lto K(X)$ be an injective, there is an injective homomorphism $A(Y) \rightarrowtail K(Y)$ so each $\bar{x}_i \in A(Y)$ maps under $A(Y) \lto K(Y) \lto K(X)$ to some element $[(U_i, \xi_i)]$. So we can pick open subsets $U_1,...,U_n$ so that their intersection $U := \bigcap_{i = 1}^nU_i$ fits into the following commutative diagram:
\begin{equation}\label{eq:inverse}
\begin{tikzcd}
K(Y)\arrow[r,"{\varphi}"] & K(X)\\
A(Y)\arrow[u,rightarrowtail]\arrow[r] & \call{O}(U)\arrow[u,rightarrowtail]
\end{tikzcd}
\end{equation}
The map $\varphi$, having domain a field, is injective. Commutativity of \eqref{eq:inverse} thus implies the homomorphism on the bottom row. We thus by Proposition \ref{prop:adjunction} obtain a dominant morphism $U \lto Y$, which yields a dominant, rational map $\beta(\varphi): X \lto Y$. By construction, for any rational map $f$ we have $\beta\alpha f = f$ and for any $k$-alg homomorphism $f$ we have $\alpha \beta f = f$.

It remains to show that the functor $X \lto K(X)$ is well defined (more precisely, that $K(X)$ is a finitely generated field extension of $k$) and essential surjectivity (more precisely, that for every finitely generated field extension $K/k$ we have that $K = K(X)$ for some variety $X$).

For the first claim, simply take an affine subset $X' \subseteq X$ (which can be done by \ref{prop:base}) so that $K(X) = K(X')$ and use Proposition \ref{prop:affine_basics}.

For the second claim, let $K/k$ be a finitely generated field extension and $\alpha_1,...,\alpha_n$ be such that $k(\alpha_1,...,\alpha_n) = K$ and consider the subring $k[\alpha_1,...,\alpha_n]$ which, being a subring of a field, is an integral domain. We thus have an isomorphism
\[k[x_1,...,x_n]/I \lto k[\alpha_1,...,\alpha_n]\]
for some prime ideal $I$ and so $K = K(Z(I))$.
\end{proof}
\begin{cor}\label{cor:function_field_birational}
For two varieties $X,Y$, the following are equivalent:
\begin{enumerate}
\item\label{cor:function_field_birational_birat} $X$ is birationally equivalent to $Y$,
\item\label{cor:function_field_birational_open} there is open sets $U \subseteq X, V \subseteq Y$ such that $U$ is isomorphic to $V$ as varieties,
\item\label{cor:function_field_birational_function} $K(X) \cong K(Y)$.
\end{enumerate}
\end{cor}
\begin{proof}
\eqref{cor:function_field_birational_birat} $\Longrightarrow$ \eqref{cor:function_field_birational_open}: Let $\varphi: X \lto Y$ be a rational map with inverse $\psi: Y \lto X$. Then say $\varphi$ is represented by $(U, \varphi)$ and $\psi$ by $(V,\psi)$. We thus have that $\varphi \psi$ is represented by $(\varphi^{-1}(V), \varphi \psi)$, so since $\varphi \psi = \operatorname{id}_Y$ as rational functions, we have that $\varphi \psi\restriction_{\varphi^{-1}(U)} = \operatorname{id}_{\varphi^{-1}(U)}$. Similarly, we have $\psi\varphi\restriction_{\varphi^{-1}(V)} = \operatorname{id}_{\psi^{-1}(V)}$. We then take our choice of open sets to be $\varphi^{-1}\psi^{-1}(U)$ and $\psi^{-1}\varphi^{-1}(V)$ and we are done.

\eqref{cor:function_field_birational_open} $\Longrightarrow$ \eqref{cor:function_field_birational_function} follows from the definition of the field of functions.

\eqref{cor:function_field_birational_function} $\Longrightarrow$ \eqref{cor:function_field_birational_birat} follows from Theorem \eqref{thm:equivalence_of_cats}.
\end{proof}
\begin{proposition}\label{prop:hypersurface}
Every variety $X$ of dimension $r$ say, is birationally equivalent to a hypersurface in $\bb{A}^{r+1}$, and by taking the projective closure, to a hypersurface in $\bb{P}^{r+1}$.
\end{proposition}
\begin{proof}
The field extension $K(X)/k$ is finitely generated and, as $k$ is perfect, separable. Hence there is a transcendence basis $x_1,...,x_r \in k$ such that $K(X)/k(x_1,...,x_r)$ is a separable extension.

By the Theorem of a primitive element, there exists an element $y \in k$ such that $K(X) = k(x_1,...,x_r,y)$. Now, $y$ is algebraic and so satisfies a polynomial equation which, after clearing denominators, yields an irreducible polynomial $f$ which $y$ satisfies. The hypersurface $f = 0$ (which lies in $\bb{A}^{r+1})$, we denote by $H$, and claim is birationally equivalent to $X$. By Theorem \ref{thm:equivalence_of_cats} it suffices to show $K(H) \cong K(X)$, however this is easy by the construction of $H$, the map
\begin{equation}
k[x_1,...,x_r,y] \lto k(x_1,...,x_r,y)
\end{equation}
induces an isomorphism
\begin{equation}
\operatorname{Frac}\big(k[x_1,...,x_n]/(f)\big) \cong k(x_1,...,x_n,y)
\end{equation}
the result follows as $K(H) \cong \operatorname{Frac}\big(k[x_1,...,x_n]/(f)\big)$ and $k(x_1,...,x_n,y) = K$.
\end{proof}
%
%
%
%
%
%







\section{Singularities}
\subsection{Singular points}
Throughout, $X$ is an affine variety of dimension $r$.
\begin{defn}
\label{def:singular_point}
Given a point $P \in X$ and generators $f_1,...,f_n$ for $I(X)$. The point $P$ is \textbf{nonsingular} if the rank of the \emph{jacobian matrix} $|| (\partial f_i/\partial x_j)(P)|| = n - r$. Otherwise the point is \textbf{singular}.
\end{defn}
\begin{remark}\label{rmk:defining_functions_invariant}
This definition is independent of the choice of generators of $I(X)$ taken, we now show this. In such a situation we have
\[(f_1,...,f_n) = I(X) = (g_1,...,g_m)\]
so for each $i=1,...,n$ there exists polynomials $h_1^i,...,h_m^i$ such that
\[f_i = h_1^i g_1 + \hdots + h_m^i g_m\]
and so for each $j = 1,...,m$:
\[(\partial f_i/\partial x_j) = g_1(\partial h_1^i/\partial x_j) + h_1^i(\partial g_1/\partial x_j) + \hdots + g_m(\partial h_m^i/\partial x_j) + h_m^i(\partial g_m/\partial x_j)\]
so that if $P \in X$:
\[(\partial f_i/\partial x_j)(P) = h_1^i(P)(\partial g_1/\partial x_j)(P) + \hdots + h_m^i(P)(\partial g_m/\partial x_j)(P)\]
Since this is true for all $i,j$ we have:
\[
||(\partial f_i/\partial x_j)(P)|| =
\begin{bmatrix}
h_1^1(P) & \hdots & h_m^1(P)\\
\vdots & \ddots & \vdots\\
h_1^n(P) & \hdots & h_m^n(P)
\end{bmatrix}
||(\partial g_i/\partial x_j)(P)||
\]
so that $\operatorname{Rank}||(\partial f_i/\partial x_j)(P)|| \leq \operatorname{Rank}||(\partial g_i/\partial x_j)(P)||$. We can also describe each $g_i$ as a linear combination of $f_j$ and perform the same argument to establish the reverse inequality. Thus these values are equal and Definition \ref{def:singular_point} is independent of the choice of generators for $I(X)$.
\end{remark}
The definition of a singular point on an arbitrary variety is given by considering its local ring.
\begin{defn}
Let $(A,\frak{m})$ be a Noetherian, local ring and let $k$ denote the residue field $A/\frak{m}$. The ring $A$ is \textbf{regular} if $\operatorname{dim}A = \operatorname{dim}_k\frak{m}/\frak{m}^2$.
\end{defn}
\begin{defn}\label{def:singular}
Let $X$ be a variety and $P \in X$ a point. Then $X$ is \textbf{nonsingular} at $P$ if $\call{O}_P$ is a regular local ring. Otherwise $X$ is \textbf{singular} at this point.
\end{defn}
Before showing that Definitions \ref{def:singular_point} and \ref{def:singular} agree in the case where $X$ is affine, we make a comment that Definition \ref{def:singular} (along with Exercise $3.3b$) makes it clear that a point being singular is invariant under isomorphism. We use this to show:
\begin{lemma}
Let $X = Z(f_1,...,f_m) \subseteq \bb{P}^n$ be a projective variety of dimension $r$ and $P \in X$ a point. If $\operatorname{Rank}||(\partial f_i/\partial x_j)(P)|| = n - r$ then $X$ is singular at this point.
\end{lemma}
\begin{proof}
Write $P = [P_0:...:P_n]$ and for simplicity, assume $P \in U_0$ and consider the isomorphism $\varphi_0: U_0 \lto \bb{A}^n$. We know that
\[\varphi_0\big(Z(f_1,...,f_m)\big) = Z\big(f_1(1,x_1,...,x_n),...,f_n(1,x_1,...,x_n)\big)\]
and $\varphi_0([P_0:...:P_n]) = (P_1/P_0,...,P_n/P_0)$. It suffices to show where $i=1,...,m,j=1,...,n$ (NB: we do not consider $j = 0$) that:
\begin{equation}\label{eq:sufficient}
\operatorname{Rank}||(\partial f_i(1,x_1,...,x_n)/\partial x_j)(P_1/P_0,...,P_n/P_0)|| = n - r
\end{equation}
The significant observation is that the matrix, where $i=1,...,m,j = 0,...,n$ (NB: now we do consider $j = 0$)
\[||(\partial f_i/\partial x_j)(P_0,...,P_n)||\]
has left most column consisting of all zeros. Moreover, the number of columns is strictly greater than the rank (which by assumption is $n - r$) and so deleting this left most column does not change the rank.

The proof is then finished when it has been shown that this rank is equal to left hand side of \eqref{eq:sufficient}, which follows from independence of representative chosen to calculate the rank (a calculation similar (but easier) than that done in Remark \ref{rmk:defining_functions_invariant}).
\end{proof}

\begin{lemma}\label{lem:equivalence_singularity}
Let $Y \subseteq \bb{A}^n$ be an affine variety. Let $P \in Y$ be a point. Then $Y$ is nonsingular at $P$ if and only if the local ring $\call{O}_P$ is a regular local ring.
\end{lemma}
\begin{proof}
Let $P$ be the point $(P_1,...,P_n)$ in $\bb{A}^n$ and let $\frak{a}_P := (x - P_1,...,x - P_n)$ be the corresponding maximal ideal in $A := k[x_1,...,x_n]$. We define a linear map $\theta: A \lto k^n$ by
\begin{equation}
\theta (f) := \Big(\frac{\partial f}{\partial x_1}(P),...,\frac{\partial f}{\partial x_n}(P)\Big)
\end{equation}
Now it is clear that $\theta(x_i - P_i)$ for $i = 1,...,n$ from a basis of $k^n$, and that $\theta(\frak{a}_P^2) = 0$, so $\theta$ induces an isomorphism $\theta': \frak{a}_P/\frak{a}_P^2 \lto k^n$.

We now let $\frak{b}$ be the ideal of $Y$ in $A$, and let $f_1,...,f_t$ be a set of generators of $\frak{b}$. The proof essentially follows from the following key observation:
\begin{equation}
\operatorname{dim}\frak{a}_p/(\frak{a}_P^2 + \frak{b}) + \operatorname{dim}(\frak{a}_P^2 + \frak{b})/\frak{a}_P^2= \operatorname{dim}\frak{a}_P/\frak{a}_P^2
\end{equation}
which is true just by counting dimensions of vector spaces. We now explain these components. First, by definition of the map $\theta'$ we have that $\operatorname{rank}||(\partial f_i/\partial x_j)(P)||$ is the dimension of the subspace $(\frak{b} + \frak{a}_P^2)/\frak{a}_P^2$ of $\frak{a}_P/\frak{a}_P^2$.  Combining this with the fact that $\operatorname{dim}\frak{a}_P/\frak{a}_P^2 = \operatorname{dim}k^n$, cf. the isomorphism $\theta'$, our equation becomes
\begin{equation}\label{eq:broken_further}
\operatorname{dim}\frak{a}_P/(\frak{a}_P^2 + \frak{b}) + \operatorname{rank}||(\partial f_i/\partial x_j)(P)||   = n
\end{equation}
On the other hand,  let $\overline{\frak{a}}_P$ denote the image of $\frak{a}_P$ under the projection $k[x_1,...,x_n] \lto A(Y)$ and consider the map $\varphi: k[x_1,...,x_n] \lto A(Y)_{\overline{\frak{a}}_P}$ which is the given by projecting to the quotient followed by localising. If $\frak{c}$ denotes the maximal ideal of $A(Y)_{\overline{\frak{a}}_P}$, we have
\begin{equation}\label{eq:sum}
\varphi^{-1}(\frak{c}) = \frak{a} + \frak{b},\qquad\text{and}\qquad\varphi^{-1}(\frak{c}^2) = \frak{a}^2 + \frak{b}
\end{equation}
Now, we know that $\call{O}_P \cong A(Y)_{\overline{\frak{a}}_P}$, so if $\frak{m}_P$ denotes the maximal ideal of $\call{O}_P$, it follows from \eqref{eq:sum} that
\begin{equation}
\frak{m}_P/\frak{m}_P^2 \cong (\frak{a} + \frak{b})/(\frak{a}^2 + \frak{b}) = \frak{a}/(\frak{a}^2 + \frak{b})
\end{equation}
Thus, \eqref{eq:broken_further} becomes
\begin{equation}
\operatorname{dim}\frak{m}_P/\frak{m}_P^2 + \operatorname{rank}||(\partial f_i/\partial x_j)(P)||   = n
\end{equation}
and we are done.
\end{proof}
We now make the obvious definition:
\begin{defn}
Let $Y$ be a variety. $Y$ is \textbf{nonsingular} at a point $P \in Y$ if the local ring $\call{O}_{P,Y}$ is a regular local ring. $Y$ is \textbf{nonsingular} if it is nonsingular at every point. If $Y$ is not nonsingular, it is \textbf{singular}.
\end{defn}
For the next result, we need the following algebraic Lemma:
\begin{lemma}\label{lem:max_ideal_lemma}
Let $A$ be a Noetherian local ring with maximal ideal $\frak{m}$. Then
\begin{equation}
\operatorname{dim}_k\frak{m}/\frak{m}^2 \geq \operatorname{dim}A
\end{equation}
\end{lemma}
\begin{proof}
For a full proof, see \cite{algebra}, here we satisfy ourselves with a sketch.

Krull's Principal ideal Theorem states that in the context of the lemma, if $\frak{p}$ is prime, minimal amongst those over a collection of elements $a_1,...,a_r \in A$ of $A$ then $\operatorname{ht}.\frak{p} \leq r$. Thus, if we pick elements $a_1,...,a_r \in A$, the image of which under $A \lto A_{\frak{m}}$ form a basis for the vector space $\frak{m}/\frak{m}^2$. There exists a prime $\frak{p}$ minimal over those containing $a_1,...,a_r$, the result follows.
\end{proof}
We will make use of the following fact:
\begin{fact}\label{fact:cover_closed_with_open}
Let $X$ be any topological space, $Z \subseteq X$ a subset, and $\lbrace U_i\rbrace_{ i \in I}$ an open cover of $X$. Then $Z$ is a closed subset of $X$ if and only if $X \cap U_i$ is a closed subset of $U_i$ for all $i$.
\end{fact}
\begin{proof}
This fact is seen to be true easily when the complements are considered:
\begin{equation}
X\setminus Z = (X \setminus Z) \bigcup_{i \in I}U_i = \bigcup_{i \in I}(U_i \setminus Z)
\end{equation}
\end{proof}
\begin{thm}
Let $Y$ be a variety. Then the set $\operatorname{Sing}Y$ of singular points of $Y$ is a proper closed subset of $Y$.
\end{thm}
\begin{proof}
First we show $\operatorname{Sing}Y$ is a closed subset.  We make use of Fact \ref{fact:cover_closed_with_open} and Proposition \ref{prop:affine_basics} (that open affines form a basis for the Zariski topology) to reduce to the case where $Y$ is affine.  By \ref{lem:max_ideal_lemma} and the proof of \eqref{lem:equivalence_singularity} the set of singular points is the set of points where the rank of the Jacobian matrix is $< n - r$, where $r = \operatorname{dim}Y$. Thus, $\operatorname{Sing}Y$ is the algebraic set defined by the ideal generated by $I(Y)$ together with all determinants of $(n - r) \times (n - r)$ submatrices of the matrix $||\partial f_i/\partial x_j||$. (see \cite{linear_algebra}) Hence $\operatorname{Sing}Y$ is closed.

We now show that $\operatorname{Sing}Y$ is a proper subset. By Proposition \ref{prop:hypersurface} we have that $Y$ is birationally equivalent to a hypersurface $H$ in $\bb{A}^n$. By Corollary \ref{cor:function_field_birational}, there exists open subsets $U,V$ of $Y,H$ respectively which are isomorphic. Since $V$ is open and $\operatorname{Sing}H$ is closed, we have that $\operatorname{Sing}H$ is a proper subset of $H$ if and only if $\operatorname{Sing}H$ is a proper subset of $V$. Since $U$ and $V$ are isomorphic, we thus reduce to the case of a hypersurface in $\bb{A}^n$, assume $Y = Z(f)$ with $f$ irreducible.

Now, $\operatorname{Sing}Y$ is the set of points $P \in Y$ such that $(\partial f/\partial x_i)(P) = 0$ for $i = 1,...,n$. If $\operatorname{Sing}Y = Y$, then the functions $\partial f/\partial x_i$ are zero on $Y$, and hence $\partial f/\partial x_i \in IY$ for each $i$. But $IY$ is the principal ideal generated by $f$, and $\operatorname{deg}(\partial f/\partial x_i) < \operatorname{deg}f$, so we must have $\partial f/\partial x_i = 0$ for all $i$.

In characteristic $0$ this is already impossible as $f$ being irreducible is non-constant, so $\partial f/\partial x_i \neq 0$ for some $i$. Now say $\operatorname{char}k = p > 0$. In this case, $\partial f/\partial x_i = 0$ implies that $f$ is a polynomial in $x_i^p$. This is true of all $i$, so taking the $p^{\text{th}}$ roots of the coefficients, which is possible as $k$ is algebraically closed, we obtain a polynomial $g$ such that $f = g^p$, contradicting irreducibility of $f$.
\end{proof}

\subsection{Blowups}
Consider the set
\[X := \lbrace (x,y,z) \mid y = zx\rbrace = Z(y - zx) \subseteq\bb{A}^3\]
Let $z_0 \in k$ be arbitrary and consider the intersection of the induced plane $z = z_0$ with $X$. We have that $y = z_0x$ which is a straight line with gradient $z_0$. Thus $X$ is a long ribbon with a twist in it (this explains the classic image associated to blowups, see \cite[\S I, 4]{hartshorne}).

We claim that $X$ is birationally equivalent to $\bb{A}^2$. Indeed we have a morphism
\begin{align*}
    \alpha: X\setminus Z(x) &\lto \bb{A}^3\\
    (x,y,z) &\longmapsto (x,y)
\end{align*}
Also, there is the morphism
\begin{align*}
    \beta: \bb{A}^3\setminus Z(x) &\lto X\\
    (x,y) &\longmapsto (x,y,y/x)
\end{align*}
The morphisms $\alpha$ and $\beta$ induce rational maps which are mutual inverses to each other.

There is a natural map $\varphi: X \lto \bb{A}^2$ given by the mapping $(x,y,z) \longmapsto (x,y)$. Now we arrive at a subtlety; given a plane curve $Z(f) \subseteq \bb{A}^2$, we do not simply take the blowup to be $\varphi^{-1}(Z(f))$. This set is equal to $Z(f,y - zx)$ which contains the line $Z(x,y)$. This line is called the \emph{exceptional curve} and we wish to omit it, lest singularities not be resolved. Thus, we define:
\begin{defn}
\label{def:plane_curve_blowup}
Let $Y = Z(f)$ be a plane curve and $P \in Y$ a point. Make a change of variables ($x_i \mapsto x_i - P_i$) which translates $P$ to the origin, denote by $Y'$ the resulting curve. Denote the origin by $\call{O}$. The \textbf{blowup} of $Y$ is
\[\tilde{Y} := \overline{\varphi^{-1}(Y' \setminus \call{O})}\]
\end{defn}
More generally:
\begin{defn}
\label{def:general_blowup}
Let $Y \subseteq \bb{A}^n$ be an affine variety. Define the set
\[X := \lbrace \big((x_1,...,x_n),[P_0:...:P_{n-1}])\big) \subseteq \bb{A}^n \times \bb{P}^{n-1} \mid x_iP_{j-1} = x_jP_{i-1}\text{, where }i,j = 1,...,n\rbrace\]
and the map:
\begin{align*}
    \varphi: X &\lto Y\\
    \big((x_1,...,x_n),[P_0:...:P_{n-1}])\big) \subseteq \bb{A}^n \times \bb{P}^{n-1} &\longmapsto (x_1,...,x_n)
\end{align*}
Then the \textbf{blowup} of $Y$ at $\call{O}$
\[\tilde{Y} := \overline{\varphi^{-1}(Y\setminus \call{O})}\]
\end{defn}
A few comments are in order. First, when we write $\bb{A}^n \times \bb{P}^{n-1}$ we do \emph{not} mean the cartesian product, but instead product of these two varieties, thus we are taking an arbitrary embedding $\bb{A}^n \rightarrowtail \bb{P}^n$ followed by the Segre embedding, see Exercise $I.3.16$ and its solution \cite{hartshorne_solutions} for an explanation on the product in the category of varieties.

Also, Definitions \ref{def:plane_curve_blowup} and \ref{def:general_blowup} agree when $Y \subseteq \bb{A}^n$ is a plane curve. In this case, Definition \ref{def:general_blowup} makes use of
\[X := \lbrace \big((x_1,x_2),(P_0,P_1)\big) \mid x_1 P_1 = x_2 P_0\rbrace\]
and $\big((x_1,x_2),[P_0,P_1]\big) \in \varphi^{-1}(Y)$ is such that $x_1 \neq -, x_2 \neq 0$ and so $P_0 = 0 \Rightarrow P_1 = 0$, thus $P_0 \neq 0$ and we may use the isomorphism
\[\bb{P}^1\setminus Z(x_0) \cong \bb{A}\]
Now we reach a subtlety, indeed this shows that $\bb{P}^1 \times \bb{A}^2 \cong \bb{A} \times \bb{A}^2$ but we must read $\times$ carefully here, this is not the product of these two \emph{topological spaces} but is the product of these two \emph{affine varieties}, which by Exercise $I.3.5$ is given by $\bb{A}^3$.

\begin{example}
Define 
$$f = y^2 - x^2(x+1)$$
By observing the jacobian matrix:
\[
\begin{pmatrix}
-3x^2 -2x & 2y
\end{pmatrix}
\]
which is the zero matrix when evaluated at $0$, that is, $f$ is singular at $0$. Now consider $\varphi^{-1}(f)$ which is given by $Z(f,y - zx)$. By substituting $y = zx$ into $f$ we have:
\[z^2x^2 - x^2(x + 1) = 0\]
which factors:
\[x^2\big(z^2 - (x+1)\big) = 0\]
The algebraic set $Z(x^2\big(z^2 - (x+1)\big))$ has two irreducible components, one corresponding to $x = 0, y = 0$, and $z$ arbitrary, this is the exceptional curve. The other, $z^2 = x + 1$ and $y = zx$, this is $\tilde{Y}$. Notice that the jacobian matrix of $\tilde{Y}$:
\[
\begin{pmatrix}
-1 & 0 & 2z\\
-z & 1 & -x
\end{pmatrix}
\]
always has full rank, and so $\tilde{Y}$ is nonsingular.
\end{example}
We can talk more generally.
\begin{defn}
Let $Y = Z(f)$ be a plane curve and $P \in Y$ a point (not necessarily a singular point). By making a change of variables if necessary, assume $P = \call{O}$ (the origin). Write $f = f_0 + \hdots + f_r$ where $f_i$ has degree $i$, then the \textbf{multiplicity} of $P$ is the least value $i$ such that $f_i \neq 0$. The linear factors of $f_i$ are the \textbf{tangent directions} at $P$. A singular point of multiplicity $2$ with distinct tangent directions is a \textbf{node}.
\end{defn}
\begin{example}
The multiplicity of the singular point $\call{O}$ of $y^2 - x^2(x + 1)$ is $2$ (clearly). The tangent directions are $y + x$ and $y - x$ and so $\call{O}$ is a node.
\end{example}
Given a plane curve $Z(f)$ we have that the blowup $\tilde{Y}$ along with the exceptional curve is given by $Z(f,y - zx)$. We can come up with an explicit description of the zero set which yields only the blowup.

Write $f = f(x,y)$ and make the substitution $y = zx$ to obtain $f(x,zx)$ and now write $f(x,zx) = x^r\bar{f}(x,z)$ for some $\bar{f}$ such that $\bar{f}(0,z) \neq 0$. Then $\tilde{Y} = Z(\bar{f},y - zx)$. To see this, if $P = (P_1,P_2,P_3) \in Z(\bar{f},y - zx)$ then 
\[0= P_2^r\bar{f}(P_1,P_3) = f(P_1,P_2)\]
so $Z(\bar{f},y - zx) \subseteq \tilde{Y}$. Conversely, we clearly have $\varphi^{-1}(Y\setminus\call{O}) \subseteq Z(\bar{f},y - zx)$
 and the latter set is closed. An application of this is given in the solution to exercise $\operatorname{I}.5.6b$ in \cite{hartshorne_solutions}.























\begin{thebibliography}{9}
\bibitem{hartshorne} Robin Hartshorne, \emph{Algebraic Geometry}, Springer-Verlag New York 1977

\bibitem{hartshorne_solutions} \emph{Hartshorne Solutions}, W. Troiani.

\bibitem{algebra} \emph{Notes on Commutative Algebra}, W. Troiani

\bibitem{dimension} \emph{Dimension Theory}, W. Troiani

\bibitem{linear_algebra} \emph{Notes on linear algebra}, W.Troiani

\bibitem{grobner} Cox, Little, O-Shea, \emph{Ideal, Varieties, and Algorithms}, Springer Cham Heidelberg New York Dordrecht London 1998.

\bibitem{matsumura} Matsumura, \emph{Commutative ring theory}, Cambridge University Press, New York 1989.

\end{thebibliography}
\end{document}
