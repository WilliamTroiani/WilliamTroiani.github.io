\documentclass[12pt]{article}

\usepackage{amsthm}
\usepackage{amsmath}
\usepackage{amsfonts}
\usepackage{mathrsfs}
\usepackage{array}
\usepackage{amssymb}
\usepackage{units}
\usepackage{graphicx}
\usepackage{tikz-cd}
\usepackage{nicefrac}
\usepackage{hyperref}
\usepackage{bbm}
\usepackage{color}
\usepackage{tensor}
\usepackage{tipa}
\usepackage{bussproofs}
\usepackage{ stmaryrd }
\usepackage{ textcomp }
\usepackage{leftidx}
\usepackage{afterpage}
\usepackage{varwidth}
\usepackage{tasks}
\usepackage{ cmll }

\newcommand\blankpage{
	\null
	\thispagestyle{empty}
	\addtocounter{page}{-1}
	\newpage
}

\graphicspath{ {images/} }

\theoremstyle{plain}
\newtheorem{thm}{Theorem}[subsection] % reset theorem numbering for each chapter
\newtheorem{proposition}[thm]{Proposition}
\newtheorem{lemma}[thm]{Lemma}
\newtheorem{fact}[thm]{Fact}
\newtheorem{cor}[thm]{Corollary}

\theoremstyle{definition}
\newtheorem{defn}[thm]{Definition} % definition numbers are dependent on theorem numbers
\newtheorem{exmp}[thm]{Example} % same for example numbers
\newtheorem{notation}[thm]{Notation}
\newtheorem{remark}[thm]{Remark}
\newtheorem{condition}[thm]{Condition}
\newtheorem{question}[thm]{Question}
\newtheorem{construction}[thm]{Construction}
\newtheorem{exercise}[thm]{Exercise}
\newtheorem{example}[thm]{Example}
\newtheorem{aside}[thm]{Aside}

\def\doubleunderline#1{\underline{\underline{#1}}}
\newcommand{\bb}[1]{\mathbb{#1}}
\newcommand{\scr}[1]{\mathscr{#1}}
\newcommand{\call}[1]{\mathcal{#1}}
\newcommand{\psheaf}{\text{\underline{Set}}^{\scr{C}^{\text{op}}}}
\newcommand{\und}[1]{\underline{\hspace{#1 cm}}}
\newcommand{\adj}[1]{\text{\textopencorner}{#1}\text{\textcorner}}
\newcommand{\comment}[1]{}
\newcommand{\lto}{\longrightarrow}
\newcommand{\rone}{(\operatorname{R}\bold{1})}
\newcommand{\lone}{(\operatorname{L}\bold{1})}
\newcommand{\rimp}{(\operatorname{R} \multimap)}
\newcommand{\limp}{(\operatorname{L} \multimap)}
\newcommand{\rtensor}{(\operatorname{R}\otimes)}
\newcommand{\ltensor}{(\operatorname{L}\otimes)}
\newcommand{\rtrue}{(\operatorname{R}\top)}
\newcommand{\rwith}{(\operatorname{R}\&)}
\newcommand{\lwithleft}{(\operatorname{L}\&)_{\operatorname{left}}}
\newcommand{\lwithright}{(\operatorname{L}\&)_{\operatorname{right}}}
\newcommand{\rplusleft}{(\operatorname{R}\oplus)_{\operatorname{left}}}
\newcommand{\rplusright}{(\operatorname{R}\oplus)_{\operatorname{right}}}
\newcommand{\lplus}{(\operatorname{L}\oplus)}
\newcommand{\prom}{(\operatorname{prom})}
\newcommand{\ctr}{(\operatorname{ctr})}
\newcommand{\der}{(\operatorname{der})}
\newcommand{\weak}{(\operatorname{weak})}
\newcommand{\exi}{(\operatorname{exists})}
\newcommand{\fa}{(\operatorname{for\text{ }all})}
\newcommand{\ex}{(\operatorname{ex})}
\newcommand{\cut}{(\operatorname{cut})}
\newcommand{\ax}{(\operatorname{ax})}
\newcommand{\negation}{\sim}
\newcommand{\true}{\top}
\newcommand{\false}{\bot}
\DeclareRobustCommand{\diamondtimes}{%
	\mathbin{\text{\rotatebox[origin=c]{45}{$\boxplus$}}}%
}
\newcommand{\tagarray}{\mbox{}\refstepcounter{equation}$(\theequation)$}
\newcommand{\startproof}[1]{
	\AxiomC{#1}
	\noLine
	\UnaryInfC{$\vdots$}
}
\newenvironment{scprooftree}[1]%
{\gdef\scalefactor{#1}\begin{center}\proofSkipAmount \leavevmode}%
	{\scalebox{\scalefactor}{\DisplayProof}\proofSkipAmount \end{center} }

\usepackage[margin=1cm]{geometry}


\title{Linear Logic}
\author{Will Troiani}
\date{February 2021}

\begin{document}
	\maketitle
	\tableofcontents
	\section{Multiplicatives}
	\begin{defn}\label{def:formulas}
		There is an infinite set of \emph{unoriented atoms} $X,Y,Z,...$ and an \emph{oriented atom} (or \emph{atomic proposition}) is a pair $(X,+)$ or $(X,-)$ where $X$ is an unoriented atom. The set of \emph{pre-formulas} is defined as follows.
		\begin{itemize}
			\item Any atomic proposition is a preformula.
			\item If $A,B$ are pre-formulas then so are $A \otimes B$, $A \parr B$.
			\item If $A$ is a pre-formula then so is $\neg A$.
		\end{itemize}
		The set of \emph{formulas} is the quotient of the set of pre-formulas by the equivalence relation $\sim$ generated by, for arbitrary formulas $A,B$ and unoriented atom $X$, the following.
		\begin{equation}\label{eq:negation}
			\neg (A \otimes B) \sim \neg B \parr \neg A,\qquad \neg (A \parr B) \sim \neg B \otimes \neg A,\qquad \neg (X, +) \sim (X, -),\qquad \neg (X,-) \sim (X,+)
		\end{equation}
	\end{defn}
	\begin{lemma}
		For all formulas $A$ we have $\neg \neg A = A$.
	\end{lemma}
	\begin{proof}
		We proceed by induction on the number $n$ given by the sum of the occurrences of $\otimes$ and $\parr$ in $A$.
		
		If $n = 0$ then $A = (X,+)$ or $A = (X,-)$. In either situation the fact that $\neg \neg A = A$ follows from \eqref{eq:negation}.
		
		Now say $n > 0$ and the result holds for all $k < n$. We either have $A = A_1 \otimes A_2$ or $A = A_1 \parr A_2$, each case is proved in a similar way, we show the deatils for when $A = A_1 \otimes A_2$.
		\begin{equation}
			\neg \neg A = \neg \neg (A_1 \otimes A_2) = \neg (\neg A_2 \parr \neg A_1) = \neg \neg A_1 \otimes \neg \neg A_2
		\end{equation}
		By the inductive hypothesis we ahve that $\neg \neg A_1 = A_1$ and $\neg \neg A_2 = A_2$. It thus follows that $\neg \neg A = A$.
	\end{proof}
	\begin{defn}\label{def:negation_normal_map}
		We define the following map $\gamma: \operatorname{Pre}\Psi \lto \operatorname{Pre}\Psi$ where $p$ is atomic and $A,B$ arbitrary preformulas:
		\begin{align*}
			\negation (p) &\longmapsto \negation p & \negation p &\longmapsto \negation p\\
			\negation (\negation A) &\longmapsto A \\
			\negation (A \otimes B) &\longmapsto \negation A \parr \negation B & \negation (A \parr B) &\longmapsto \negation A \otimes \negation B\\
			A \otimes B &\longmapsto \gamma(A) \otimes \gamma(B) & A \parr B &\longmapsto \gamma(A) \parr \gamma(B)
		\end{align*}
	\end{defn}
	\begin{lemma}\label{lem:negation_normal_form}
		For any formula $A$, there exists $n > 0$ and a formula $B$ such that for all $m \ge n$ we have $\gamma^m(A) = B$.
	\end{lemma}
	\begin{proof}
		Follows from induction on the complexity $c(A)$ of $A$.
	\end{proof}
	\begin{defn}
		In the notation of Lemma \ref{lem:negation_normal_form}, $B$ is the \textbf{negation normal form} corresponding to $A$, we write $\operatorname{NF}(A) = B$.
		
		There is a map \reflectbox{$\gamma$}$: \operatorname{Pre}\Psi \lto \operatorname{Pre}\Psi$ similar to $\gamma$:
		\begin{align*}
			\negation p &\longmapsto \negation (p) & \negation A &\longmapsto \negation A\\
			p &\longmapsto \negation (\negation p) \\
			\negation A \parr \negation B&\longmapsto  \negation (A \otimes B) & \negation A \otimes \negation B &\longmapsto \negation (A \parr B) \\
			A \otimes B &\longmapsto \gamma(A) \otimes \gamma(B) & A \parr B &\longmapsto \gamma(A) \parr \gamma(B)
		\end{align*}
		This leads to the \textbf{negation abnormal form} of a formula $A$ which we denote by $\operatorname{NAB}(A)$.
	\end{defn}
	\begin{defn}
		We let $\cong$ be the smallest equivalence relation on the set of preformuals $\operatorname{Pre}\Psi$ such that $A \cong \gamma(A)$. The set $\Psi$ of \textbf{formulas} is the set of equivalence classes of preformulas under $\cong$.
	\end{defn}
	\begin{defn}
		A finite sequence of formulas is a \textbf{sequent} and we write $\vdash  A_1,..., A_n$ for the sequent $(A_1,...,A_n)$.
	\end{defn}
	\begin{defn}\label{def:mult_lin_log_ded_rule}
		A \textbf{multiplicative, linear logic deduction rule} (or simply \textbf{deduction rule}) results from one of the schemata below by a substitution of the following kind: replace $A,B$ by arbitrary formulas, and $\Gamma, \Gamma', \Delta, \Delta'$ by arbitrary (possibly empty) sequences of formulas separated by commas:
		\begin{itemize}
			\item the \textbf{identity group}:
			\begin{itemize}
				\item \textbf{Axiom}
				\begin{prooftree}
					\AxiomC{}
					\RightLabel{$\ax$}
					\UnaryInfC{$\vdash A, \sim A$}
				\end{prooftree}
				\item \textbf{Cut}:
				\begin{prooftree}
					\AxiomC{$\vdash \Gamma, A, \Gamma'$}
					\AxiomC{$\vdash \Delta, \negation A, \Delta'$}
					\RightLabel{$\cut$}
					\BinaryInfC{$\vdash \Gamma, \Gamma', \Delta, \Delta'$}
				\end{prooftree}
			\end{itemize}
			\item the \textbf{multiplicative rules}
			\begin{itemize}
				\item \textbf{Times}:
				\begin{prooftree}
					\AxiomC{$\vdash \Gamma, A, \Gamma'$}
					\AxiomC{$\vdash \Delta, B, \Delta'$}
					\RightLabel{$\otimes$}
					\BinaryInfC{$\vdash \Gamma, \Gamma', A \otimes B, \Delta, \Delta'$}
				\end{prooftree}
				\item \textbf{Par}
				\begin{prooftree}
					\AxiomC{$\vdash \Gamma, A, B, \Gamma'$}
					\RightLabel{$\parr$}
					\UnaryInfC{$\vdash \Gamma, A \parr B, \Gamma'$}
				\end{prooftree}
			\end{itemize}
			\item the \textbf{structural rule}:
			\begin{itemize}
				\item \textbf{Exchange}
				\begin{prooftree}
					\AxiomC{$\vdash \Gamma, A, B, \Gamma'$}
					\RightLabel{$\ex$}
					\UnaryInfC{$\vdash \Gamma, B, A, \Gamma'$}
				\end{prooftree}
			\end{itemize}
		\end{itemize}
	\end{defn}
	\begin{defn}\label{def:proof}
		A \textbf{proof in MLL} is a finite, rooted, planar, tree where each edge is labelled by a sequent and each node except for the root is labelled by a valid deduction rule. If the edge connected to the root is labelled by the sequent $\vdash \Gamma$ then we call the proof a \textbf{proof of $\vdash \Gamma$}.
	\end{defn}
	There is a lot of redundancy in Definition \ref{def:proof} (for instance, $\vdash$) so we introduce another way of writing proofs:
	\begin{defn}\label{def:proof_structures}
		A \textbf{multiplicative proof structure} (or simply \textbf{proof structure}) is a finite, directed, mixed, multigraph $\pi$ with vertices labelled by formulas (Definition \ref{def:formulas}). Two vertices $(v_1,v_2)$ are \textbf{premises} of a vertex $v$ if there exists directed edges $(e_1,e_2)$ of the form $e_1: v_1 \lto v, e_2: v_2 \lto v$ and are \textbf{conclusions} if they are of the form $e_1: v \lto v_1, e_2: v \lto v_2$. A choice of ordering is made on each pair of premises and each pair of conclusions. The smaller of the premises is the \textbf{left premise} and the larger is the \textbf{right premise}. Similarly for \textbf{left conclusions} and \textbf{right conclusions}.
		
		A proof structure also consists of a label of exactly one element from the set $\lbrace \ax, \cut\rbrace$ for each undirected edge $e = (v_1,v_2)$, this choice is the \textbf{label of $e$}. The vertices adjacent to an edge labelled $\ax$ are \textbf{premises} of $e$. Two vertices adjacent to an edge labelled $\cut$ are \textbf{conclusions} of $e$. An undirected edge $e$ labelled $\ax$ along with its conclusions form an \textbf{axiom link}, and an undirected edge $e$ labelled $\cut$ along with its premises form a \textbf{cut link}. The vertices of a proof structure are required to satisfy the following conditions:
		\begin{itemize}
			\item Every directed edge is a premise of some vertex.
			
			\item If $v_1,v_2$ are premises of a vertex $v$ and $v_i$ is labelled $A_i$ for $i = 1,2$, then $v$ is labelled either $A \otimes B$ or $A \parr B$. Such a triple $(v,v_1,v_2)$ is a \textbf{tensor link} if $v$ is labelled $A \otimes B$ and a \textbf{par link} otherwise.
			
			\item Every vertex is the conclusions of one and only one link.
			
			\item Every vertex is premise to at most one link.
		\end{itemize}
		The collection of axiom, cut, tensor, and par links form the \textbf{links} of the proof structure. The set of links of a proof structure $\pi$ is denoted $\call{L}_\pi$.
	\end{defn}
	\begin{defn}\label{def:occurrences_labels}
		An \textbf{occurrence of a formula} in a proof structure $\pi$ is the labelled formula corresponding to some vertex in $\pi$.
	\end{defn}
	Loosely speaking, logic is about determining correct arguemtns. That is, from the space of arguments (either correct or incorrect), logic determines techniques to extract the elements from the subspace of correct arguments. In the current context, we take the set of proof structures to be the space of proofs, both correct and incorrect, and we take the subset of so called \emph{proof nets} to be those with meaningful logical content. Proof nets are the proof structures which lie in the image of a translation map (Definition \ref{def:translation_map} below) between sequent style proofs and proof structures, we now explain this sequent style logical system.
	
	\begin{defn}\label{def:int_mult_ded_rule} 
		The set of \textbf{intuitionistic formulas} $I\Psi$ is defined in the same way as $\Psi$ in Definition \ref{def:formulas} but we omit \ref{def:formulas_negation}.
		
		Let $\call{P}^n$ be the set of all length $n$ sequences of labelled intuitionistic variables with $\call{P}^0 := \lbrace \varnothing \rbrace$, and $\call{P} := \cup_{n = 0}^\infty \call{P}^n$. A \textbf{sequent} is a pair $(\Gamma,A)$ where $\Gamma \in \call{P}$ and $A \in I\Psi$, written $\Gamma \vdash A$. We call $\Gamma$ the \textbf{antecedent} and $A$ the \textbf{succedent} of the sequent. Given $\Gamma$ and a labelled intuitionistic formula $A$ we write $\Gamma, A$ for the element of $\call{P}$ given by appending $A$ to the end of $\Gamma$. We write $\vdash A$ for $\varnothing \vdash A$.
		
		An \textbf{intuitionistic, multiplicative deduction rule} (or simply \textbf{deduction rule}) results from one of the schemata below by a substitution of the following kind: replace $A,B,C$ by arbitrary labelled intuitionistic formulas, and $\Gamma,\Delta, \Theta$ by arbitrary (possibly empty) sequences of labelled intuitionistic formulas separated by commas:
		\begin{itemize}
			\item The \textbf{identity group}:
			\begin{center}
				\begin{tabular}{ >{\centering}m{2cm} >{\centering}m{7cm} >{\centering}m{0.5cm} }
					\textbf{Axiom}
					&
					\begin{prooftree}
						\AxiomC{}
						\RightLabel{$({\operatorname{ax}})$}
						\UnaryInfC{$A \vdash A$}
					\end{prooftree}
					&
					\tagarray{\label{LL:ax}}
				\end{tabular}
			\end{center}
			
			\begin{center}
				\begin{tabular}{ >{\centering}m{2cm} >{\centering}m{7cm} >{\centering}m{0.5cm} }
					\textbf{Cut}
					&
					\begin{prooftree}
						\AxiomC{$\Gamma \vdash A$}
						\AxiomC{$\Delta, A,\Theta \vdash B$}
						\RightLabel{$({\operatorname{cut}})$}
						\BinaryInfC{$\Gamma, \Delta, \Theta \vdash B$}
					\end{prooftree}
					&
					\tagarray{\label{LL:cut}}
				\end{tabular}
			\end{center}
			\item The \textbf{logical rules}:
			\begin{center}
				\begin{tabular}{ >{\centering}m{3cm} >{\centering}m{5cm} >{\centering}m{5cm} >{\centering}m{0.5cm} }
					\textbf{Left/right times}
					&
					\AxiomC{$\Gamma, A, B, \Gamma' \vdash C$}
					\RightLabel{$\ltensor$}
					\UnaryInfC{$\Gamma, A \otimes B, \Gamma' \vdash C$}
					\DisplayProof
					&
					\AxiomC{$\Gamma \vdash A$}
					\AxiomC{$\Delta\vdash B$}
					\RightLabel{$\rtensor$}
					\BinaryInfC{$\Gamma, \Delta \vdash A \otimes B$}
					\DisplayProof
					&
					\tagarray{\label{LL:times}}
				\end{tabular}
			\end{center}
			
			\begin{center}
				\begin{tabular}{ >{\centering}m{3cm} >{\centering}m{5cm} >{\centering}m{7cm} >{\centering}m{0.5cm} }
					\textbf{Right/left implication}
					&
					\AxiomC{$\Gamma, A, \Gamma' \vdash B$}
					\RightLabel{$\rimp$}
					\UnaryInfC{$\Gamma, \Gamma' \vdash A \multimap B$}
					\DisplayProof
					&
					\AxiomC{$\Gamma \vdash A$}
					\AxiomC{$\Delta, B, \Delta' \vdash C$}
					\RightLabel{$\limp$}
					\BinaryInfC{$A \multimap B, \Gamma, \Delta \vdash C$}
					\DisplayProof
					&
					\tagarray{\label{LL:times}}
				\end{tabular}
			\end{center}
			
			\item The \textbf{structural rule}
			
			\begin{center}
				\begin{tabular}{ >{\centering}m{2cm} >{\centering}m{7cm} >{\centering}m{0.5cm} }
					\textbf{Exchange}
					&
					\begin{prooftree}
						\AxiomC{$\Gamma, A, B, \Gamma' \vdash C$}
						\RightLabel{$({\operatorname{ex}})$}
						\UnaryInfC{$\Gamma, B, A, \Gamma' \vdash C$}
					\end{prooftree}
					&
					\tagarray{\label{LL:exchange}}
				\end{tabular}
			\end{center}
		\end{itemize}
		A \textbf{proof in IMLL} is defined the same way as in Definition \ref{def:proof}.
	\end{defn}
	\begin{remark}
		We often refer to labelled intuitionistic formulas $x:A$ just by \emph{formulas}, and write $A$.
	\end{remark}
	\begin{defn}\label{def:translation_map}
		Let $\Sigma$ denote the set of multiplicative, linear logic proofs and $\operatorname{MPS}$ the set of multiplicative proof structures. We let
		\begin{equation}
			T: \Sigma \lto \operatorname{MPS}
		\end{equation}
		denote the function defined inductively by associating to each deduction rule of Definition \ref{def:mult_lin_log_ded_rule} a multiplicative proof structure:
		\begin{center}
			\begin{tabular}{ >{\centering}m{2cm} >{\centering}m{5cm} >{\centering}m{0.5cm} >{\centering}m{5cm}}
				\text{Axiom} & \begin{prooftree}
					\AxiomC{}
					\RightLabel{$\ax$}
					\UnaryInfC{$\vdash A, \sim A$}
				\end{prooftree} & $\stackrel{T}{\lto}$ &
				\begin{tikzcd}[column sep = small]
					A\arrow[r,dash,bend left] & \negation A
				\end{tikzcd}
			\end{tabular}
		\end{center}
		
		\begin{center}
			\begin{tabular}{ >{\centering}m{2cm} >{\centering}m{7cm} >{\centering}m{0.5cm} >{\centering}m{7cm}}
				\text{Cut} &
				\begin{prooftree}
					\startproof{$\pi_1$}
					\UnaryInfC{$\vdash \Gamma, A, \Gamma'$}
					\startproof{$\pi_2$}
					\UnaryInfC{$\vdash \Delta, \negation A, \Delta'$}
					\RightLabel{$\cut$}
					\BinaryInfC{$\vdash \Gamma, \Gamma', \Delta, \Delta'$}
				\end{prooftree} & $\stackrel{T}{\lto}$ &
				\begin{tikzcd}[column sep = small]
					T(\pi_1)\arrow[d,dash] & T(\pi_2)\arrow[d,dash]\\
					A\arrow[r,dash, bend right] & \negation A
				\end{tikzcd}
			\end{tabular}
		\end{center}
		
		\begin{center}
			\begin{tabular}{ >{\centering}m{2cm} >{\centering}m{7cm} >{\centering}m{0.5cm} >{\centering}m{7cm}}
				\text{Times} &
				\begin{prooftree}
					\startproof{$\pi_1$}
					\UnaryInfC{$\vdash \Gamma, A, \Gamma'$}
					\startproof{$\pi_2$}
					\UnaryInfC{$\vdash \Delta, B, \Delta'$}
					\RightLabel{$\otimes$}
					\BinaryInfC{$\vdash \Gamma, \Gamma', A \otimes B, \Delta, \Delta'$}
				\end{prooftree} & $\stackrel{T}{\lto}$ &
				\begin{tikzcd}[column sep = small]
					T(\pi_1)\arrow[d,dash] && T(\pi_2)\arrow[d,dash]\\
					A\arrow[dr,dash] &&  B\arrow[dl,dash]\\
					& A \otimes B
				\end{tikzcd}
			\end{tabular}
		\end{center}
		
		\begin{center}
			\begin{tabular}{ >{\centering}m{2cm} >{\centering}m{7cm} >{\centering}m{0.5cm} >{\centering}m{7cm}}
				\text{Par} &
				\begin{prooftree}
					\startproof{$\pi$}
					\UnaryInfC{$\vdash \Gamma, A, B, \Gamma'$}
					\RightLabel{$\parr$}
					\UnaryInfC{$\vdash \Gamma, A \parr B, \Gamma'$}
				\end{prooftree} & $\stackrel{T}{\lto}$ &
				\begin{tikzcd}[column sep = small]
					& T(\pi)\arrow[dl,dash]\arrow[dr,dash]\\
					A\arrow[dr,dash] & &  B\arrow[dl,dash]\\
					& A \parr B
				\end{tikzcd}
			\end{tabular}
		\end{center}
		
		\begin{center}
			\begin{tabular}{ >{\centering}m{2cm} >{\centering}m{7cm} >{\centering}m{0.5cm} >{\centering}m{7cm}}
				\text{Exchange} &
				\begin{prooftree}
					\startproof{$\pi$}
					\UnaryInfC{$\vdash \Gamma, A, B, \Gamma'$}
					\RightLabel{$\ex$}
					\UnaryInfC{$\vdash \Gamma, B, A, \Gamma'$}
				\end{prooftree} & $\stackrel{T}{\lto}$ &
				\begin{tikzcd}[column sep = small]
					T(\pi)
				\end{tikzcd}
			\end{tabular}
		\end{center}
		A \textbf{multiplicative proof net} (or simply \textbf{proof net}) is a multiplicative proof structure which lies in the image of $T$.
	\end{defn}
	\begin{defn}
		Let $\Pi$ denote the set of intuitionistic, multiplicative, linear logic proofs. Then again, there is a translation
		\begin{equation}
			S: \Pi \lto \operatorname{MPS}
		\end{equation}
		defined inductively:
		\begin{center}
			\begin{tabular}{ >{\centering}m{2cm} >{\centering}m{5cm} >{\centering}m{0.5cm} >{\centering}m{5cm}}
				\text{Axiom} & \begin{prooftree}
					\AxiomC{}
					\RightLabel{$({\operatorname{ax}})$}
					\UnaryInfC{$A \vdash A$}
				\end{prooftree} & $\stackrel{T}{\lto}$ &
				\begin{tikzcd}[column sep = small]
					A\arrow[r,dash,bend left] & \negation A
				\end{tikzcd}
			\end{tabular}
		\end{center}
		
		\begin{center}
			\begin{tabular}{ >{\centering}m{2cm} >{\centering}m{7cm} >{\centering}m{0.5cm} >{\centering}m{7cm}}
				\text{Cut} &
				\begin{prooftree}
					\startproof{$\pi_1$}
					\UnaryInfC{$\Gamma \vdash A$}
					\startproof{$\pi_2$}
					\UnaryInfC{$\Delta, A,\Theta \vdash B$}
					\RightLabel{$({\operatorname{cut}})$}
					\BinaryInfC{$\Gamma, \Delta, \Theta \vdash B$}
				\end{prooftree} & $\stackrel{T}{\lto}$ &
				\begin{tikzcd}[column sep = small]
					T(\pi_1)\arrow[d,dash] & T(\pi_2)\arrow[d,dash]\\
					A\arrow[r,dash, bend right] & \negation A
				\end{tikzcd}
			\end{tabular}
		\end{center}
		
		\begin{center}
			\begin{tabular}{ >{\centering}m{2cm} >{\centering}m{7cm} >{\centering}m{0.5cm} >{\centering}m{7cm}}
				\text{Left times} &
				\startproof{$\pi$}
				\UnaryInfC{$\Gamma, A, B, \Gamma' \vdash C$}
				\RightLabel{$\ltensor$}
				\UnaryInfC{$\Gamma, A \otimes B, \Gamma' \vdash C$}
				\DisplayProof & $\stackrel{T}{\lto}$ &
				\begin{tikzcd}[column sep = small]
					& T(\pi)\arrow[dl,dash]\arrow[dr,dash]\\
					\negation A\arrow[dr,dash] & & \negation B\arrow[dl,dash]\\
					& \negation A \parr \negation B
				\end{tikzcd}
			\end{tabular}
		\end{center}
		
		\begin{center}
			\begin{tabular}{ >{\centering}m{2cm} >{\centering}m{7cm} >{\centering}m{0.5cm} >{\centering}m{7cm}}
				\text{Right times} &
				\startproof{$\pi_1$}
				\UnaryInfC{$\Gamma \vdash A$}
				\startproof{$\pi_2$}
				\UnaryInfC{$\Delta\vdash B$}
				\RightLabel{$\rtensor$}
				\BinaryInfC{$\Gamma, \Delta \vdash A \otimes B$}
				\DisplayProof & $\stackrel{T}{\lto}$ &
				\begin{tikzcd}[column sep = small]
					T(\pi_1)\arrow[d,dash] && T(\pi_2)\arrow[d,dash]\\
					A\arrow[dr,dash] &&  B\arrow[dl,dash]\\
					& A \otimes B
				\end{tikzcd}
			\end{tabular}
		\end{center}
		
		\begin{center}
			\begin{tabular}{ >{\centering}m{2cm} >{\centering}m{7cm} >{\centering}m{0.5cm} >{\centering}m{7cm}}
				\text{Right implication} &
				\startproof{$\pi$}
				\UnaryInfC{$\Gamma, A, \Gamma' \vdash B$}
				\RightLabel{$\rimp$}
				\UnaryInfC{$\Gamma, \Gamma' \vdash A \multimap B$}
				\DisplayProof & $\stackrel{T}{\lto}$ &
				\begin{tikzcd}[column sep = small]
					& T(\pi)\arrow[dr,dash]\arrow[dl,dash]\\
					\negation A\arrow[dr,dash] & & B\arrow[dl,dash]\\
					& \negation A \parr B
				\end{tikzcd}
			\end{tabular}
		\end{center}
		
		\begin{center}
			\begin{tabular}{ >{\centering}m{2cm} >{\centering}m{7cm} >{\centering}m{0.5cm} >{\centering}m{7cm}}
				\text{Left implication} &
				\startproof{$\pi_1$}
				\UnaryInfC{$\Gamma \vdash A$}
				\startproof{$\pi_2$}
				\UnaryInfC{$\Delta, B, \Delta' \vdash C$}
				\RightLabel{$\limp$}
				\BinaryInfC{$A \multimap B, \Gamma, \Delta \vdash C$}
				\DisplayProof & $\stackrel{T}{\lto}$ &
				\begin{tikzcd}[column sep = small]
					T(\pi_1)\arrow[d,dash] && T(\pi_2)\arrow[d,dash]\\
					A\arrow[dr,dash] && \negation B\arrow[dl,dash]\\
					& A \otimes \negation B
				\end{tikzcd}
			\end{tabular}
		\end{center}
		
		\begin{center}
			\begin{tabular}{ >{\centering}m{2cm} >{\centering}m{7cm} >{\centering}m{0.5cm} >{\centering}m{7cm}}
				\text{Exchange} &
				\begin{prooftree}
					\startproof{$\pi$}
					\UnaryInfC{$\vdash \Gamma, A, B, \Gamma'$}
					\RightLabel{$\ex$}
					\UnaryInfC{$\vdash \Gamma, B, A, \Gamma'$}
				\end{prooftree} & $\stackrel{T}{\lto}$ &
				\begin{tikzcd}[column sep = small]
					T(\pi)
				\end{tikzcd}
			\end{tabular}
		\end{center}
		A \textbf{intuitionistic, multiplicative proof net} (or simply \textbf{intuitionistic proof net}) is an proof structure which lies in the image of $S$.
	\end{defn}
	\begin{defn}
		There is also a map $R: \Pi \lto \Sigma$ which simply moves formulas to the right of the turnstile.
	\end{defn}
	It is easy to see that the following diagram:
	\begin{equation}
		\begin{tikzcd}
			\Pi\arrow[r,"R"]\arrow[dr,swap,"S"] & \Sigma\arrow[d,"T"]\\
			& \operatorname{MPN}
		\end{tikzcd}
	\end{equation}
	commutes.
	\begin{lemma}
		The map $R$ is injective.
	\end{lemma}
	\begin{proof}
		There is a map $\operatorname{im}R \lto \Pi$ which puts all formulas of a proof $\pi \in \operatorname{im}R$ into negation abnormal form which will leave every formula $A$ of every sequent in the form $\negation B$ for some $B$ except for one (as $\pi \in \operatorname{im}R$). We move all formulas except this special one per sequent to the left of the turnstile.
	\end{proof}
	\begin{lemma}\label{lem:negation_surj}
		The map $R$ is not surjective.
	\end{lemma}
	\begin{proof}
		Define the following function
		\begin{equation}
			f: \Pi \lto \bb{Z}_2 \times \bb{Z}_2
		\end{equation}
		which, for a proof $\pi \in \Pi$, computes the following element of $\bb{Z}_2 \times \bb{Z}_2$: beginning at $(0,0)$, add $(0,1)$ for every occurrence of $\rimp$ and add $(0,0)$ for every instance of $\ltensor$. Define also the function
		\begin{equation}
			g: \Sigma \lto \bb{Z}_2 \times \bb{Z}_2
		\end{equation}
		which, for a proof $\pi \in \Sigma$, computes the following element of $\bb{Z}_2 \times \bb{Z}_2$: beginning at $(0,0)$, add $(0,1)$ for every $\parr$ rule in $\pi$ involving a formula $A$ such that $\operatorname{NF}(A) = \negation A'$ for some $A'$ on the left and a formula $B$ on the right such that $\operatorname{NF}(B) \neq \negation B'$ for any $B'$. Add $(1,0)$ for every $\parr$ rule involving a formula $A$ on the left satisfying $\operatorname{NF}(A) \neq \negation A'$ for any $A'$ and a formula $B$ on the right satisfying $\operatorname{NF}(B) = \negation B'$. Similarly for $(0,0)$ and $(1,1)$. Then the following diagram commutes:
		\begin{equation}
			\begin{tikzcd}
				\Pi\arrow[r,"R"]\arrow[dr,swap,"f"] & \Sigma\arrow[d,"g"]\\
				& \bb{Z}_2 \times \bb{Z}_2
			\end{tikzcd}
		\end{equation}
		Moreover, $g$ is surjective, the four elements $(0,0),(1,0),(0,1),(1,1)$ are respectively mapped to by
		\begin{center}
			\begin{tabular}{ >{\centering}m{6cm} >{\centering}m{7cm} >{\centering}m{0.5cm} >{\centering}m{7cm}}
				\begin{prooftree}
					\AxiomC{}
					\RightLabel{$\ax$}
					\UnaryInfC{$\vdash A, \negation A$}
					\AxiomC{}
					\RightLabel{$\ax$}
					\UnaryInfC{$\vdash A, \negation A$}
					\RightLabel{$\otimes$}
					\BinaryInfC{$\vdash A \otimes  A, \negation A, \negation A$}
					\RightLabel{$\parr$}
					\UnaryInfC{$\vdash A \otimes A, \negation A \parr \negation A$}
				\end{prooftree}
				&
				\begin{prooftree}
					\AxiomC{}
					\RightLabel{$\ax$}
					\UnaryInfC{$\vdash A, \negation A$}
					\RightLabel{$\parr$}
					\UnaryInfC{$\vdash A \parr \negation A$}
				\end{prooftree}
			\end{tabular}
		\end{center}
		\begin{center}
			\begin{tabular}{ >{\centering}m{6cm} >{\centering}m{7cm} >{\centering}m{0.5cm} >{\centering}m{7cm}}
				\begin{prooftree}
					\AxiomC{}
					\RightLabel{$\ax$}
					\UnaryInfC{$\vdash A, \negation A$}
					\AxiomC{}
					\RightLabel{$\ax$}
					\UnaryInfC{$\vdash A \negation A$}
					\RightLabel{$\otimes$}
					\BinaryInfC{$\vdash A, A, \negation A \otimes \negation A$}
					\RightLabel{$\parr$}
					\UnaryInfC{$\vdash A \parr A, \negation A \otimes \negation A$}
				\end{prooftree}
				&
				\begin{prooftree}
					\AxiomC{}
					\RightLabel{$\ax$}
					\UnaryInfC{$\vdash A, \negation A$}
					\RightLabel{$\ex$}
					\UnaryInfC{$\vdash \negation A, A$}
					\RightLabel{$\parr$}
					\UnaryInfC{$\vdash (\negation A) \parr A$}
				\end{prooftree}
			\end{tabular}
		\end{center}
		and $f$ is clearly not surjective as there are no proofs which map to $(1,0)$ nor $(1,1)$. Thus $R$ is not surjective.
	\end{proof}
	\begin{remark}
		As proof structures suppress exchanges, neither of the maps $T,S$ are injective. By definition of proof nets, the map $T$ is surjective. The image of $S$ is the set of \textbf{multiplicative, intuitionistic proof nets}. These will not be considered again in these notes, but it would be interesting to find a correctness criterion for intuitionistic proof nets similar to the \emph{long trip condition} (Section \ref{sec:sequentialisation}).
	\end{remark}
	\begin{remark}
		The proof of Lemma \ref{lem:negation_surj} would be more interesting if $\Sigma/g$ was in bijection with $\Pi$.
	\end{remark}
	In Section \ref{sec:sequentialisation} we will see what the image of the map $T$ is.
	
	\section{The Sequentialisation Theorem}\label{sec:sequentialisation}
	\begin{defn}
		Let $\pi$ be a proof structure and denote the set of tensor and par links of $\pi$ by $\operatorname{Link}_{\otimes,\parr}\pi$ (or simply $\operatorname{Link}\pi$). A \textbf{switching} of $\pi$ is a function
		\begin{equation}
			S: \operatorname{Link}\pi \lto \lbrace L,R\rbrace
		\end{equation}
		A \textbf{switching} of a particular link $\tau$ is a choice of $L,R$ associated to $\tau$.
	\end{defn}
	\begin{defn}\label{def:trip}
		Let $\pi$ be a proof structure. Let $\call{O}(\pi)$ denote the set of occurrences of formulas in $\pi$ (Definition \ref{def:occurrences_labels}).   We consider two disjoint copies of this set 
		\begin{equation}
			\call{U}(\pi) := \call{O}(\pi) \coprod \call{O}(\pi)
		\end{equation}
		where elements from the first copy are the \textbf{up elements}, and elements from the second copy are the \textbf{down elements}. We write $\uparrow A$ for the up element corresponding to an occurrence of a formula $A$ in $\pi$, and similarly for $A\downarrow$. Given a switching $S$ of $\pi$, a \textbf{pretrip of $\pi$ with respect to $S$} is a finite sequence $(x_1,...,x_n)$ of elements of $\call{U}(\pi)$ such that:
		\begin{enumerate}
			\item the sequence is a loop, that is, $x_1 = x_n$, and all elements (except the first and the last) are distinct,
			\item\label{def:trip_axiom} if $x_j=\uparrow A$ and $A$ is part of an axiom link then $x_{j+1} = \negation A\downarrow$,
			\item if $x_j = A\downarrow$ and $A$ is part of a cut link then $x_{j+1} = \uparrow \negation A$,
			\item for any tensor link $\tau$ with premises $A,B$ such that $\tau$ has switching $L$, we have:
			\begin{itemize}
				\item if $x_j = A \downarrow$ then $x_{j+1} = (A \otimes B)\downarrow$,
				\item if $x_j = \uparrow (A \otimes B)$ then $x_{j+1} = \uparrow B_j$,
				\item if $x_j = B \downarrow$ then $x_{j+1} = \uparrow A$.
			\end{itemize}
			and if $\tau$ has switching $R$, we have:
			\begin{itemize}
				\item if $x_j = A \downarrow$ then $x_{j+1} = \uparrow B$,
				\item if $x_j = \uparrow (A \otimes B)$ then $x_{j+1} = \uparrow A$,
				\item if $x_j = B \downarrow$ then $x_{j+1} = (A \otimes B) \downarrow$.
			\end{itemize}
			(see Figure \ref{fig:tensorswitching})
			\item\label{def:trip_right_par} for any par link $\tau$ with premises $A,B$ such that $\tau$ has switching $L$, we have:
			\begin{itemize}
				\item if $x_j = \uparrow (A \parr B)$ then $x_{j+1} = \uparrow A$,
				\item if $x_j = A\downarrow$ then $x_{j+1} = (A \parr B)\downarrow$,
				\item if $x_j = B\downarrow$ then $x_{j+1} = \uparrow B$.
			\end{itemize}
			and if $\tau$ evaluates under $S$ to $R$, we have:
			\begin{itemize}
				\item if $x_j = A\downarrow$ then $x_{j+1} = \uparrow A$,
				\item if $x_j = \uparrow (A\parr B)$ then $x_{j+1} = \uparrow B$,
				\item if $x_j = B\downarrow$ then $x_{j+1} = (A\parr B)\downarrow$.
			\end{itemize}
			(see Figure \ref{fig:parrswitching})
		\end{enumerate}
		%\begin{figure}[h]
		%  \centering
		%\includegraphics[width = 0.8\textwidth]{TensorSwitching.png}
		%\caption{Tensor link, $L$ switching, $R$ switching}
		%\label{fig:tensorswitching}
		%\end{figure}
		\begin{figure}[h]
			\centering
			\includegraphics[width = 0.8\textwidth]{ParrSwitching.png}
			\caption{Par link, $L$ switching, $R$ switching.}
			\label{fig:parrswitching}
		\end{figure}
		\begin{defn}
			Let $\operatorname{Pre}\call{T}(\pi,S)$ denote the set of all pretrips of $\pi$ with respect to $S$. We define an equivalence relation on this set $\sim$ where two pretrips $(x_1,...,x_n)$ and $(y_1,...,y_m)$ are equivalent if $n = m$, and there exists an integer $k$ such that $x_{i + k} = y_i$ (where $i + k$ means $\operatorname{mod} n$) for all $i = 1,...,n$.
			
			A \textbf{trip} of $\pi$ with respect to $S$ is an equivalence class of pretrips. We denote the set of all trips by $\call{T}(\pi,S)$.  If the set $\call{T}(\pi,S)$ admits more than one element, these elements are called \textbf{short trips}, and if it admits only one element, this element is the \textbf{long trip}. We refer to the statement ``for all switchings $S$, the set $\call{T}(\pi,S)$ contains exactly one element" as the \textbf{long trip condition}.
			
			A \textbf{short pretrip} is a choice of representative for a pretrip, and a \textbf{long pretrip} is a choice of representatitive of a long trip.
		\end{defn}
		
	\end{defn}
	Given a proof structure $\pi$ satisfying the long trip condition and a tensor link $\tau := (A,i,B,j, A \otimes B,k)$ of $\pi$, let $S$ be a switching of $\pi$ and $t := (x_1,...,x_n)$ be the long pretrip of $\pi$ satisfying $x_1 =A\downarrow$. Since $\pi$ satisfies the long trip condition, it must be the case that $\uparrow (A \otimes B)$ and $B\downarrow$ occur somewhere in $t$, can we determine which occurs earlier? Let $m,l > 0$ be such that $x_m = \uparrow (A \otimes B), x_l = B\downarrow$ and assume $l < m$. Say $S(\tau) = L$, then $t$ has the shape
	\begin{equation}\label{eq:sequence_left_switch_tensor}
		(A\downarrow, (A \otimes B)\downarrow, ..., B \downarrow, \uparrow A, ..., \uparrow (A \otimes B), \uparrow B, ..., A\downarrow)
	\end{equation}
	Now consider the switching given by
	\[\hat{S}(\sigma) = \begin{cases}
		S(\sigma),& \sigma \neq \tau\\
		R, & \sigma = \tau
	\end{cases}
	\]
	Then \eqref{eq:sequence_left_switch_tensor} becomes:
	\begin{equation}
		(A\downarrow, \uparrow B, ..., A\downarrow)
	\end{equation}
	which is a short pretrip, contradicting the assumption that $\pi$ satisfies the long trip condition. Thus $m < l$. We have proven (the first half) of:
	\begin{lemma}\label{lem:stays_contained_tensor}
		Let $\pi$ be a proof structure satisfying the long trip condition, $\tau := (A,i,B,j, A \otimes B,k)$ be a tensor link of $\pi$, $S$ be a switching of $\pi$ and $(x_1,...,x_n)$ the long pretrip satisfying $x_1 = A \downarrow$. If $m,l > 0$ are such that $x_m = \uparrow (A \otimes B), x_l = B \downarrow$, then
		\begin{itemize}
			\item if $S(\tau) = L$ then $m < l$,
			\item if $S(\tau) = R$ then $l < m$
		\end{itemize}
	\end{lemma}
	The proof of the other half is similar to what has already been written, however since Lemma \ref{lem:stays_contained_tensor} contradicts \cite[Lemma 2.9.1]{linearlogic} we write out the details here:
	\begin{proof}
		Say $m < l$, then $t$ has the shape
		\begin{equation}\label{eq:sequence_right_switch_tensor}
			(A\downarrow, \uparrow B, ..., \uparrow (A \otimes B), \uparrow A, ..., B \downarrow, (A \otimes B)\downarrow, ..., A\downarrow)
		\end{equation}
		Now consider the switching given by
		\[
		S'(\sigma) = 
		\begin{cases}
			S(\sigma),& \sigma \neq \tau\\
			L, & \sigma = \tau
		\end{cases}
		\]
		Then \eqref{eq:sequence_right_switch_tensor} becomes:
		\begin{equation}
			(A\downarrow, (A \otimes B)\downarrow,..., A\downarrow)
		\end{equation}
		which is a short pretrip.
	\end{proof}
	\begin{lemma}\label{lem:stays_contained_par}
		Let $\pi$ be a proof structure satisfying the long trip condition, $\tau := (A,i,B,i,A\parr B, k)$ be a par link of $\pi$, $S$ be a switching of $\pi$ and $(x_1,...,x_n)$ be the long pretrip satisfying $x_1 = A\downarrow$. If $m,l > 0$ are such that $x_m = \uparrow (A \parr B), x_l = B\downarrow$, then
		\begin{itemize}
			\item if $S(\tau) = L$ then $m < l$,
			\item if $S(\tau) = R$ then $l < m$
		\end{itemize}
	\end{lemma}
	\begin{proof}
		Exercise.
	\end{proof}
	\begin{remark}
		A long pretrip starting at position $1$ of Figure \ref{fig:stays_contained_explained_tens} necessarily moves to $1'$, granted the switching of the displayed tensor link is $L$. The long pretrip will necessarily return to this link, and moreover it will do so for the first time after leaving $1'$ either at position $2$ or $3$ (position $1$ will lead to a short trip). Lemma \ref{lem:stays_contained_tensor} states that in fact position $2$ will be taken next, then position $3$ at a later point. A similar story rings true if the switching is $R$.
		\begin{figure}[h]
			\centering
			\includegraphics{TensorLeftSwitching.png}
			\caption{Left switching}
			\label{fig:stays_contained_explained_tens}
		\end{figure}
		This gives a nice interpretation of Lemma \ref{lem:stays_contained_tensor} that long trips \emph{return to where they left} at each tensor link.
		
		The situation is a bit different for par links; we visit the premises before returning to the conclusion, see Figure \ref{fig:stays_contained_explained_par}.
		\begin{figure}[h]
			\centering
			\includegraphics[width = 7cm]{ParLeftSwitching.png}
			\caption{Left switching}
			\label{fig:stays_contained_explained_par}
		\end{figure}
	\end{remark}
	Loosely speaking, for a proof structure $\pi$ satisfying the long trip condition to ``split" at a tensor link $\tau := (A,i,B,j,A\otimes B,k)$ into two distinct proof structures $\pi_1,\pi_2$ each satisfying the long trip condition, it would necessarily be the case that any pretrip $\sigma$ of $\pi$ starting at $\uparrow A$ visits the entirety of $\call{U}(\pi_1)$ before returning to the tensor link $\tau$, lest $\pi_1$ admit a short trip. Moreover, it must be the case that $\sigma$ admits no occurrence of formulas in $\pi_2$ lest the result of removing the tensor link $\tau$ not result in disjoint proof structures. Thus, if such a link $\tau$ exists, it is \emph{maximal} in the sense that there is no other tensor link $\tau' := (A',i',B',j',A' \otimes B',k')$ where a pretrip starting at $A'$ contains the entirety of any pretrip starting at $A$. The remainder of this Section will amount to proving the converse, that any such maximal tensor link ``splits" $\pi$.
	\begin{defn}\label{def:pretrip_from_A}
		Let $\pi$ be a proof structure satisfying the long trip condition, $S$ a switching of $\pi$, and $A$ an occurrence of a formula in $\pi$. Consider the long pretrip $(x_1,...,x_n)$ satisfying $x_1 = \uparrow A$. We denote by
		\begin{equation}
			\operatorname{PTrip}(\pi,S,A,\uparrow)
		\end{equation}
		the subsequence $(x_1,...,x_m)$ of $(x_1,...,x_n)$ satisfying $x_m = A\downarrow$. We define
		\begin{equation}
			\operatorname{PTrip}(\pi, S, A, \downarrow)
		\end{equation}
		similarly.
		
		Also, for $a \in \lbrace \uparrow,\downarrow\rbrace $ we define the following set
		\begin{equation}
			\operatorname{Visit}_S(A,a) := \lbrace C \in \call{O}(\pi) \mid \uparrow C, C\downarrow \text{ occur in } \operatorname{PTrip}(\pi,S,A,a)\rbrace
		\end{equation}
		The \textbf{up empire of $A$} is the following set:
		\begin{equation}
			\operatorname{Emp}_{\uparrow}A := \lbrace C \in \call{O}(\pi) \mid \text{For all switchings }S\text{ we have } \uparrow C, C\downarrow \text{ occur in } \operatorname{PTrip}(\pi,S, A,\uparrow)\rbrace
		\end{equation}
		The \textbf{down empire of $A$} is defined symmetrically.
	\end{defn}
	\begin{lemma}
		for any formula $A$ which is a premise to either a tensor or par link, we have: $$\uparrow C \text{ occurs in } \operatorname{PTrip}(\pi,S,A,\uparrow)\qquad\text{ if and only if }\qquad C\downarrow \text{ occurs in } \operatorname{PTrip}(\pi,S,A,\uparrow)$$
	\end{lemma}
	\begin{defn}
		The \textbf{complexity} of a preformula $A$ is the sum of the number of occurrences of $\otimes$ and the number of occurrences of $\parr$ which appear in $A$. Notice that this number is invariant under choice of representative for a formula and so we also have the \textbf{complexity} of a formula.
	\end{defn}
	\begin{proof}
		For simplicity we denote $\operatorname{PTrip}(\pi, S, A, a)$ by $\operatorname{PTrip}(A, a)$.
		
		We proceed by induction on the \emph{complexity} of $A$ denoted $c(A)$. Say $A$ is atomic so that $c(A) = 0$. Then $A$ is part of an axiom link and
		\begin{equation}
			\operatorname{PTrip(A,\uparrow)} = \uparrow A, \operatorname{PTrip}(\negation A, \downarrow), A\downarrow
		\end{equation}
		If $\negation A$ is a conclusion then we have $C = \negation A$ and we are done.
		
		If $\negation A$ is a premise to a tensor link
	\end{proof}
	
	With this new terminology we now have some corollaries of Lemmas \ref{lem:stays_contained_tensor} and \ref{lem:stays_contained_par}:
	\begin{cor}\label{cor:pretrip_innards}
		Let $\pi$ be a proof structure satisfying the long trip condition, and let $S$ be a switching of $\pi$, for a formula $A$ and $a \in \lbrace \uparrow, \downarrow\rbrace$, denote $\operatorname{PTrip}(\pi,S, A, a)$ by $\operatorname{PTrip}(A,a)$:
		\begin{enumerate}
			\item if $A$ is part of an axiom link then
			\begin{equation}
				\operatorname{PTrip}(A,\uparrow) = \uparrow A, \operatorname{PTrip}(\negation A, \downarrow), A \downarrow
			\end{equation}
			\item if $\tau$ is a tensor link with conclusion $A \otimes B$:
			\begin{enumerate}
				\item if $S(\tau) = L$:
				\begin{equation}
					\operatorname{Ptrip}(A, \downarrow) = A\downarrow, \operatorname{PTrip}(A \otimes B, \downarrow), \operatorname{PTrip}(B, \uparrow), \uparrow A
				\end{equation}
				\begin{equation}
					\operatorname{PTrip}(B, \downarrow) = B\downarrow, \operatorname{PTrip}(A,\uparrow), \operatorname{PTrip}(A \otimes B, \downarrow), \uparrow B
				\end{equation}
				\begin{equation}
					\operatorname{PTrip}(A \otimes B, \uparrow) = \uparrow A \otimes B, \operatorname{PTrip}(B, \uparrow), \operatorname{PTrip}(A,\uparrow), A \otimes B \downarrow
				\end{equation}
				\item if $S(\tau) = R$:
				\begin{equation}
					\operatorname{PTrip}(A, \downarrow) = A\downarrow, \operatorname{PTrip}(B, \uparrow),\operatorname{PTrip}(A \otimes B, \downarrow), \uparrow A
				\end{equation}
				\begin{equation}
					\operatorname{PTrip}(B, \downarrow) = B\downarrow,  \operatorname{PTrip}(A \otimes B, \downarrow), \operatorname{PTrip}(A,\uparrow),\uparrow B
				\end{equation}
				\begin{equation}
					\operatorname{PTrip}(A \otimes B, \uparrow) = \uparrow A \otimes B,  \operatorname{PTrip}(A,\uparrow), \operatorname{PTrip}(B, \uparrow),A \otimes B \downarrow
				\end{equation}
			\end{enumerate}
			\item if $A$ is a premise of a par link $\tau$ with conclusion $A \parr B$:
			\begin{enumerate}
				\item if $S(\tau) = L$:
				\begin{equation}
					\operatorname{PTrip}(A,\downarrow) = A\downarrow, \operatorname{PTrip}(A \parr B, \downarrow), \uparrow A
				\end{equation}
				\begin{equation}
					\operatorname{PTrip}(B, \downarrow) = B\downarrow, \uparrow B
				\end{equation}
				\begin{equation}
					\operatorname{PTrip}(A \parr B, \uparrow) = \uparrow A \parr B, \operatorname{PTrip}(A, \uparrow), A \parr B \downarrow
				\end{equation}
				\item if $S(\tau) = R$:
				\begin{equation}
					\operatorname{PTrip}(A,\downarrow) = A\downarrow, \uparrow A
				\end{equation}
				\begin{equation}
					\operatorname{PTrip}(B, \downarrow) = B\downarrow, \operatorname{PTrip}(A \parr B, \downarrow), \uparrow B
				\end{equation}
				\begin{equation}
					\operatorname{PTrip}(A \parr B, \uparrow) = \uparrow A \parr B, \operatorname{PTrip}(B, \uparrow), A \parr B \downarrow
				\end{equation}
			\end{enumerate}
		\end{enumerate}
	\end{cor}
	In particular:
	\begin{cor}\label{cor:stays_contained_corollary}
		For any formula $A$ which is a premise to either a tensor or par link, and any $a \in \lbrace \uparrow, \downarrow \rbrace$, we have: $$\uparrow C \text{ occurs in } \operatorname{PTrip}(\pi,S,A,\uparrow)\qquad\text{ if and only if }\qquad C\downarrow \text{ occurs in } \operatorname{PTrip}(\pi,S,A,\downarrow)$$ and similarly for $\operatorname{PTrip}(\pi,S,A,\downarrow)$.
	\end{cor}
	\begin{proof}
		By induction on the length of the sequence $\operatorname{PTrip}(\pi,S,A,a)$ and appealing to Corollary \ref{cor:pretrip_innards}.
	\end{proof}
	\begin{cor}\label{cor:empire_features}
		Let $\pi$ be a proof structure satisfying the long trip condition,
		we have:
		\begin{enumerate}
			\item\label{cor:empire_features_up_ax} for any axiom link with conclusions $A, \negation A$:
			\begin{equation}
				\operatorname{Emp}_{\uparrow}A = \operatorname{Emp}_{\downarrow}(\negation A) \cup \lbrace A \rbrace
			\end{equation}
			\item\label{cor:empire_features_down_cut} for any cut link with premises $A, \negation A$:
			\begin{equation}
				\operatorname{Emp}_{\downarrow}A = \operatorname{Emp}_{\uparrow}(\negation A) \cup \lbrace A \rbrace
			\end{equation}
			\item\label{cor:empire_features_empty_int} for any tensor link with premises $A,B$:
			\begin{equation}
				\operatorname{Emp}_{\uparrow}A \cap \operatorname{Emp}_{\uparrow}B = \varnothing
			\end{equation}
			\item\label{cor:empire_features_up_tens_par} for any tensor or par link with premises $A,B$ and conclusion $C$:
			\begin{equation}
				\operatorname{Emp}_{\uparrow}C = \operatorname{Emp}_{\uparrow}A \cup \operatorname{Emp}_{\uparrow}B \cup \lbrace C \rbrace
			\end{equation}
			\item\label{cor:empire_features_down_tens} for any tensor link with premises $A,B$:
			\begin{equation}
				\operatorname{Emp}_{\downarrow}B = \operatorname{Emp}_{\uparrow}A \cup \operatorname{Emp}_{\downarrow}(A \otimes B) \cup \lbrace B \rbrace
			\end{equation}
			and similarly,
			\begin{equation}
				\operatorname{Emp}_{\downarrow}A = \operatorname{Emp}_{\uparrow}B \cup \operatorname{Emp}_{\downarrow}(A \otimes B) \cup \lbrace A\rbrace
			\end{equation}
		\end{enumerate}
	\end{cor}
	\begin{remark}
		Recall Definition \ref{def:proof_structures} that there are two types of tensor links $(A,i,B,j,A\otimes B,k)$,
		$(B,j,A,i,A\otimes B,k)$ and two types of par links $(A,i,B,j,A\parr B, k),(B,j,A,i,A\parr B,k)$. Lemmas \ref{lem:stays_contained_tensor} and \ref{lem:stays_contained_par} were only stated for links of the form $(A,i,B,j,A\otimes B,k), (A,i,B,j,A\parr B, k)$ however they hold for \emph{all} tensor and par links. One merely replaces all instances of $A \otimes B$ with $B \otimes A$, and all instances of $A \parr B$ with $B \parr A$ in the proofs.
	\end{remark}
	\begin{defn}
		Given any link $\tau$ we write $B \in \tau$ if $B$ occurs as either a premise or a conclusion of $\tau$.
		
		Let $\pi$ be a proof structure satisfying the long trip condition, and $a \in \lbrace \uparrow, \downarrow\rbrace$. The set of \textbf{links of $A$ with respect to $S$} is the set
		\begin{equation}
			\operatorname{Link}_aA := \lbrace \tau \in \operatorname{Link}\pi \mid \forall B \in \tau, B \in \operatorname{Emp}_aA \rbrace
		\end{equation}
	\end{defn}
	\begin{defn}
		Let $\pi$ be a proof structure satisfying the long trip condition and let $a \in \lbrace \uparrow, \downarrow \rbrace$. Define the set
		\begin{equation}
			\operatorname{Link}^0_{\parr,a}A := \lbrace \tau \in \operatorname{Link}\pi \mid \text{Exactly one premise of }\tau\text{ is in }\operatorname{Emp}_{a}A\rbrace
		\end{equation}
	\end{defn}
	\begin{lemma}[Realisation Lemma]\label{lem:realisation_switching}
		Let $\pi$ be a cut-free proof structure satisfying the long trip condition, let $a \in \lbrace \uparrow, \downarrow \rbrace$ and $A$ an occurrence of a formula in $\pi$. Define the following function:
		\begin{align*}
			S: \operatorname{Link}_{\parr,a}^0A &\lto \lbrace L,R\rbrace\\
			\tau &\longmapsto
			\begin{cases}
				L, & \text{if the right premise of }\tau\text{ is in }\operatorname{Emp}_{a}A\\
				R, & \text{if the left premise of }\tau\text{ is in }\operatorname{Emp}_{a}A
			\end{cases}
		\end{align*}
		and extend this to a switching $\hat{S}: \operatorname{Link}\pi \lto \lbrace L,R \rbrace$ arbitrarily. Then
		\begin{equation}
			\operatorname{Emp}_aA = \operatorname{Visit}_{\hat{S}}(A,a)
		\end{equation}
	\end{lemma}
	\begin{proof}
		We proceed by induction on the size $|\operatorname{Link}_a(A)|$ of the set $\operatorname{Link}_a(A)$. For the base case, assume $|\operatorname{Link}_a(A)| = 0$. The formula $A$ is part of an axiom link and so $\operatorname{Emp}_{\uparrow}A = A, \negation A$ and $\operatorname{Emp}_{\downarrow}A = A$, the result follows easily.
		
		Now assume that $|\operatorname{Link}_a A| = n > 0$ and the result holds for any formula $B$ such that $|\operatorname{Link}_aB| < n$. First say $a = \uparrow$, and $A$ is a conclusion of either a tensor or a par link
		\[
		\begin{tikzcd}[column sep = tiny]
			A_1\arrow[dr,dash] && \negation A_2\arrow[dl,dash]\\
			& A
		\end{tikzcd}
		\]
		where $A = A_1 \otimes A_2$ or $A = A_1 \parr A_2$. By \eqref{cor:empire_features_up_tens_par} we have
		\begin{align*}
			\operatorname{Emp}_{\uparrow}A &= \operatorname{Emp}_{\uparrow}A_1 \cup \operatorname{Emp}_{\uparrow}A_2 \cup \lbrace A\rbrace\\
			&= \operatorname{Visit}_{\hat{S}}(A_1,\uparrow) \cup \operatorname{Visit}_S(A_2,\uparrow) \cup \lbrace A \rbrace\\
			&= \operatorname{Visit}_{\hat{S}}(A, \uparrow)
		\end{align*}
		where the second equality follows from the inductive hypothesis.
		
		Assume $A$ is part of an axiom link. By \eqref{cor:empire_features_up_ax}
		\begin{equation}
			\operatorname{Emp}_{\uparrow}A = \operatorname{Emp}_{\downarrow}(\negation A) \cup \lbrace A \rbrace
		\end{equation}
		with
		\begin{equation}
			|\operatorname{Link}_{\uparrow}A| = |\operatorname{Link}_{\downarrow}(\negation A)|
		\end{equation}
		Since $|\operatorname{Link}_{\downarrow}(\negation A)| > 0$ we necessarily have that $\negation A$ is not a conclusion. Thus, since $\pi$ is cut-free, $A$ is connected to an occurrence $\negation A$ which is a premise to either a tensor link or a par link. In the case of the former, we have:
		\[
		\begin{tikzcd}[column sep = tiny]
			C\arrow[dr,dash] && \negation A\arrow[dl,dash]\\
			& C \otimes \negation A
		\end{tikzcd}
		\]
		then by \eqref{cor:empire_features_down_tens}:
		\begin{align*}
			\operatorname{Emp}_{\downarrow}(\negation A) &= \operatorname{Emp}_{\uparrow}C \cup \operatorname{Emp}_{\downarrow}(C \otimes \negation A) \cup \lbrace \negation A\rbrace\\
			&= \operatorname{Visit}_{\hat{S}}(C,\uparrow) \cup \operatorname{Visit}_{\hat{S}}(C \otimes \negation A,\downarrow) \cup \lbrace \negation A\rbrace\\
			&= \operatorname{Visit}_{\hat{S}}(\negation A, \downarrow)
		\end{align*}
		where the second equality follows from the inductive hypothesis.
		
		If $\negation A$ is a premise of a par link
		\[
		\begin{tikzcd}[column sep = tiny]
			C\arrow[dr,dash] && \negation A\arrow[dl,dash]\\
			& C \parr \negation A
		\end{tikzcd}
		\]
		then by construction of $\hat{S}$, where we use the specific definition of $S$ for the first time,
		\begin{align*}
			\operatorname{Emp}_{\downarrow}(\negation A) &= \lbrace \negation A\rbrace\\
			&= \operatorname{Visit}_{\hat{S}}(\negation A,\downarrow)
		\end{align*}
		The case when $a = \downarrow$ is exactly similar and so we omit the proof.
	\end{proof}
	\begin{defn}
		A tensor or par link is \textbf{terminal} if it is a conclusion.
	\end{defn}
	\begin{cor}\label{lem:par_link_existence}
		Let $\pi$ be a cut-free proof structure satisfying the long trip condition. Let
		\[
		\tau := \begin{tikzcd}[column sep = tiny]
			A\arrow[dr,dash] && B\arrow[dl,dash]\\
			& A \otimes B
		\end{tikzcd}
		\]
		be a terminal tensor link of $\pi$. Then $\pi$ admits a par link
		\[
		\sigma := \begin{tikzcd}[column sep = tiny]
			C\arrow[dr,dash] && D\arrow[dl,dash]\\
			& C \parr D
		\end{tikzcd}
		\]
		such that either $C \in \operatorname{Emp}_{\uparrow}A$ and $D \in \operatorname{Emp}_{\uparrow}B$ or $C \in \operatorname{Emp}_{\uparrow}B$ and $D \in \operatorname{Emp}_{\uparrow}A$ if and only if for any switching $S$ of $\pi$ we have that either
		\[\operatorname{Emp}_{\uparrow}A \subsetneq \operatorname{Visit}_S(A,\uparrow)\qquad\text{or}\qquad \operatorname{Emp}_{\uparrow}B \subsetneq \operatorname{Visit}_S(B,\uparrow)\]
	\end{cor}
	\begin{proof}
		Say $\pi$ admitted $\sigma$ and $C \in \operatorname{Emp}_{\uparrow}A$ and $D \in \operatorname{Emp}_{\uparrow}B$. If the switching $S$ is such that $S(\tau) = L$ then $C \parr D \in \operatorname{Visit}_S(B)\setminus \operatorname{Emp}_{\uparrow}B$ and if $S(\tau) = R$ then $C \parr D \in \operatorname{Visit}_S(A) \setminus \operatorname{Emp}_{\uparrow}A$. The other case is similar.
		
		Conversely, say $\pi$ admits no such par link $\sigma$, that is, assume
		\begin{equation}
			\operatorname{Link}^0_{\parr,\uparrow}(A) \cap \operatorname{Link}^0_{\parr,\uparrow}(B) = \varnothing
		\end{equation}
		Then there is by Lemma \ref{lem:realisation_switching} a well defined function $$S: \operatorname{Link}_{\parr,\uparrow}^0(A) \cup \operatorname{Link}^0_{\parr,\uparrow}(B) \lto \lbrace L, R \rbrace$$ which extends to a switching $\hat{S}$ such that
		\begin{equation}
			\operatorname{Emp}_{\uparrow}A = \operatorname{Visit}_{\hat{S}}(A,\uparrow)\qquad\text{and}\qquad \operatorname{Emp}_{\uparrow}B = \operatorname{Visit}_{\hat{S}}(B,\uparrow)
		\end{equation}
	\end{proof}
	
	\begin{lemma}[Separation Lemma]
		A cut-free proof structure $\pi$ satisfying the long trip condition, with only tensor links amongst its conclusions admits a tensor link
		\[
		\tau := \begin{tikzcd}[column sep = tiny]
			A\arrow[dr,dash] && B\arrow[dl,dash]\\
			& A \otimes B
		\end{tikzcd}
		\]
		satisfying
		\begin{equation}
			\call{O}(\pi) = \operatorname{Emp}_{\uparrow}A \cup \operatorname{Emp}_{\uparrow}B \cup \lbrace A \otimes B\rbrace
		\end{equation}
		Moreover, removing $A \otimes B$ results in a disconnected graph with each component a proof structure satisfying the long trip condition.
	\end{lemma}
	\begin{proof}
		Consider the set of tensor links $\operatorname{Link}_{\otimes}(\pi)$ of $\pi$. We endow this with the following partial order $\leq$: a pair of links:
		\[
		\sigma := 
		\begin{tikzcd}[column sep = tiny]
			A\arrow[dr,dash] && B\arrow[dl,dash]\\
			& A \otimes B
		\end{tikzcd}
		%
		\qquad
		%
		\rho :=
		\begin{tikzcd}[column sep = tiny]
			C\arrow[dr,dash] && D\arrow[dl,dash]\\
			& C \otimes D
		\end{tikzcd}
		\]
		are such that $\tau \leq \sigma$ if $\operatorname{Emp}_{\uparrow}A \cup \operatorname{Emp}_{\uparrow}B \subseteq \operatorname{Emp}_{\uparrow}C \cup \operatorname{Emp}_{\uparrow}D$. Let $\tau$ (with conclusion $A \otimes B$ say) be a tensor link maximal with respect to $\leq$. We show that $\tau$ satisfies the required property.
		
		Say $\call{O}(\pi) \neq \operatorname{Emp}_{\uparrow}A \cup \operatorname{Emp}_{\uparrow}B \cup \lbrace A \otimes B\rbrace$. Then by Lemma \ref{lem:par_link_existence} there exists a par link
		\[
		\sigma := \begin{tikzcd}[column sep = tiny]
			C\arrow[dr,dash] && D\arrow[dl,dash]\\
			& C \parr D
		\end{tikzcd}
		\]
		such that either $C \in \operatorname{Emp}_{\uparrow}A$ and $D \in \operatorname{Emp}_{\uparrow}B$ or $C \in \operatorname{Emp}_{\uparrow}B$ and $D \in \operatorname{Emp}_{\uparrow}A$. We show the proof in the case of the former. Since $\pi$ admits no terminal par links, this link is above a tensor link
		\[
		\rho := \begin{tikzcd}[column sep = tiny]
			E\arrow[dr,dash] && F\arrow[dl,dash]\\
			& E \otimes F
		\end{tikzcd}
		\]
		Notice that if $\rho = \tau$, then either $C \parr D \in \operatorname{Emp}_{\uparrow}A$ or $C \parr D \in \operatorname{Emp}_{\uparrow}B$ which in either case implies $\operatorname{Emp}_{\uparrow}A \cap \operatorname{Emp}_{\uparrow}B \neq \varnothing$, contradicting Corollary \ref{cor:empire_features}, \ref{cor:empire_features_empty_int}, and so $\rho \neq \tau$. Without any loss of generality, assume that $\sigma$ sits above $F$. Let $S$ be a switching of $\pi$ so that $\operatorname{Emp}_{\uparrow}F = \operatorname{Visit}_S(F,\uparrow)$ and so that $S(\sigma) = L$, which exists by Lemma \ref{lem:par_link_existence}. Let $t = (x_1,...,x_n)$ be the long pretrip of $\pi$ with respect to $S$ satisfying $x_1 = F\uparrow$. We have by Lemma \ref{lem:stays_contained_par} that $t$ takes the following shape:
		\begin{equation}\label{eq:condemning_shape}
			\uparrow F, ..., \uparrow(C \parr D), \uparrow C, ..., D\downarrow, \uparrow D, ..., C\downarrow, (C \parr D)\downarrow, ..., F\downarrow,...
		\end{equation}
		We have that $D \in \operatorname{Emp}_{\uparrow}B$ so for simplicity, rewrite \eqref{eq:condemning_shape} as $t' = (x_{1 + k},...,x_{n+k})$ for some $k > 0$ (where $i + k$ means $i + k\operatorname{mod} n$) so that $t$ takes the shape
		\begin{equation}\label{eq:condemning_shape_altered}
			..., \uparrow F, ..., \uparrow(C \parr D), \uparrow C, ..., D\downarrow, \uparrow D, ..., C\downarrow, (C \parr D)\downarrow, ..., F\downarrow,...
		\end{equation}
		with $\uparrow B$ occurring to the left of $D \downarrow$ and $B \downarrow$ occurring to the right of $\uparrow D$.
		We have that $C \not\in \operatorname{Emp}_{\uparrow}B$ and so by Corollary \ref{cor:stays_contained_corollary}:
		\begin{equation}
			\uparrow B \text{ occurrs in }\uparrow C,..., D\downarrow\text{ and } B\downarrow\text{ occurrs in }\uparrow D, ..., C\downarrow
		\end{equation}
		However, this implies that $B \in \operatorname{Visit}_S(F,\uparrow)$ which by Lemma \ref{lem:realisation_switching} implies $B \in \operatorname{Emp}_{\uparrow}F$.
		
		By reversing the switching of $\sigma$ and interchanging the rolls of $C,D$ in the above argument, we also have that $A \in \operatorname{Emp}_{\uparrow}F$, contradicting the maximality of $\tau$. This proves the first claim.
		
		For the second claim, since $\call{O}(\pi) = \operatorname{Emp}_{\uparrow}A \cup \operatorname{Emp}_{\uparrow}B \cup \lbrace A \otimes B\rbrace$ we have by Lemma \ref{lem:par_link_existence} that
		\begin{equation}
			\operatorname{Link}_{\parr,\uparrow}^0(A\otimes B) = \varnothing
		\end{equation}
		and we saw in the proof of Lemma \ref{lem:realisation_switching} that a switching $S$ which realises $\operatorname{Emp}_{\uparrow}A$ is given by setting all switchings arbitrarily except for those in $\operatorname{Link}_{\parr,\uparrow}^0(A\otimes B)$. This means that for any switching $S$ of $\pi$:
		\begin{equation}
			\operatorname{Visit}_{S}(A,\uparrow) = \operatorname{Emp}_{\uparrow}A \qquad\text{and}\qquad \operatorname{Visit}_{S}(B,\uparrow) = \operatorname{Emp}_{\uparrow}B
		\end{equation}
		which is to say the two subproof structures given by removing $A \otimes B$ never admit a short trip, that is, they each satisfy the long trip condition.
	\end{proof}
	\begin{thm}[The Sequentialisation Theorem]\label{thm:sequentialisation}
		A proof structure $\pi$ (possibly with cuts) satisfies the long trip condition if and only $\pi$ is a proof net.
	\end{thm}
	\begin{proof}
		First assume that $\pi$ is cut-free.
		
		We proceed by induction on the size $|\operatorname{Link}\pi|$ of the set $\operatorname{Link}\pi$. If there this is zero then $\pi$ consists of a single axiom link and so the result is clear.
		
		For the inductive step, we consider two cases, first say $\pi$ admits a par link for a conclusion. Then removing this par link clearly results in two cut-free subproof structures satsifying the long trip condition and so the result follows from the inductive hypothesis. If no such terminal par link exists, then by the Separation Lemma there exists some tensor link in the conclusion for which we can remove and apply the inductive hypothesis.
		
		Now say that $\pi$ contained cuts. We replace each cut with a tensor link to create a new proof $\zeta$. That there exists a proof $\Xi$ which maps to $\zeta$ follows from the part of the result proved already as $\zeta$ is cut-free. We adapt $\Xi$ appropriately by replacing $\otimes$-rules by $\operatorname{cut}$-rules and we are done.
	\end{proof}
	\section{Cut}
	\begin{defn}
		A \emph{substructure} is a subgraph a proof structure. Note: a substructure need not be a proof structure itself.
	\end{defn}
	
	\begin{defn}\label{def:cut_reduction}
		Cut-reduction $\lto_{\operatorname{cut}}$ is the smallest, compatible equivalence relation on the set of all substructures containing the following generators.
		\begin{itemize}
			\item \textbf{Axiom redex}
			\begin{equation}\label{eq:cut_red_ax}
				\begin{tikzcd}
					& & \vdots\arrow[d,dash]\\
					A\arrow[r,dash, bend left]\arrow[d,dash] & \negation A\arrow[r,dash, bend right] & A\\
					\vdots
				\end{tikzcd}
				\lto_{\cut}
				\begin{tikzcd}
					\vdots\arrow[d,dash]\\
					A\arrow[d,dash]\\
					\vdots
				\end{tikzcd}
			\end{equation}
			\item \textbf{Tensor-par redex}
			\begin{equation}\label{eq:cut_red_tens}
				\begin{tikzcd}[column sep = tiny]
					A\arrow[dr,dash] && B\arrow[dl,dash] & \negation A\arrow[dr,dash] && \negation B\arrow[dl,dash]\\
					& A \otimes B\arrow[rrr,dash, bend right] & & & \negation A \parr \negation B
				\end{tikzcd}
			\end{equation}
			$\lto_{\operatorname{cut}}$
			\begin{equation}
				\begin{tikzcd}
					A\arrow[rr,dash,bend right] & B\arrow[rr,dash,bend right] & \negation A & \negation B
				\end{tikzcd}
			\end{equation}
		\end{itemize}
		The reflexive, symmetric, transitive closure of cut-reduction is \textbf{cut-equivalence}, and is denoted $\sim_{\operatorname{cut}}$.
	\end{defn}
	\begin{proposition}[Church-Rosser]\label{prop:church_rosser}
		If $\pi_1$ is a proof structure and $\pi_1 \lto_{\operatorname{cut}} \pi_2, \pi_1 \lto_{\operatorname{cut}} \pi_3$ then there exists a proof structure $\pi_4$ such that $\pi_2 \lto_{\operatorname{cut}} \pi_4, \pi_3 \lto_{\operatorname{cut}} \pi_4$.
	\end{proposition}
	\begin{proof}
		The key observation is that reducing any redex in a proof does not eliminate any other redex.
	\end{proof}
	\begin{defn}\label{def:permutations}
		Let $\pi$ be a proof net with $n$ axiom links. Assume the occurrences of the axioms of $\pi$ have been labelled by integers $1,...,2n$. For each $1 \leq m \leq 2n$ let $\alpha_{\pi}(m)$ denote the integer such that the formulas labelled $m,\alpha_{\pi}(m)$ are connected by an axiom link in $\pi$. This defines a permutation (which is a disjoint union of transpositions) which we call the \textbf{axiom link permutation associated to $\pi$}.
		
		There is another permutation of $\lbrace 1,...,2n\rbrace$ defined by $\pi$. Let $S$ be a switching of $\pi$ and for each $1 \leq m \leq 2n$ let $\beta_{\pi}^S(m)$ denote the integer such that the first occurrence of any $\uparrow A_1,...,\uparrow A_{2n}$ in $\operatorname{PTrip}(\pi,S,A_m,\downarrow)$ (Definition \ref{def:pretrip_from_A}) is $\uparrow A_{\beta_{\pi}(m)}$.
		
		The set of all premutations of the second form is denoted:
		\begin{equation}
			\Sigma(\pi) := \lbrace \beta_{\pi}^S \mid S\text{ is a switching of }\pi\rbrace
		\end{equation}
		We will often denote elements of $\beta_{\pi}^S \in \Sigma(\pi)$ simply by $\beta$.
	\end{defn}
	\begin{remark}
		A provable formula $A$ uniquely defines a cut-free proof structure with soul conclusion $A$ up to the axiom links. For instance, the formula 
		\begin{equation}
			(A \otimes A) \parr (\negation A \parr \negation A)
		\end{equation}
		corresponds to the sub-proof structure given by ignoring the dashed lines and the axiom links of \eqref{eq:counter_one}:
		\begin{equation}\label{eq:counter_one}
			\begin{tikzcd}[column sep = tiny]
				A \arrow[rr,dash, bend left] \arrow[rd,dash]\arrow[rrrrrr,bend left,dash,dashed] &                         & \negation A \arrow[rr,dash,bend left, dashed] \arrow[rrrd,dash] &                                                     & A \arrow[rr,dash, bend left] \arrow[llld,dash] &                                           & \negation A \arrow[ld,dash] \\
				& A \otimes A \arrow[rrd,dash] &                        &                                                     &                                    & \negation A \parr \negation A \arrow[lld,dash] &                        \\
				&                         &                        & (A \otimes A) \parr (\negation A \parr \negation A) &                                    &                                           &                       
			\end{tikzcd}
		\end{equation}
		The proof net given by ignoring the dashed lines in \eqref{eq:counter_one} corresponds to the permutation $(12)(34)$, and that given by ignoring the axiom links and including the dashed lines is $(14)(23)$.
		
		Note: in the notation of \cite{multiplicatives} this sub-proof structure would be denoted $T_{A'}$, where $A' = (A \otimes A) \parr (\negation A \parr \negation A)$.
	\end{remark}
	\begin{defn}
		Let $\pi$ be a proof net possibly containing cut links. A \textbf{reduction sequence} is a sequence
		\begin{equation}
			\pi = \pi_0 \lto_{\operatorname{cut}} \pi_1 \lto_{\operatorname{cut}} \hdots \lto_{\operatorname{cut}} \pi_n
		\end{equation}
		with $\pi_n$ cut-free.
	\end{defn}
	\begin{lemma}\label{lem:red_sequence_existence}
		Every proof net $\pi$ admits a reduction sequence.
	\end{lemma}
	\begin{proof}
		Given a cut link $\tau := (A,i,\negation A, j)$ in $\pi$, the \textbf{complexity of $\tau$}, $c(\tau)$ is the sum of the number of occurrences of $\otimes$ and the number of occurrences of $\parr$ in $A$. We proceed by induction on the maximum of the complexities of all cut links in $\pi$.
		
		Say this maximum is $0$. Then all cut-links have the shape of \eqref{eq:cut_red_ax} (using the fact that $\pi$ is a \emph{proof net}, not merely a proof structure). We can use \eqref{eq:cut_red_ax} finitely many times (in any order) to deduce the result.
		
		Now say the maximum is $n > 0$. We then apply \eqref{eq:cut_red_tens} to all cut links of complexity $n$ (in any order) to obtain a new proof structure $\zeta$. It follows from Lemmas \ref{fig:stays_contained_explained_tens}, \ref{fig:stays_contained_explained_par} that $\pi$ satisfying the long trip condition ensures that $\zeta$ does, and so we may apply the inductive hypothesis.
	\end{proof}
	\begin{defn}
		Let $\operatorname{Red}\pi$ denote the set of all reduction sequences of $\pi$. The \textbf{length} $l(\underline{x})$ of a reduction sequence $\underline{x} \in \operatorname{Red}\pi$ is the length of the sequence $\underline{x}$.
	\end{defn}
	\begin{cor}\label{cor:stable_length}
		The length of a reduction path is independent of the choice of reduction path.
	\end{cor}
	\begin{proof}
		The proof is purely geometric. Let
		\begin{equation}
			\underline{x} := ( \pi = \pi_1 \lto_{\operatorname{cut}} \hdots \lto_{\operatorname{cut}} \pi_n)
		\end{equation}
		be the reduction path described by Lemma \ref{lem:red_sequence_existence} and let
		\begin{equation}
			\underline{y} := (\pi = \zeta_0 \lto_{\operatorname{cut}} \hdots \lto_{\operatorname{cut}} \zeta_n)
		\end{equation}
		be any other reduction sequence. By Lemma \ref{prop:church_rosser} we have $\pi_n = \zeta_n$. Also using \ref{prop:church_rosser}, the pair of reduction paths can be completed to some grid defined by a subset of $\bb{N} \times \bb{N}$. All paths $p$ consisting of only upwards steps or right steps such that $p$ is bound to this grid have the same length and so $l(\underline{x}) = l(\underline{y})$.
	\end{proof}
	\begin{defn}\label{def:normal_form}
		The proof of Corollary \ref{cor:stable_length} shows that every reduction path of a proof net $\pi$ leads to the same cut-free proof $\zeta$. We call $\zeta$ the \textbf{normal form} of $\pi$.
	\end{defn}
	\begin{cor}
		Multiplicative proof nets are strongly normalising.
	\end{cor}
	\section{Orthogonality}
	
	
	
	
	
	
	
	
	
	\begin{proposition}
		Let $\pi$ be a proof structure, then $\pi$ is a proof net if and only if for all $\beta \in \Sigma(\pi)$ the permutation $\alpha_{\pi}\beta$ is cyclic.
	\end{proposition}
	\begin{lemma}\label{lem:pars_or_not}
		Let $\pi$ be a proof net with conclusions $A_1,...,A_n$ and let $\zeta$ be a proof net obtained by beginning with $\pi$ and in any order forming par links which connect all the conclusions $A_1,...,A_n$ so that $\zeta$ has conclusions $B_1,...,B_m$ where $m \leq n$ and each $B_i$ is constructed only by $\parr$ and a subset of the formulas $A_1,...,A_n$. Then $\Sigma(\pi) = \Sigma(\zeta)$.
	\end{lemma}
	\begin{proof}
		Easy proof by induction on the integer given by the number of par links in $\zeta$ minus the number of par links in $\pi$.
	\end{proof}
	\begin{example}\label{ex:counter}
		\begin{figure}[h]
			\centering
			\includegraphics[width = 17cm]{Permutations.png}
			\caption{The set $\Sigma(\pi_1)$}
			\label{fig:counter_one}
		\end{figure}
		Let $\pi_1$ be as defined as follows:
		\begin{equation}
			\pi_1 = \begin{tikzcd}
				A \arrow[r, dash,bend left] \arrow[rd,dash] & \negation A             & A \arrow[r, dash,bend left] \arrow[ld,dash] & \negation A                         & A \arrow[r, dash,bend left] \arrow[rd,dash] & \negation A             & A \arrow[r, dash,bend left] \arrow[ld,dash] & \negation A \\
				& A \otimes A \arrow[rrd,dash] &                                   &                                     &                                   & A \otimes A \arrow[lld,dash] &                                   &             \\
				&                         &                                   & (A \otimes A) \otimes (A \otimes A) &                                   &                         &                                   &            
			\end{tikzcd}
		\end{equation}
		written as a permutation we have: $\alpha_{\pi_1} = (12)(34)(56)(78)$.
		
		We see from Figure \ref{ex:counter} that we can write down $\Sigma(\pi_1)$:
		\begin{equation}
			\Sigma(\pi_1) = \lbrace (1375), (1357), (1753), (1573)\rbrace
		\end{equation}
		\begin{remark}\label{rmk:trunk}
			Clearly, the set $\Sigma(\pi_1)$ only depends on the typing tree (the cut-free sub-proof structure corresponding to a provable formula $A$ as described in Remark \ref{rmk:trunk}), which in turn only depends on the formula $A$.
		\end{remark}
	\end{example}
	\section{Geometry of Interaction Zero}
	Geometry of interaction zero requires that all formulas occurring in axiom links are atomic, every proof structure not of this form can be related to one which is by $\eta$-equivalence:
	\begin{defn}[$\eta$-equivalence]
		Let $\sim_{\eta}$ denote the smallest, compatible, equivalence relation on the set of all proof structures satisfying the following for all labelled formulas $A,B$:
		\begin{equation}
			\begin{tikzcd}[column sep = small]
				A\otimes B \arrow[r,dash, bend left] & \negation A \parr \negation B
			\end{tikzcd}
			\sim_{\eta}
			\begin{tikzcd}
				A\arrow[rrr,bend left, dash]\arrow[dr,dash] & & B\arrow[rrr,dash,bend left]\arrow[dl,dash] & \negation A\arrow[dr,dash] & & \negation B\arrow[dl,dash]\\
				& A \otimes B & & & \negation A \parr \negation B
			\end{tikzcd}
		\end{equation}
	\end{defn}
	
	
	
	
	We begin with the following crucial observation:
	\begin{lemma}\label{lem:GoI_lemma}
		Let $\pi$ be a proof structure and assume there is a cut in $\pi$ of $A$ against $\negation A$. Write
		\begin{equation}
			A :=  A_1 \boxtimes_1 \hdots \boxtimes_{n-1} A_n
		\end{equation}
		where for each $i$ we have $\boxtimes_i \in \lbrace \otimes, \parr\rbrace$. Let $\zeta$ be a proof structure equivalent to $\pi$ under cut-reduction which is obtained by performing all tensor-par reductions (Definition \ref{def:cut_reduction}), notice this necessarily terminates. Then for each $i$ there exists a cut of $A_i$ against $\negation A_i$.
	\end{lemma}
	\begin{proof}
		By induction on $n$, where the base case follows trivially and the inductive step by inspection of \eqref{eq:cut_red_tens}.
	\end{proof}
	\begin{example}
		We denote by $\pi$ the following proof net.
		\begin{equation}\label{eq:standard_example}
			\begin{tikzcd}
				{A}\arrow[r,bend left, dash] & {\negation A}\arrow[dr] & & {A}\arrow[dl]\arrow[r,bend left, dash] & {\negation A} & {A}\arrow[rr,bend left]\arrow[dr] & & {\negation A}\arrow[dl]\\
				& & {\negation A \otimes A}\arrow[rrrr, bend right, dash] & & & & {A \parr \negation A}
			\end{tikzcd}
		\end{equation}
		Reducing the cut-redex we obtain:
		\begin{equation}
			\begin{tikzcd}
				{A}\arrow[r,bend left,dash] & {\negation A}\arrow[rrrr,bend right,dash] & & { A}\arrow[r,bend left,dash]\arrow[rrr,bend right,dash] & {\negation A} & {A}\arrow[r,bend left,dash] & {\negation A}
			\end{tikzcd}
		\end{equation}
		The cut link in $\pi$ consists of $\negation A \otimes A$ and $A \parr \negation A$, the order of these formulas was respected by the cut-reduction step, in the sense made precise by Lemma \ref{lem:GoI_lemma}, as seen in this example as the resulting cuts are between ``the two first elements" and ``the two last elements", ie, between $\negation A$ and $A$, and between $A$ and $\negation A$.
	\end{example}
	Lemma \ref{lem:GoI_lemma} will be used to prove both Geometry of Interaction zero and Geometry of Interaction One.
	\begin{defn}
		A proof net is \textbf{neat} if
		\begin{itemize}
			\item Only atomic formulas appear in axiom links,
			\item Every name of every atomic labelled formula is distinct.
		\end{itemize}
	\end{defn}
	\begin{defn}\label{def:GoI_permutation}
		Let $\pi$ be a neat proof net. Denote by $\operatorname{Var}(\pi)$ the set of all names of labelled formulas, ie, if $x:A$ appears in $\pi$, then $x \in \operatorname{Var}(\pi)$. We describe a bijection $\delta_\pi: \operatorname{Var}(\pi) \lto \operatorname{Var}(\pi)$ by giving an element of $S_m$, the permutation group on $m$ elements, where $m$ is the number of elements in $\operatorname{Var}(\pi)$. Towards this end, enumerate the elements of $\operatorname{Var}(\pi)$. Furthermore, if $\zeta$ is the normal form of $\pi$ (Definition \ref{def:normal_form}) obtained from $\pi$ by cut-elimination, assume the enumeration of $\operatorname{Var}(\pi)$ is such that the variables numbered $1,...,m'$ are exactly those which exist in $\zeta$.
		
		For each cut link in $\pi$, say of $A := A_{n_1} \boxtimes_{n_1}\hdots \boxtimes_{n_{r-1}} A_{n_r}$ against $\negation A$, for every $i \in \lbrace n_1,...,n_r\rbrace$ let $\gamma(i)$ denote the integer such that the the $i^{\text{th}}$ element of $\negation A$ is labelled by the variable name corresponding to $\gamma(i)$.
		
		For each $i$ let $d_i$ denote the least integer such that 
		\begin{equation}
			(\alpha_{\pi} \circ \gamma_{\pi})^{d_i}(i) \in \lbrace 1,...,m'\rbrace    
		\end{equation}
		Notice that such an integer $d_i$ always exists as $\pi$ is a proof net, so the permutation $\alpha_{\pi} \circ \gamma_{\pi}$ is cyclic.
		
		We then define the following permutation:
		\begin{align*}
			\delta_{\pi}: \lbrace 1,...,m\rbrace &\lto \lbrace 1,...,m\rbrace\\
			i &\longmapsto \text{the integer }j\text{ such that } (\alpha_{\pi} \circ \gamma_{\pi})^{d_i}(i) = j
		\end{align*}
	\end{defn}
	\begin{thm}[Geometry of Interaction]
		Let $\pi$ be a proof net possibly with cuts and let $\zeta$ be the normal form of $\pi$ (Definition \ref{def:normal_form}). Then
		\begin{equation}
			\delta_\pi = \alpha_{\zeta}
		\end{equation}
	\end{thm}
	\begin{proof}
		Follows from construction and Lemma \ref{lem:GoI_lemma}.
	\end{proof}
	\section{Geometry of Interaction One}
	We now consider conclusion-conclusion paths in a proof net and associate to this collection a bounded linear operator upon a Hilbert space. First, we recall some general theory from functional analysis.
	\subsection{Internalisation of direction sum and tensor product}\label{sec:interal_sum_tens}
	We focus on the specific Hilbert space $\bb{H} = \ell^2$ of sequences $\underline{z} = (z_0,z_1,...)$ of complex numbers which are square summable, ie, $\sum_{n = 0}^\infty |z_n|^2$ converges. This has an inner product defined by
	\begin{equation}
		\big\langle \underline{z}, \underline{w}\big\rangle = \sum_{n = 0}^\infty z_n\overline{w}_n
	\end{equation}
	In fact, the sum $\bb{H}^m$ of $m$ copies of $\bb{H}$ also has an inner product structure, defined by
	\begin{equation}\label{eq:bijection}
		\Big\langle \big(\underline{z}^1,...,\underline{z}^m\big),\big(\underline{w}^1,...,\underline{w}^m\big)\Big\rangle_{\bb{H}^m} = \sum_{j = 1}^m\langle \big(\underline{z}^j,\underline{w}^j\big)\rangle_{\bb{H}}
	\end{equation}
	We fix the standard basis for $\ell^2$ consisting of sequences $\underline{e}^i$ such that all entries are equal to $0$ except for the $i^{\operatorname{th}}$ which is equal to $1$. We note that this basis is countable. A basis for $\ell^2 \oplus \ell^2$ is given by all $(\underline{e}^i, 0)$ and $(0,\underline{e}^i)$ which is also countable, thus, bijections $\alpha: \bb{N} \coprod \bb{N} \lto \bb{N}$ induce isomorphisms $\ell^2 \lto \ell^2 \oplus \ell^2$. More explicitly, if $\alpha: \bb{N} \coprod \bb{N} \lto \bb{N}$ is such a bijection then there exists injective functions $\alpha_1,\alpha_2: \bb{N} \lto \bb{N}$ which make the following diagram commute.
	\begin{equation}
		\begin{tikzcd}
			\bb{N}\arrow[dr,"{\alpha_1}"]\arrow[d,rightarrowtail]\\
			\bb{N}\arrow[r,"{\alpha}"] \coprod \bb{N}\arrow[r,"{\alpha}"] & \bb{N}\\
			\bb{N}\arrow[ur,swap,"{\alpha_2}"]\arrow[u,rightarrowtail]
		\end{tikzcd}
	\end{equation}
	The induced isomorphism $\hat{\alpha}: \ell^2 \lto \ell^2 \oplus \ell^2$ is then given by the following explicit formula, where $z = \sum_{i = 0}^\infty z_i\underline{e}^i$:
	\begin{equation}\label{eq:alpha_hat}
		\hat{\alpha}(z) = \sum_{i = 0}^\infty\Big( z_{\alpha_1(i)}\underline{e}^i,z_{\alpha_2(i)}\underline{e}^i\Big)
	\end{equation}
	The following calculation shows that $\hat{\alpha}$ is an isometry:
	\begin{align*}
		\big\langle \hat{\alpha}(\underline{z}), \hat{\alpha}(\underline{w})\big\rangle &= \Big\langle \sum_{i = 0}^\infty \big(z_{\alpha_1(i)}\underline{e}^i, z_{\alpha_2(i)}\underline{e}^i\big), \sum_{i = 0}^\infty \big( w_{\alpha_1(i)}\underline{e}^i, w_{\alpha_2(i)}\underline{e}^i\big) \Big\rangle\\
		&= \Big\langle \sum_{i = 0}^\infty z_{\alpha_1(i)}\underline{e}^i, \sum_{i = 0}^\infty w_{\alpha_1(i)}\underline{e}^i\Big\rangle + \Big\langle \sum_{i = 0}^\infty z_{\alpha_2(i)}\underline{e}^i, \sum_{i = 0}^\infty w_{\alpha_2(i)}\underline{e}^i \Big\rangle\\
		&= \sum_{i = 0}^\infty z_{\alpha_1(i)}\overline{w}_{\alpha_i(i)} + \sum_{i = 0}^\infty z_{\alpha_2(i)}\overline{w}_{\alpha_2(i)}\\
		&= \sum_{i = 0}^\infty z_i\overline{w}_i\\
		&=  \langle \underline{z},\underline{w}\rangle
	\end{align*}
	We claim that \eqref{eq:alpha_hat} can also be written as $\hat{\alpha}(z) = \big(p^\ast(z), q^\ast(z)\big)$ for operators $p,q: \ell^2 \lto \ell^2$ determined by continuity and
	\begin{equation}
		p(\underline{e}^i) = \underline{e}^{\alpha_1(i)},\qquad q(\underline{e}^i) = \underline{e}^{\alpha_2(i)}
	\end{equation}
	These maps are norm preserving and so are clearly bounded, thus we have well defined linear operators. It can be established by a direct calculation that these have adjoints respectively determined by continuity and
	\begin{align}
		p^\ast(\underline{e}^i) &= \underline{e}^{\alpha_1^{-1}(i)}\text{ if }\alpha_1^{-1}(i)\text{ exists, otherwise }p^\ast(\underline{e}^i) = 0\\
		q^\ast(\underline{e}^i) &= \underline{e}^{\alpha_2^{-1}(i)}\text{ if }\alpha_2^{-1}(i)\text{ exists, otherwise }p^\ast(\underline{e}^i) = 0
	\end{align}
	For example: let $w = \sum_{i = 0}^\infty w_i\underline{e}^i$:
	\begin{align*}
		\big\langle p(z),w \big\rangle = \sum_{i = 0}^\infty z_i\overline{w}_{\alpha_1(i)} = \big\langle z, p^\ast(w)\big\rangle
	\end{align*}
	We thus have the formula:
	\begin{equation}
		\hat{\alpha} = p^\ast \oplus q^\ast
	\end{equation}
	In a similar way, given any $n > 0$ along with a bijection $\alpha: \bb{N} \lto \coprod_{i = 1}^n\bb{N}$, there is a corresponding induced isometric isomorphism $\hat{\alpha}: \bb{H} \lto \bb{H}^n$ which has an explicit formula, where $z = \sum_{i = 0}^\infty z_i\underline{e}_i$:
	\begin{equation}
		\hat{\alpha}(z) = \sum_{i = 0}^\infty \Big(z_{\alpha_1(i)}\underline{e}^i,...,z_{\alpha_n(i)}\underline{e}^i\Big)
	\end{equation}
	\begin{example}
		A simple example is given by the following:
		\begin{align*}
			\alpha_1: \bb{N} &\lto \bb{N} & \alpha_2: \bb{N} &\lto \bb{N}\\
			n &\longmapsto 2n & n &\longmapsto 2n + 1
		\end{align*}
		which induces $\alpha: \bb{N} \coprod \bb{N} \lto \bb{N}$, defined by $\alpha(n,1) = 2n$ and $\alpha(n,2) = 2n+1$. The functions $\alpha_1,\alpha_2,\alpha$ make the following a coproduct diagram:
		\begin{equation}
			\begin{tikzcd}
				\bb{N}\arrow[dr,"{\alpha_1}"]\arrow[d,rightarrowtail]\\
				\bb{N}\arrow[r,"{\alpha}"] \coprod \bb{N}\arrow[r,"{\alpha}"] & \bb{N}\\
				\bb{N}\arrow[ur,swap,"{\alpha_2}"]\arrow[u,rightarrowtail]
			\end{tikzcd}
		\end{equation}
		and indeed $\alpha$ is a bijection. We thus have two functions:
		\begin{align*}
			p: \ell^2 & \lto \ell^2 & q: \ell^2 &\lto \ell^2\\
			(z_1,z_2,...) &\longmapsto (0, z_1, 0, z_2, ...) & (z_1,z_2,...) &\longmapsto (z_1, 0, z_2, 0, ...)
		\end{align*}
		which have the following adjoints:
		\begin{align*}
			p^\ast: \ell^2 &\longmapsto \ell^2 & q^\ast: \ell^2 &\lto \ell^2\\
			(z_1,z_2,...) &\longmapsto (z_2,z_4,...) & (z_1,z_2,...) &\longmapsto (z_1,z_3,...)
		\end{align*}
		\begin{aside}
			The following calculation shows that $p^\ast$ is adjoint to $p$, the corresponding calculation for $q$ is similar:
			\begin{align*}
				\big\langle p(z_1,z_2,...),(w_1,w_2,...)\big\rangle &= \big\langle (0,z_1,0,z_2,...),(w_1,w_2,...)\big\rangle\\
				&= \big\langle (z_1,z_2,...),(w_2,w_4,...)\big\rangle\\
				&= \big\langle (z_1,z_2,...),p^\ast(w_1,w_2,...)\big\rangle
			\end{align*}
		\end{aside}
		The function $p^\ast,q^\ast$ induce $\hat{\alpha} = p^\ast \oplus q^\ast: \ell^2 \lto \ell^2 \oplus \ell^2$ defined by
		\begin{equation}
			\hat{\alpha}(z_1,z_2,...) = \big((z_2,z_4,...),(z_1,z_3,...)\big)
		\end{equation}
		We make a few observations:
		\begin{lemma}\label{lem:operator_properties}
			The functions $p,q,p^\ast,q^\ast$ satisfy the following:
			\begin{itemize}
				\item $p^\ast p = \operatorname{id}_{\ell^2} = q^\ast q$,
				\item $p p^\ast + q q^\ast = \operatorname{id}_{\ell^2}$,
				\item $p^\ast q = 0 = q^\ast p$.
			\end{itemize}
		\end{lemma}
	\end{example}
	\subsection{Proofs as operators}
	The general theory is easiest to understand when we start with an example:
	\begin{example}
		Let $\pi$ denote the following proof net.
		\begin{equation}\label{eq:standard_example}
			\begin{tikzcd}[column sep = tiny]
				{A}\arrow[r,bend left, dash] & {y: \negation A}\arrow[dr] & & {A}\arrow[dl]\arrow[r,bend left, dash] & {\negation A} & {A}\arrow[rr,bend left]\arrow[dr] & & {\negation A}\arrow[dl]\\
				& & {\negation A \otimes A}\arrow[rrrr, bend right, dash] & & & & {A \parr \negation A}
			\end{tikzcd}
		\end{equation}
		We now remove the cut-link to obtain a proof-\emph{structure} $\pi'$. Label the left edges of the premises of each tensor and par link by $p$ and the right ones by $q$ (indeed these are the same $p$ and $q$ as in Section \ref{sec:interal_sum_tens}), and label the edges corresponding to axiom and cut links by the identity map $\operatorname{id}$ (this is the identity on the space $\ell^2$):
		\begin{equation}
			\begin{tikzcd}[column sep = tiny]
				{A}\arrow[r,bend left, dash, "{\operatorname{id}}"] & {\negation A}\arrow[dr,swap, "p"] & & {A}\arrow[dl,"q"]\arrow[r,bend left,dash, "{\operatorname{id}}"] & {\negation A} & {A}\arrow[rr,bend left,dash, "{\operatorname{id}}"]\arrow[dr,swap,"p"] & & {\negation A}\arrow[dl,"q"]\\
				& & {\negation A \otimes A} & & & & {A \parr v: \negation A}
			\end{tikzcd}
		\end{equation}
		Now, to each pair of conclusions is the collection of paths in $\pi'$ (where we allow for paths which traverse arrows in either direction). We introduce some notation, associated to the pair $A, \negation A$ is the set of paths which we denote $\operatorname{Path}(A,\negation A)$, notice this set has one element, but $\operatorname{Path}(A \parr \negation A,A \parr \negation A)$ has two elements. Each path induces an operator $\ell^2 \lto \ell^2$ given by composing the labels on the edges in the path, where if an edge is traversed from the target to the source, we take the adjoint of the label. For example, the path
		\begin{equation}
			(A, \negation A, \negation A \otimes A, A, \negation A) \in \operatorname{Path}(A, \negation A)
		\end{equation}
		is the associated operator $\operatorname{id}q^\ast p\operatorname{id} = q^\ast p$. 
		\begin{remark}
			Notice by Lemma \ref{lem:operator_properties} that $q^\ast p$ is the zero operator, this corresponds to the fact that if we perform cut-elimination on $\pi$, the resulting proof net does \emph{not} admit a path from $x:A$ to $w:\negation A$.
		\end{remark}
		Ranging over all paths between all pairs of conclusions in $\pi'$ defines a set of operators which can be organised into an incidence matrix as follows, we let $r$ denote $pq^\ast + qp^\ast$:
		\begin{equation}
			\begin{matrix}
				& \negation A \otimes A, & A \parr \negation A & A &  \negation A \\
				\negation A \otimes z: A &0&&p&q&&\\
				A \parr v: \negation A &&r&\\
				A &p^\ast&&0& p^\ast q&&\\
				\negation A &q^\ast&& q^\ast p&0
			\end{matrix}
		\end{equation}
		Let $\llbracket \pi_1 \rrbracket$ denote this matrix. Notice that the first two columns and first two rows are labelled by the formulas involved in the cut-link of $\pi$. Thus, we define $\sigma$ to be the matrix which permutes the first two columns:
		\begin{equation}
			\sigma =
			\begin{pmatrix}
				0 & 1 & 0 & 0\\
				1 & 0 & 0 & 0\\
				0 & 0 & 1 & 0\\
				0 & 0 & 0 & 1
			\end{pmatrix}
		\end{equation}
		and consider $\llbracket \pi \rrbracket \sigma \llbracket \pi \rrbracket$, which is a matrix whose $ij^{\text{th}}$ entry corresponds to the sum of operators corresponding to the paths in $\pi'$ which traverse the cut once, where the start of the path is the conclusion in $\pi'$ with label corresponding to column $j$, and whose end point is the conclusion with label corresponding to row $i$. In our current example, this is:
		\begin{equation}
			\llbracket \pi \rrbracket \sigma \llbracket \pi \rrbracket =
			\begin{pmatrix}
				0&&&\\
				&0&r p &r q &\\
				& p^\ast r &0&\\
				& q^\ast r &&0
			\end{pmatrix}
		\end{equation}
		Multiplying by $\sigma \llbracket \pi \rrbracket$ yields:
		\begin{equation}\label{eq:cut_elimination_GoI}
			\llbracket \pi \rrbracket \sigma \llbracket \pi \rrbracket \sigma \llbracket \pi \rrbracket = 
			\begin{pmatrix}
				0&&&\\
				&0&&\\
				&& p^\ast r p & q^\ast r p\\
				&& p^\ast r q & q^\ast r q
			\end{pmatrix}
			=
			\begin{pmatrix}
				0&&&\\
				&0&&\\
				&& 0 & 1\\
				&& 1 & 0
			\end{pmatrix}
		\end{equation}
		What happens if we perform the same process to $\pi$ after we have performed cut-elimination? Under this process, $\pi$ corresponds to the proof consisting of a single axiom link:
		\begin{equation}
			\begin{tikzcd}
				A\arrow[r,bend left,dash] & \negation A
			\end{tikzcd}
		\end{equation}
		which corresponds to the matrix
		\begin{equation}
			\begin{pmatrix}
				0 & 1\\
				1 & 0
			\end{pmatrix}
		\end{equation}
		which appears as a minor in \eqref{eq:cut_elimination_GoI}. The general theory will show that this is not a coincidence.
	\end{example}
	\begin{defn}
		Let $A_1,A_2$ be conclusions in a proof structure $\pi$. The set of paths (possibly traversing edges in reverse direction) in $\pi$ from $A_1$ to $A_2$ is denoted
		\begin{equation}
			\operatorname{Path}(A_1,A_2)
		\end{equation}
	\end{defn}
	\begin{defn}
		Let $\pi$ be a proof structure, remove all edges corresponding to cut links to obtain a proof structure $\pi'$ with conclusions $A_1,...,A_n$.  Let $m < n$ be an integer and assume that $A_1,...,A_n$ are ordered so that $A_1,...,A_m$ are conclusions of $\pi'$ but \emph{not} conclusions of $\pi$ (in other words, $A_1,...,A_m$ are the formulas appearing in cut links of $\pi$).  We construct an $n \times n$ matrix $\llbracket \pi \rrbracket$ via the following procedure:
		\begin{enumerate}
			\item Let $L$ be either a tensor or par link in $\pi'$. Label the edge corresponding to the \emph{left} premise of $L$ by $p$, and label the edge corresponding to the \emph{right} premise of $L$ by $q$,
			\item Label the remaining edges of $\pi'$ by $\operatorname{id}$.
			\item For each path $\nu \in \operatorname{Path}(A_i,A_j)$ we let $o_\nu$ denote the operator $\ell^2 \lto \ell^2$ given by the composite of the operators in the path $\nu$, where we take the adjoint of an operator if the corresponding edge is traversed in reverse direction in $\nu$. Define 
			$$o_{ij} = \sum_{\nu \in \operatorname{Path}(A_j,A_i)}o_\nu$$
			\item Define the matrix
			\begin{equation}
				(\llbracket \pi \rrbracket)_{ij} = o_{ij}
			\end{equation}
		\end{enumerate}
	\end{defn}
	Recall that a proof structure $\pi$ with a single conclusion is unique up to the axiom links. Thus, if a proof structure $\pi$ admits a cut between formulas $A := A_1 \boxtimes_1 \hdots \boxtimes_{n-1} A_n$ and $B:=B_1 \diamondtimes_1 \hdots \diamondtimes_{n-1} B_n$, we can consider the unique substructures $\pi_1,\pi_2$ which are given by removing the axiom links of any proof structure with unique conclusions $A,B$ respectively.  Let $\zeta$ be the substructure consisting of $\pi_1,\pi_2$ and a cut-link connecting their unique conclusions. By uniqueness of $\pi_1,\pi_2$ and using Lemma \ref{lem:GoI_lemma}, the following Proposition is clear:
	\begin{proposition}\label{prop:main_GoI_prop}
		Let $\nu_{ij}$ be the (unique) path from $A_i$ to $B_j$ in $\zeta$. Then if $o_{\nu_{ij}}$ denotes the corresponding operator and $\delta_{ij}$ the dirac $\delta$-function, we have
		\begin{equation}
			o_{\nu_{ij}} = \delta_{ij}
		\end{equation}
	\end{proposition}
	We wish to compose the incidence matrix with itself but with columns of cut formulas interchanged, hence we introduce:
	\begin{defn}
		Let $\pi$ be a proof net and $\zeta$ the corresponding proof structure given by removing all edges corresponding to cut-links in $\pi$. Then if $A_1,...,A_n$ are the conclusions of $\zeta$ and $m < n$ is such that $A_1,...,A_m$ are conclusions of $\zeta$ but \emph{not} of $\pi$, then we let $\sigma_m$ denote the $(2m + n) \times (2m + n)$ matrix whose top left $2m \times 2m$ minor is given by
		\begin{equation}
			\begin{pmatrix}
				0 & 1\\
				1 & 0\\
				& & \ddots\\
				& & & 0 & 1\\
				& & & 1 & 0
			\end{pmatrix}
		\end{equation}
		and whose lower right $n \times n$ minor is the identity. The rest of the entries are 0.
		
		We define
		\begin{equation}
			\operatorname{Ex}(\llbracket \pi \rrbracket) = \llbracket \pi \rrbracket + \llbracket \pi \rrbracket \sigma \llbracket \pi \rrbracket + \llbracket \pi \rrbracket \sigma \llbracket \pi \rrbracket \sigma \llbracket \pi \rrbracket + \hdots
		\end{equation}
		\textcolor{red}{(It has yet to be shown that this is finite)}.
	\end{defn}
	\begin{cor}[Geometry of Interaction One]\label{cor:GoI_one}
		Let $\pi$ be a proof net and $\zeta$ the cut-free proof equivalent under cut elimination to $\pi$. Then the matrix $\llbracket \zeta\rrbracket$ exists as a minor in $Ex(\llbracket \pi \rrbracket)$.
	\end{cor}
	\begin{proof}
		Let $(A_i,A_j)$ be a pair of conclusions in $\zeta$, assume furthermore that the $ji^{\text{th}}$ entry of $\llbracket \zeta \rrbracket$ is non-zero. The pair $(A_i,A_j)$ also exists as a pair of conclusions in $\pi$. Let $\nu_{ij}$ be a path in $\pi$ connecting $A_i$ to $A_j$, such a path necessarily exists since such a path exists in $\zeta$, indeed the cut-elimination rules preserve connected and disconnectedness. If $\nu_{ij} = 0$ then there necessarily exists a path which traverses a single cut once, where the cut is between formulas $A_1 \boxtimes_1 \hdots \boxtimes_{n-1} A_n$ and $B_1 \diamondtimes_1 \hdots \diamondtimes_{n-1} B_n$ such that some $A_k$ is connected to some $B_l$ for $k \neq l$. This implies by Proposition \ref{prop:main_GoI_prop} that the $ji^{\text{th}}$ entry of $\llbracket \zeta \rrbracket$ is zero, a contradiction.
		
		Hence $\nu_{ij} \neq 0$, and indeed the remaining elements of the proof $\pi$ are preserved by cut-elimination, so in fact $\nu_{ij} = (\llbracket \zeta \rrbracket)_{ji}$.
	\end{proof}
	The following allows for an alternate description of conclusion-conclusion paths in a proof structure:
	\begin{cor}
		Let $\pi$ be a proof structure and $A_1,A_2$ a pair of distinct conclusions in $\pi$. There exists a unique path in $\pi$ from $A_1$ to $A_2$ whose corresponding operator is equal to the identity.
	\end{cor}
	\begin{proof}
		Clear in the cut-free case, Corollary \ref{cor:GoI_one} then implies the general case.
	\end{proof}
	We thus have:
	\begin{thm}
		Every conclusion to conclusion path $\nu$ in a proof net $\pi$, where the operator $o_{\nu}$ corresponding to $\nu$ is equal to the identity, induces a unique set of arcs through vertices in the ``web" corresponding to $\pi$ as given by \emph{CatGoI}. Moreover, two distinct such paths have no vertex arcs in common, and all vertex arcs are given in this way.
	\end{thm}
	
	
	
	
	
	
	
	
	
	
	
	
	
	
	
	
	
	
	
	
	
	
	
	
	
	
	
	
	
	
	
	
	
	
	
	
	
	
	
	
	
	
	
	
	
	
	
	
	\bibliographystyle{amsalpha}
	\providecommand{\bysame}{\leavevmode\hbox to3em{\hrulefill}\thinspace}
	\providecommand{\href}[2]{#2}
	\begin{thebibliography}{99}
		\bibitem{ILSC} Intuitionistic, Linear Sequent Calculus, W. Troiani.
		
		\bibitem{linearlogic} \emph{Linear Logic}, J.Y. Girard
		
		\bibitem{multiplicatives} \emph{Multiplicatives}, J.Y. Girard.
		
		\bibitem{girard} \emph{Geometry of Interaction I}, J.Y. Girard.
		
		\bibitem{ILSC} \emph{Intuitionistic, linear sequent calculus} W. Troiani.
		
		\bibitem{proof nets} \emph{proof nets}, W Troiani
		
		
	\end{thebibliography}
\end{document}
