\documentclass[12pt]{article}

\usepackage{amsthm}
\usepackage{amsmath}
\usepackage{amsfonts}
\usepackage{mathrsfs}
\usepackage{amssymb}
\usepackage{units}
\usepackage{graphicx}
\usepackage{tikz-cd}
\usepackage{nicefrac}
\usepackage{hyperref}
\usepackage{bbm}
\usepackage{color}
\usepackage{tensor}
\usepackage{tipa}
\usepackage{bussproofs}
\usepackage{ stmaryrd }
\usepackage{ textcomp }
\usepackage{leftidx}
\usepackage{afterpage}
\usepackage{varwidth}

\newcommand\blankpage{
    \null
    \thispagestyle{empty}
    \addtocounter{page}{-1}
    \newpage
    }

\graphicspath{ {images/} }

\theoremstyle{plain}
\newtheorem{thm}{Theorem}[subsection] % reset theorem numbering for each chapter
\newtheorem{proposition}[thm]{Proposition}
\newtheorem{lemma}[thm]{Lemma}
\newtheorem{fact}[thm]{Fact}
\newtheorem{cor}[thm]{Corollary}

\theoremstyle{definition}
\newtheorem{defn}[thm]{Definition} % definition numbers are dependent on theorem numbers
\newtheorem{exmp}[thm]{Example} % same for example numbers
\newtheorem{notation}[thm]{Notation}
\newtheorem{remark}[thm]{Remark}
\newtheorem{condition}[thm]{Condition}
\newtheorem{question}[thm]{Question}
\newtheorem{construction}[thm]{Construction}
\newtheorem{exercise}[thm]{Exercise}
\newtheorem{example}[thm]{Example}

\newcommand{\bb}[1]{\mathbb{#1}}
\newcommand{\scr}[1]{\mathscr{#1}}
\newcommand{\call}[1]{\mathcal{#1}}
\newcommand{\psheaf}{\text{\underline{Set}}^{\scr{C}^{\text{op}}}}
\newcommand{\und}[1]{\underline{\hspace{#1 cm}}}
\newcommand{\adj}[1]{\text{\textopencorner}{#1}\text{\textcorner}}
\newcommand{\comment}[1]{}
\newcommand{\lto}{\longrightarrow}

\usepackage[margin=1cm]{geometry}

\title{CatGoI}
\date{August 2020}

\begin{document}

\section{Bicategory Theory}
\section{Bicategory Theory}
Often one motivates the notion of a \emph{bicategory} by observing the definition of a category and then ``suping up" the hom-sets. From this angle though there are multiple reasonable generalisations, ought associativity hold strictly or up to natural isomorphism?

So instead we seek our motivation from a different source, recall:
\begin{defn}
A \textbf{monoidal category} consists of
\begin{itemize}
\item a category $\scr{C}$,
\item a special object $\mathbbm{1} \in \scr{C}$,
\item a functor $\otimes: \scr{C} \times \scr{C} \lto \scr{C}$
\end{itemize}
along with three natural isomorphisms $\lambda: \mathbbm{1} \times \operatorname{id}_{\scr{C}} \lto \operatorname{id}_{\scr{C}}, \rho: \operatorname{id}_{\scr{C}} \times \mathbbm{1} \lto \operatorname{\scr{C}}, \alpha: (\operatorname{id}_{\scr{C}} \times \operatorname{id}_{\scr{C}}) \times \operatorname{id}_{\scr{C}} \lto \operatorname{id}_{\scr{C}} \times (\operatorname{id}_{\scr{C}} \times \operatorname{id}_{\scr{C}})$,
with all of this data satisfying:
\begin{itemize}
\item the \textbf{pentagon diagram}, ie, commutativity for all $A,B,C,D \in \scr{C}$ of the following:
\begin{equation}
\begin{tikzcd}[column sep = tiny]
& A \otimes (B \otimes (C \otimes D))\arrow[r]\arrow[dl]  & A \otimes ((B \otimes C) \otimes D)\arrow[dr] \\
(A \otimes B) \otimes (C \otimes D)\arrow[drr]& & & (A \otimes (B \otimes C)) \otimes D\arrow[dl]\\
& & ((A \otimes B) \otimes C) \otimes D
\end{tikzcd}
\end{equation}
\item the \textbf{identity diagrams}, ie, commutativity for all $A,B,C \in \scr{C}$ of the following:
\begin{equation}
\begin{tikzcd}
(A \otimes \mathbbm{1}) \otimes B\arrow[r]\arrow[dr] & A \otimes (\mathbbm{1} \otimes B)\arrow[d]\\
& A \otimes B
\end{tikzcd}
\end{equation}
\end{itemize}
\end{defn}
We wish to define a \emph{bicategory} so that a monoidal category is a bicategory with a single object. Thus, we see that we take associativity and identity up to natural isomorphism, hence:
\begin{defn}
A \textbf{bicategory} $\scr{C}$ consists of
\begin{itemize}
\item a collection $\operatorname{Obj}\scr{C}$ of \textbf{objects},
\item for every pair $X,Y$ of objects, a category
\begin{equation}
\operatorname{Hom}_{\scr{C}}(X,Y)
\end{equation}
whose objects are \textbf{$1$-morphisms} with \textbf{domain} $X$ and \textbf{codomain} $Y$, and whose morphisms are \textbf{$2$-morphisms},
\item for every triple $X,Y,Z$ of objects, a functor called \textbf{horizontal composition}
\begin{align*}
\circ_{X,Y,Z}: \operatorname{Hom}_{\scr{C}}(Y,Z) \times \operatorname{Hom}_{\scr{C}}(X,Y) &\lto \operatorname{Hom}_{\scr{C}}(X,Z)\\
(f,g) &\longmapsto f \circ g\\
(\alpha: f \Rightarrow g, \beta: h \Rightarrow j) &\longmapsto \beta \ast \alpha
\end{align*}
\item for each tuple of objects $X,Y,Z,W$ a natural isomorphism $\alpha_{X,Y,Z,W}$ from the functor defined by the composite:
\begin{equation}
\operatorname{Hom}_{\scr{C}}(Z,W) \times \operatorname{Hom}_{\scr{C}}(Y,Z) \times \operatorname{Hom}_{\scr{C}}(X,Y) \lto \operatorname{Hom}_{\scr{C}}(Z,W) \times \operatorname{Hom}_{\scr{C}}(X,Z) \lto \operatorname{Hom}_{\scr{C}}(X,W)
\end{equation}
and the functor defined by the composite
\begin{equation}
\operatorname{Hom}_{\scr{C}}(Z,W) \times \operatorname{Hom}_{\scr{C}}(Y,Z) \times \operatorname{Hom}_{\scr{C}}(X,Y) \lto \operatorname{Hom}_{\scr{C}}(Y,W) \times \operatorname{Hom}_{\scr{C}}(X,Y) \lto \operatorname{Hom}_{\scr{C}}(X,W)
\end{equation}
\item for every object $X \in \scr{C}$ a functor $\mathbbm{1}_X: \bold{1} \lto \operatorname{Hom}(X,X)$, where $\bold{1}$ is the category with a single object and single morphism,
\item for every pair of objects $X,Y$ a pair of natural isomorphisms, $\rho$ which maps from the functor defined by the composite:
\begin{equation}
\bold{1} \times \operatorname{Hom}(X,Y) \lto \operatorname{Hom}(Y,Y) \times \operatorname{Hom}(X,Y) \lto \operatorname{Hom}(X,Y)
\end{equation}
to the functor defined by the composite
\begin{equation}
\bold{1} \times \operatorname{Hom}(X,Y) \stackrel{\sim}{\lto} \operatorname{Hom}(X,Y)
\end{equation}
and $\lambda$, which is defined similarly,
\end{itemize}
satisfying:
\begin{itemize}
\item for all objects $X,Y,Z,W$ the following diagram commutes
\begin{equation}
\begin{tikzcd}[column sep = tiny]
& X \circ (Y \circ (Z \circ W))\arrow[r]\arrow[dl]  & X \circ ((Y \circ Z) \circ W)\arrow[dr] \\
(X \circ Y) \circ (Z \circ W)\arrow[drr]& & & (X \circ (Y \circ Z)) \circ W\arrow[dl]\\
& & ((X \circ Y) \circ Z) \circ W
\end{tikzcd}
\end{equation}
\item for all objects $X,Y$ the following diagram commutes:
\begin{equation}
\begin{tikzcd}
(X \circ \mathbbm{1}) \circ Y\arrow[r]\arrow[dr] & X \circ (\mathbbm{1} \circ Y)\arrow[d]\\
& X \circ Y
\end{tikzcd}
\end{equation}
\end{itemize}

\end{defn}
\begin{remark}
The point is the following:
\begin{enumerate}
\item hom sets have become categories,
\item composition has become a functor
\item identity morphisms are not an element of a set now,  they are an object of a category,
\item composition now has the freedom to hold only up to isomorphism,
\item associativity only holds up to isomorphism,
\item that units act like units only holds up to isomorphism,
\item these are 3-ary and 2-ary isomorphisms so we need compatibility diagrams.
\end{enumerate}
\end{remark}
The example we are chiefly concerned with is the \emph{bicategory of Landau-Ginzburg models}, we now head towards this definition.
\begin{defn}
A \textbf{potential} is 
\end{defn}

\section{The semantics}




\section{Relevant calculations}

\subsection{Calculating $\bigcap_{i = 1}^m \operatorname{Ker}(\bar{\nu}_i + \nu_i)$}
``A typical element of $\wedge k \underline{\psi} \otimes \wedge k \underline{\theta}$ annihilated by $\bar{\nu}_i + \nu_i$ is an \emph{entangled bit}":
\begin{align*}
(\overline{\nu}_i + \nu_i)(1 + \psi_i\theta_{\sigma^{-1}i}) &= \bar{\nu}_i + \nu_i + \bar{\nu}_i\psi\theta_{\sigma^{-1}i} + \nu_i\psi_i\theta_{\sigma^{-1}i}\\
&= \theta_{\sigma^{-1}i}^\ast + \theta_{\sigma^{-1}i} + \psi_i - \psi^\ast_i + \theta_{\sigma^{-1}i}^\ast \psi\theta_{\sigma^{-1}i} + \theta_{\sigma^{-1}i}\psi_i\theta_{\sigma^{-1}i} + \psi_i\psi_i\theta_{\sigma^{-1}i} - \psi^\ast_i \psi_i\theta_{\sigma^{-1}i}\\
&= \theta_{\sigma^{-1}i}^\ast + \theta_{\sigma^{-1}i} + \psi_i- \psi_i^\ast  + \psi_i + 0 - \theta_{\sigma^{-1}i} - 0\\
&= \theta_{\sigma^{-1}i}^\ast - \psi_i^\ast
\end{align*}



\end{document}



























\maketitle
\tableofcontents