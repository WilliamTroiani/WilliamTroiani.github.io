\documentclass{birkjour}

\usepackage{amsthm}
\usepackage{amsmath}
\usepackage{amsfonts}
\usepackage{mathrsfs}
\usepackage{amssymb}
\usepackage{units}
\usepackage{graphicx}
\usepackage{tikz-cd}
\usepackage{nicefrac}
\usepackage{hyperref}
\usepackage{bbm}
\usepackage{color}
\usepackage{tensor}
\usepackage{tipa}
\usepackage{bussproofs}
\usepackage{ stmaryrd }
\usepackage{ textcomp }
\usepackage{leftidx}
\usepackage{afterpage}
\usepackage{appendix}
\usepackage{ tipa }
\usepackage{quiver}
\usepackage{adjustbox}

\newcommand\blankpage{
	\null
	\thispagestyle{empty}
	\addtocounter{page}{-1}
	\newpage
}

\graphicspath{ {images/} }

\theoremstyle{plain}
\newtheorem{thm}{Theorem}[subsection] % reset theorem numbering for each chapter
\newtheorem{proposition}[thm]{Proposition}
\newtheorem{lemma}[thm]{Lemma}
\newtheorem{fact}[thm]{Fact}
\newtheorem{cor}[thm]{Corollary}

\theoremstyle{definition}
\newtheorem{defn}[thm]{Definition} % definition numbers are dependent on theorem numbers
\newtheorem{exmp}[thm]{Example} % same for example numbers
\newtheorem{notation}[thm]{Notation}
\newtheorem{remark}[thm]{Remark}
\newtheorem{condition}[thm]{Condition}
\newtheorem{question}[thm]{Question}
\newtheorem{construction}[thm]{Construction}
\newtheorem{exercise}[thm]{Exercise}
\newtheorem{example}[thm]{Example}
\newtheorem{observation}[thm]{Observation}

\providecommand{\keywords}[1]{\textbf{\textit{Key words---}} #1}

\newcommand{\bb}[1]{\mathbb{#1}}
\newcommand{\scr}[1]{\mathscr{#1}}
\newcommand{\call}[1]{\mathcal{#1}}
\newcommand{\psheaf}{\text{\underline{Set}}^{\scr{C}^{\text{op}}}}
\newcommand{\und}[1]{\underline{\hspace{#1 cm}}}
\newcommand{\adj}[1]{\text{\textopencorner}{#1}\text{\textcorner}}
\def\doubleunderline#1{\underline{\underline{#1}}}
\newcommand{\comment}[1]{}
\newcommand{\lto}{\longrightarrow}
\newcommand{\rone}{(\operatorname{R}\bold{1})}
\newcommand{\lone}{(\operatorname{L}\bold{1})}
\newcommand{\rimp}{(\operatorname{R} \multimap)}
\newcommand{\limp}{(\operatorname{L} \multimap)}
\newcommand{\rtensor}{(\operatorname{R}\otimes)}
\newcommand{\ltensor}{(\operatorname{L}\otimes)}
\newcommand{\rtrue}{(\operatorname{R}\top)}
\newcommand{\rwith}{(\operatorname{R}\&)}
\newcommand{\lwithleft}{(\operatorname{L}\&)_{\operatorname{left}}}
\newcommand{\lwithright}{(\operatorname{L}\&)_{\operatorname{right}}}
\newcommand{\rplusleft}{(\operatorname{R}\oplus)_{\operatorname{left}}}
\newcommand{\rplusright}{(\operatorname{R}\oplus)_{\operatorname{right}}}
\newcommand{\lplus}{(\operatorname{L}\oplus)}
\newcommand{\prom}{(\operatorname{prom})}
\newcommand{\ctr}{(\operatorname{ctr})}
\newcommand{\der}{(\operatorname{der})}
\newcommand{\weak}{(\operatorname{weak})}
\newcommand{\exi}{(\operatorname{exists})}
\newcommand{\fa}{(\operatorname{for\text{ }all})}
\newcommand{\ex}{(\operatorname{ex})}
\newcommand{\cut}{(\operatorname{cut})}
\newcommand{\ax}{(\operatorname{ax})}
\newcommand{\negation}{\sim}
\newcommand{\true}{\top}
\newcommand{\false}{\bot}
\DeclareRobustCommand{\diamondtimes}{%
	\mathbin{\text{\rotatebox[origin=c]{45}{$\boxplus$}}}%
}
\newcommand{\tagarray}{\mbox{}\refstepcounter{equation}$(\theequation)$}
\newcommand{\startproof}[1]{
	\AxiomC{#1}
	\noLine
	\UnaryInfC{$\vdots$}
}
\newenvironment{scprooftree}[1]%
{\gdef\scalefactor{#1}\begin{center}\proofSkipAmount \leavevmode}%
	{\scalebox{\scalefactor}{\DisplayProof}\proofSkipAmount \end{center} }

%\usepackage[margin=1cm]{geometry}

%\bibliographystyle{dinat}

\begin{document}

\section{Introduction}
We provide an application of the internal logic of a topos, which we now explain. Proofs of statements about the topos $\underline{\operatorname{Sets}}$ of sets are sometimes easier to construct than proofs of statements in arbitrary topoi because of the structure of $\underline{\operatorname{Sets}}$. In particular, the fact that sets contain \emph{elements} can be a very helpful observation. The internal logic of a topos offers this structure to arbitrary topoi as the interal logic admits to each type $X$ a (countably) infinite set of \emph{variables} $x,y,...$, see Definition \ref{def:type_theory} for a precise definition. Section \ref{sec:surjective_families} shows a concrete application of this structure. The general idea is the following: given a surjective family of functions of sets $\lbrace t_i: A_i \lto A\rbrace_{i = 0}^\infty$ and a family of functions of sets $\lbrace g_i: A_i \lto U\rbrace_{i = 0}^\infty$ we can define a function $f: A \lto U$ given by ``choosing a lift" $a_i \in A_i$ of $a \in A$ along some $t_i$ and defining $f(a) := g_i(a_i)$, provided that the functions $t_i$ indeed can be suitably glued together. There seems to be no easy way to describe the map $f:A \lto U$ without using the internal logic. This is the content of Section \ref{sec:surjective_families} which indeed is self contained and does not require the content of Section \ref{sec:arbitrary_topos}.

\section{Surjective families}\label{sec:surjective_families}\label{sec:surjective_family}
As another application of the Internal logic of a topos, we show how one can construct lifts along surjective families using the familiar idea from the topos $\underline{\operatorname{Sets}}$ of sets and functions.

Given a surjective family of functions of sets $\lbrace t_i: A_i \to A\rbrace_{i = 0}^\infty$ and a family of functions $\lbrace g_i: A_i \to U\rbrace_{i = 0}^\infty$, if
\begin{equation}
	\label{welldefinedtwo}
	\forall a_i \in A_i, \forall a_j \in A_j, t_i(a_i) = t_j(a_j) \Rightarrow g_i(a_i) = g_j(a_j)
\end{equation}
then there exists a well defined function $f: A \to U$ which is given by ``choosing a lift" $a_i \in A_i$ of $a \in A$ along some $t_i$ and then defining $f(a) := g_i(a_i)$. In choosing a lift $a_i \in A_i$ of $a \in A$ there are two pieces of information, one is that $t_i(a_i) = a$ and the other is $g_i(a_i)$, which can be captured by the following subset of $A \times U$:
\[\big\lbrace\vec{z} \in A \times U \mid \exists i \in \bb{N}, \exists a_i \in A_i, \vec{z} = \big(t_i(a_i),g_i(a_i)\big)\big\rbrace \subseteq A \times U\]
which we denote by $X$. Thus there is the following diagram
\[
\begin{tikzcd}
	& X\arrow[d,rightarrowtail]\\
	& A \times U\arrow[dl,swap,"{\pi_l}"]\arrow[dr,"{\pi_R}"]\\
	A & & U
\end{tikzcd}
\]
It then follows from Equation \ref{welldefinedtwo} that there exists a bijection $\hat{f}: X \stackrel{\sim}{\longrightarrow} A$. The function $\pi_R \hat{f}^{-1}$ is then equal to $f$. The following Lemma generalises this description to an arbitrary elementary topos $\call{E}$.
\begin{lemma}
	\label{functionglueingintro}
	Let $\lbrace t_i(a_i): A \rbrace_{i = 0}^\infty$ be a finite set of terms satisfying the following sequent:
	\begin{equation}
		\label{epimorphicfamily}
		\vdash_{a:A}\bigvee_{i = 0}^\infty \exists a_i: A_i, t_i(a_i) = a
	\end{equation}
	Let $\lbrace g_i: A_i \to U\rbrace_{i = 0}^\infty$ be a set of morphisms in $\call{E}$, and assume the following sequent holds for each $i,j$:
	\begin{equation}
		\label{welldefined}
		\vdash \forall a_i : A_i, \forall a_j: A_j, t_i(a_i) = t_j(a_j) \Rightarrow g_i(a_i) = g_j(a_j)
	\end{equation}
	Then there exists a (necessarily unique) morphism $f: A \to  U$ such that for each $i$, the following diagram commutes
	\[
	\begin{tikzcd}
		A_i\arrow[dr,"{g_i}"]\arrow[d,swap,"{t_i}"]\\
		A\arrow[r,swap,"f"] & U
	\end{tikzcd}
	\]
\end{lemma}
The significance of Lemma \ref{functionglueingintro} is emphasised by the fact that in the topos \underline{Sets} both the coproduct of a collection of functions and the coequaliser of a pair of functions are examples of morphisms satisfying the hypotheses of Lemma \ref{functionglueingintro} (and thus, so are all colimits) as Example \ref{ex:coproduct} and \ref{ex:coequaliser} demonstrate.
\begin{example}
	\label{ex:coproduct}
	Say $\lbrace g_i: A_i \to U\rbrace_{i = 0}^\infty$ is a family of functions, and $\iota_i: A_i \rightarrowtail \coprod_{i = 0}^\infty A_i$ is the $i^\text{th}$ inclusion map. Then there exists a unique morphism $f: \coprod_{i = 0}^\infty A_i \to U$ such that $f\iota_i = g_i$ for each $i$, and this function $f$ is given by:
	\begin{align*}
		f: \coprod_{i = 0}^\infty A_i &\to U\\
		a_i &\mapsto
		g_i(a_i)\text{, where }\iota_i(a_i) = a_i
	\end{align*}
\end{example}
\begin{example}\label{ex:coequaliser}
	Say $g_0,g_1: A'' \to A'$ and $e: A' \twoheadrightarrow  \operatorname{Coeq}(g_0,g_1)$ is the coequaliser of $g_0$ and $g_1$. Then given a morphism $g:A' \to U$ such that $g g_0 = g g_1$ there exists a unique function $f: \operatorname{Coeq}(g_0,g_1) \to U$ such that $fe = g$, and this function $f$ is given by:
	\begin{align*}
		f: \operatorname{Coeq}(g_0,g_1) &\to U\\
		[a] &\mapsto g(a)\text{, where }e(a) = [a]
	\end{align*}
\end{example}
\begin{remark}
	One might notice that Example \ref{ex:coproduct} allows for a (countably) infinite disjoint union (and indeed, an even more general situation can be considered), so why does this paper only concern itself with \emph{finite} colimits? The reason is simply because the internal logic only allows for the construction of product terms which are \emph{finite} in size. One could allow for a more general language, however doing so is non-standard and so we omit this level of generality within this paper.
\end{remark}
\begin{lemma}
	\label{functionglueing}
	Let $\lbrace t_i(a_i): A \rbrace_{i = 0}^\infty$ be a finite set of terms satisfying the following sequent:
	\begin{equation}
		\label{epimorphicfamily}
		\vdash_{a:A}\bigvee_{i = 0}^\infty \exists a_i: A_i, t_i(a_i) = a
	\end{equation}
	Let $\lbrace g_i: A_i \to U\rbrace_{i = 0}^\infty$ be a set of morphisms in $\call{E}$, and assume the following sequent holds for each $i,j$:
	\begin{equation}
		\label{welldefined}
		\vdash \forall a_i : A_i, \forall a_j: A_j, t_i(a_i) = t_j(a_j) \Rightarrow g_i(a_i) = g_j(a_j)
	\end{equation}
	Then there exists a (necessarily unique) morphism $f: A \to  U$ such that for each $i$, the following diagram commutes
	\[
	\begin{tikzcd}
		A_i\arrow[dr,"{g_i}"]\arrow[d,swap,"{t_i}"]\\
		A\arrow[r,swap,"f"] & U
	\end{tikzcd}
	\]
\end{lemma}
\begin{proof}
	First, define the following subobject
	\[\llbracket (z: A \times U) . \bigvee_{i = 0}^n\big(\exists a_i : A_i, z = \langle t_i(a_i), g_ia_i \rangle \big)\rrbracket \stackrel{c}{\rightarrowtail} A \times U\]
	To ease notation let $\phi_i(z,a_i)$ be the formula $z = \langle t_i(a_i),g_ia_i\rangle$ and $\phi(z)$ the formula $\bigvee_{i = 0}^\infty \exists a_i:A_i, \phi_i(z,a_i)$.
	Then consider the composite
	\[\llbracket (z: A \times U) .\phi(z)\rrbracket \stackrel{c}{\rightarrowtail} A \times U \stackrel{\pi_A}{\to} A\]
	It now suffices to show that this is an isomorphism, as then the morphism $f$ can be taken to be $\pi_U c (\pi_A c)^{-1}$. Since every elementary topos is \emph{balanced}, that is, any morphism in a topos which is both epic and monic is an isomorphism (see \cite[\S IV.1 Prop 2]{MM}), it suffices to show this of $\pi_A c$. By Lemma \cite[3.4.2]{troiani} it suffices to show
	\begin{equation}
		\label{moniccondition}
		\phi(z_1) \wedge \phi(z_2) \vdash_{z_1,z_2 : A \times U} \operatorname{fst}(z_1) = \operatorname{fst}(z_2) \Rightarrow z_1 = z_2
	\end{equation}
	and
	\begin{equation}
		\label{epiccondition}
		\vdash_{a:A} \exists z: A \times U, \phi(z) \wedge \operatorname{fst}(z) = a
	\end{equation}
	Recall that the following Sequents hold in any elementary topos:
	\begin{enumerate}
		\item \label{substitutionlemma} \[(\vec{x} = \vec{s}) \wedge \psi \vdash_{\vec{y}} \psi[\vec{x} := \vec{s}]\]
		where $\vec{x}$ is a string of variables, $\vec{s}$ is a string of terms with the same length and type as $\vec{x}$, and no free variable in $\psi$ becomes bound in $\psi[\vec{x} := \vec{s}]$ (this follows from the \emph{substitution} and \emph{equality} axioms given in \cite[\S4.1 Definition 1.3.1]{Johnstone}).
		\item \label{producttypes} \[\vdash_{w:W_1 \times W_2}\langle \operatorname{fst}(w), \operatorname{scd}(w)\rangle = w\]
		(See \cite[\S D4.1 Lemma 4.1.6]{Johnstone})
		\item \label{application}
		\[z_1 = z_2 \vdash_{z_1 : Z, z_2 : Z} \operatorname{scd}(z_1) = \operatorname{scd}(z_2)\]
		\item \label{witness} 
		\[\psi[x := t] \vdash_{\vec{y}} \exists x: X, \psi\]
	\end{enumerate}
	To show that Sequent \ref{moniccondition} holds, it suffices to show that for each $i,j$:
	\begin{equation}
		\label{firstreduction}
		\exists a_i: A_i, \exists a_j: A_j, \phi_i(z_1,a_i) \wedge \phi_j(z_2, a_j) \wedge \operatorname{fst}(z_1) = \operatorname{fst}(z_2) \vdash_{z_1,z_2: A \times U} z_1 = z_2
	\end{equation}
	by \ref{producttypes} above, it suffices to show:
	\begin{align*}
		\exists a_i: A_i, \exists a_j: A_j, &\phi_i(z_1,a_i) \wedge \phi_j(z_2, a_j) \wedge \operatorname{fst}(z_1) = \operatorname{fst}(z_2) \\
		&\vdash_{z_1,z_2: A \times U} \operatorname{scd}(z_1) = \operatorname{scd}(z_2)
	\end{align*}
	
	
	
	ie,
	\[z_1 = \langle t_i(a_i),g_ia_i\rangle \wedge z_2 = \langle t_j(a_j),g_ja_j\rangle \wedge \operatorname{fst}(z_1) = \operatorname{fst}(z_2) \vdash_\Gamma \operatorname{scd}(z_1) = \operatorname{scd}(z_2)\]
	where $\Gamma = (a_i: A_i, a_j: A_j, z_1 : A \times U, z_2: A \times U)$. By \ref{substitutionlemma} above, it suffices to show
	\[\operatorname{fst}(\langle t_i(a_i),g_ia_i\rangle) = \operatorname{fst}(\langle t_j(a_j), g_ja_j\rangle) \vdash_\Gamma \operatorname{scd}(z_1) = \operatorname{scd}(z_2)\]
	that is,
	\[t_i(a_i) = t_j(a_j) \vdash_\Gamma \operatorname{scd}(z_1) = \operatorname{scd}(z_2)\]
	Since $=$ is an equivalence relation \cite[\S 4.1 Definition 1.3.1b]{Johnstone} and using \ref{application} above, we have
	\[z_1 = \langle t_i(a_i),g_ia_i\rangle \wedge z_2 = \langle t_j(a_j),g_ja_j\rangle \vdash_\Gamma \operatorname{scd}(z_1) = g_ia_i \wedge \operatorname{scd}(z_2) = g_ja_j\]
	thus it suffices to show:
	\[t_i(a_i) = t_j(a_j) \vdash_\Gamma g_ia_i = g_ja_j\]
	which is exactly Sequent \ref{welldefined}.\\\\
	%
	To show that Sequent \ref{epiccondition} holds, by \ref{witness} above, it suffices to show
	\[\vdash_{a:A}\bigvee_{i = 0}^\infty \exists a_i: A_i, \langle t_i(a_i),g_ia_i\rangle = \langle t_i(a_i), g_ia_i\rangle \wedge t_ia_i = a\]
	which follows from Sequent \ref{epimorphicfamily}.
\end{proof}

	\begin{thebibliography}{99}
	\bibitem{May} J. May, \emph{A Crash Course in Algebraic Topology}, University of Chicago Press, Chicago, 1999.
	\bibitem{borceux} F. Borceux, \emph{Handbook of Categorical Algebra 1, Basic Category Theory}, University Press, Cambridge, 1994.
	\bibitem{Godel} K. Godel, \emph{Uber formal unentscheidbare Sätze der Principia Mathematica und verwandter Systeme I}, Monatshefte für Mathematik und Physik, 38: 173–198.
	\bibitem{Turing} A. Turing, \emph{On Computable Numbers, with an Application to the Entscheidungsproblem}, Proceedings of the London Mathematical Society, 1936.
	\bibitem{Church} A. Church, \emph{A Note on the Entscheidungsproblem}, New York: The Association for Symbolic Logic, Inc., 1936
	\bibitem{Lawvere} W. Lawvere, \emph{An Elementary Theory of the Category of Sets}, Proceedings of
	the National Academy of Science of the U.S.A 52, 1506–1511
	\bibitem{russell} B. Russell, A. Whitehead, \emph{The Principles of Mathematics}, Cambridge, Cambridge University Press, 1903,
	\bibitem{su} M Sorensen, P. Urzyczyn, \emph{Lectures on the Curry-Howard Isomorphism}
	\bibitem{MM} I. Meordijk, S. Mac Lane, \emph{Sheaves in Geometry and Logic}, Springer-Verlag, New York, 1992.
	\bibitem{lambekscott} J. Lambek, P.J. Scott, \emph{Introduction to Higher Order Categorical Logic}, Cambridge University Press, New York, 1986.
	\bibitem{Hatcher} A. Hatcher, \emph{Algebraic Topology}, Cambridge University Press, New York, 2002.
	\bibitem{Johnstone} P. Johnstone, \emph{Sketches of an Elephant; A Topos Theory Compendium}, Clarendon Press, Oxford, 2002
	\bibitem{Munkres} J. Munkres, \emph{Elements of Algebraic Topology}, Avalon Publishing, 1996.
	\bibitem{MacLane} S. MacLane, \emph{Homology}, Springer-Verlag, 1969
	\bibitem{troiani} W. Troiani, \emph{Simplicial sets are algorithms}, (Masters Thesis) \url{http://therisingsea.org/notes/MScThesisWillTroiani.pdf}
	\bibitem{Mikkelson} C. J. Mikkelson \emph{Lattice Theoretic and Logical Aspects of Elementary Topoi}, Aarhus Universitet, Matematisk Institut (January 1, 1976).
	\bibitem{Pare} R. Paré, \emph{Colimits in Topoi}, Bulletin of the American Mathematical Society, 1974.
	\bibitem{Milnor} J. Milnor, \emph{The Geometric Realization of a Semi-Simplicial Complex} The Annals of Mathematics, 2nd Ser., Vol. 65, No. 2. (Mar., 1957), pp. 357-362.
	\bibitem{TroianiInternal} W. Troiani, \emph{The internal logic and finite colimits}
	\end{thebibliography}

\end{document}