\documentclass[12pt]{article}

\usepackage{amsthm}
\usepackage{amsmath}
\usepackage{amsfonts}
\usepackage{mathrsfs}
\usepackage{amssymb}
\usepackage{units}
\usepackage{graphicx}
\usepackage{tikz-cd}
\usepackage{nicefrac}
\usepackage{hyperref}
\usepackage{bbm}
\usepackage{color}
\usepackage{tensor}
\usepackage{tipa}
\usepackage{bussproofs}
\usepackage{ stmaryrd }
\usepackage{ textcomp }
\usepackage{leftidx}
\usepackage{afterpage}
\usepackage{varwidth}

\newcommand\blankpage{
    \null
    \thispagestyle{empty}
    \addtocounter{page}{-1}
    \newpage
    }

\graphicspath{ {images/} }

\theoremstyle{plain}
\newtheorem{thm}{Theorem}[subsection] % reset theorem numbering for each chapter
\newtheorem{proposition}[thm]{Proposition}
\newtheorem{lemma}[thm]{Lemma}
\newtheorem{fact}[thm]{Fact}
\newtheorem{cor}[thm]{Corollary}

\theoremstyle{definition}
\newtheorem{defn}[thm]{Definition} % definition numbers are dependent on theorem numbers
\newtheorem{exmp}[thm]{Example} % same for example numbers
\newtheorem{notation}[thm]{Notation}
\newtheorem{remark}[thm]{Remark}
\newtheorem{condition}[thm]{Condition}
\newtheorem{question}[thm]{Question}
\newtheorem{construction}[thm]{Construction}
\newtheorem{exercise}[thm]{Exercise}
\newtheorem{example}[thm]{Example}
\newtheorem{observation}[thm]{Observation}

\newcommand{\bb}[1]{\mathbb{#1}}
\newcommand{\scr}[1]{\mathscr{#1}}
\newcommand{\call}[1]{\mathcal{#1}}
\newcommand{\psheaf}{\text{\underline{Set}}^{\scr{C}^{\text{op}}}}
\newcommand{\und}[1]{\underline{\hspace{#1 cm}}}
\newcommand{\adj}[1]{\text{\textopencorner}{#1}\text{\textcorner}}
\newcommand{\comment}[1]{}
\newcommand{\lto}{\longrightarrow}

\usepackage[margin=1cm]{geometry}

\title{Categorical Geometry of Interaction}
\author{Will Troiani}
\date{September 2021}

\begin{document}
\maketitle

Throughout we work over the complex numbers $\bb{C}$.
\begin{notation}
Fix integers $n,m,l > 0$ and denote the following polynomials (with all coefficients equal to the complex number $1$):
\begin{equation}
W := \sum_{i = 1}^n x_i^2, \quad V := \sum_{i = 1}^m y_i^2, \quad U := \sum_{i =1}^lz_i^2
\end{equation}
\end{notation}
Recall that any nondegenerate, symmetric, bilinear form over a finite dimensional complex vector space is admits a matrix representation
\begin{equation}
\begin{pmatrix}
I_q & 0\\
0 & -I_r
\end{pmatrix}
\end{equation}
for some positive integers $q,r$. See \cite{commalg} for more details. Thus the setting we are working in is quite general.
\begin{notation}
For $X,Y \in \lbrace W,U,V\rbrace$ we denote by $C_{XY}$ the Clifford algebra associated to the polynomial $X - Y$.
\end{notation}
\begin{example}
Explicitly, to construct $C_{UV}$ we consider the complex vector space $\bb{C}^{m + l}$ equipped with the following symmetric bilinear form $B$ (written with respect to the standard basis $e_1,...,e_{m+l}$ of $\bb{C}^{m + l}$):
\begin{equation}
\begin{pmatrix}
I_l & 0\\
0 & -I_m
\end{pmatrix}
\end{equation}
Then $C_{UV} := C_B$.

This Clifford algebra is generated by the elements $e_1,...,e_{m+l}$. It is helpful to ascribe new names to these elements, namely $\mu_1,...,\mu_l$ for $e_1,...,e_l$ and $\nu_1,...,\nu_m$ for $e_{i+1},...,e_m$. These elements satisfy the following properties (note, all commutators are graded)
\begin{equation}\label{eq:clifford_relations}
[\mu_i,\mu_j] = 2\delta_{ij}, \qquad [\mu_i,\nu_j] = 0,\qquad [\nu_i,\nu_j] = -2\delta_{ij}
\end{equation}
So indeed, one may think of $C_{UV}$ as the free $\bb{C}$-algebra generated by $\lbrace \mu_1,...,\mu_l,\nu_1,...,\nu_m\rbrace$ subject to \eqref{eq:clifford_relations}.
\end{example}
\begin{notation}
We denote by $\bar{\nu}_1,...,\bar{\nu}_m, \omega_1,...,\omega_n$ the multiplicative generators of the Clifford algebra $C_{VW}$ subject to the relations:
\begin{equation}
[\bar{\nu}_i, \bar{\nu}_j] = 2\delta_{ij},\qquad [\bar{\nu}_i, \omega_j] = 0, \qquad [\omega_i,\omega_j] = -2\delta_{ij}
\end{equation}
\end{notation}
We now construct a matrix factorisation, for an introduction to matrix factorisations see \cite{commalg}.

Let $\tilde{X}$ be a $\bb{Z}_2$-graded $C_{VW}$-module which is also a $k$-module, where $k$ is some commutative $\bb{Q}$-algebra. Denote by $X$ the $\bb{C}[x_1,...,x_n,y_1,...,y_m]$-algebra $k[x_1,...,x_n,y_1,...,y_m]\otimes \tilde{X}$ and let $\partial$ be given by
\begin{equation}
\partial = \sum_{i = 1}^n x_i \omega_i + \sum_{j = 1}^m y_j \bar{\nu}_j
\end{equation} Notice:
\begin{align*}
\partial^2 &= \Big(\sum_{i = 1}^n x_i \omega_i + \sum_{j = 1}^m y_j \bar{\nu}_j\Big)^2\\
&= \sum_{i,i' = 1}^n x_i\omega_ix_{i'}\omega_{i'} + \sum_{i = 1}^n\sum_{j = 1}^m x_i\omega_i y_j\bar{\nu}_j + \sum_{j = 1}^m\sum_{i = 1}^n y_j\bar{\nu}_jx_i\omega_i + \sum_{j,j'=1}^m y_j\bar{\nu}_j y_{j'}\bar{\nu}_{j'}\\
&= \sum_{i < i'}x_ix_{i'}(\omega_i\omega_{i'} + \omega_{i'}\omega_i) + \sum_{i = 1}^n x_i^2 \omega_i^2 + \sum_{i = 1}^n\sum_{j = 1}^m x_iy_j(\omega_i\bar{\nu}_j + \bar{\nu}_j \omega_i) + \sum_{j < j'}y_jy_{j'}(\bar{\nu}_j\bar{\nu}_{j'} + \bar{\nu}_{j'}\bar{\nu}_j)\\
&= \sum_{i < i'}x_ix_{i'}[\omega_i, \omega_{i'}] + \sum_{i = 1}^n x_i^2 \omega_i^2 + \sum_{i=1}^n\sum_{j=1}^nx_iy_j[\omega_i, \bar{\nu}_j] + \sum_{j < j'}y_jy_{j'}[\bar{\nu}_j,\bar{\nu}_{j'}] + \sum_{j = 1}^n y_j^2\bar{\nu}_{j}^2\\
&= 0 - \sum_{i=1}^n x_i^2 + 0 + 0 + \sum_{j = 1}^ny_j^2\\
&= (V - W)\operatorname{id}
\end{align*}
so indeed we have a matrix factorisation.










































\begin{thebibliography}{9}
\bibitem{commalg} W. Troiani, \emph{Commutative algebra}
\end{thebibliography}

\end{document}