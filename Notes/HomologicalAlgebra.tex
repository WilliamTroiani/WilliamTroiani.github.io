\documentclass[12pt]{article}

\usepackage{amsthm}
\usepackage{amsmath}
\usepackage{amsfonts}
\usepackage{mathrsfs}
\usepackage{array}
\usepackage{amssymb}
\usepackage{units}
\usepackage{graphicx}
\usepackage{tikz-cd}
\usepackage{nicefrac}
\usepackage{hyperref}
\usepackage{bbm}
\usepackage{color}
\usepackage{tensor}
\usepackage{tipa}
\usepackage{bussproofs}
\usepackage{ stmaryrd }
\usepackage{ textcomp }
\usepackage{leftidx}
\usepackage{afterpage}
\usepackage{varwidth}
\usepackage{tasks}
\usepackage{ cmll }

\newcommand\blankpage{
    \null
    \thispagestyle{empty}
    \addtocounter{page}{-1}
    \newpage
    }

\graphicspath{ {images/} }

\theoremstyle{plain}
\newtheorem{thm}{Theorem}[subsection] % reset theorem numbering for each chapter
\newtheorem{proposition}[thm]{Proposition}
\newtheorem{lemma}[thm]{Lemma}
\newtheorem{fact}[thm]{Fact}
\newtheorem{cor}[thm]{Corollary}

\theoremstyle{definition}
\newtheorem{defn}[thm]{Definition} % definition numbers are dependent on theorem numbers
\newtheorem{exmp}[thm]{Example} % same for example numbers
\newtheorem{notation}[thm]{Notation}
\newtheorem{remark}[thm]{Remark}
\newtheorem{condition}[thm]{Condition}
\newtheorem{question}[thm]{Question}
\newtheorem{construction}[thm]{Construction}
\newtheorem{exercise}[thm]{Exercise}
\newtheorem{example}[thm]{Example}

\def\doubleunderline#1{\underline{\underline{#1}}}
\newcommand{\bb}[1]{\mathbb{#1}}
\newcommand{\scr}[1]{\mathscr{#1}}
\newcommand{\call}[1]{\mathcal{#1}}
\newcommand{\psheaf}{\text{\underline{Set}}^{\scr{C}^{\text{op}}}}
\newcommand{\und}[1]{\underline{\hspace{#1 cm}}}
\newcommand{\adj}[1]{\text{\textopencorner}{#1}\text{\textcorner}}
\newcommand{\comment}[1]{}
\newcommand{\lto}{\longrightarrow}
\newcommand{\rone}{(\operatorname{R}\bold{1})}
\newcommand{\lone}{(\operatorname{L}\bold{1})}
\newcommand{\rimp}{(\operatorname{R} \multimap)}
\newcommand{\limp}{(\operatorname{L} \multimap)}
\newcommand{\rtensor}{(\operatorname{R}\otimes)}
\newcommand{\ltensor}{(\operatorname{L}\otimes)}
\newcommand{\rtrue}{(\operatorname{R}\top)}
\newcommand{\rwith}{(\operatorname{R}\&)}
\newcommand{\lwithleft}{(\operatorname{L}\&)_{\operatorname{left}}}
\newcommand{\lwithright}{(\operatorname{L}\&)_{\operatorname{right}}}
\newcommand{\rplusleft}{(\operatorname{R}\oplus)_{\operatorname{left}}}
\newcommand{\rplusright}{(\operatorname{R}\oplus)_{\operatorname{right}}}
\newcommand{\lplus}{(\operatorname{L}\oplus)}
\newcommand{\prom}{(\operatorname{prom})}
\newcommand{\ctr}{(\operatorname{ctr})}
\newcommand{\der}{(\operatorname{der})}
\newcommand{\weak}{(\operatorname{weak})}
\newcommand{\exi}{(\operatorname{exists})}
\newcommand{\fa}{(\operatorname{for\text{ }all})}
\newcommand{\ex}{(\operatorname{ex})}
\newcommand{\cut}{(\operatorname{cut})}
\newcommand{\ax}{(\operatorname{ax})}
\newcommand{\negation}{\sim}
\newcommand{\true}{\top}
\newcommand{\false}{\bot}
\newcommand{\tagarray}{\mbox{}\refstepcounter{equation}$(\theequation)$}
\newcommand{\startproof}[1]{
\AxiomC{#1}
\noLine
\UnaryInfC{$\vdots$}
}
\newenvironment{scprooftree}[1]%
  {\gdef\scalefactor{#1}\begin{center}\proofSkipAmount \leavevmode}%
  {\scalebox{\scalefactor}{\DisplayProof}\proofSkipAmount \end{center} }

\usepackage[margin=1cm]{geometry}

\title{Homological Algebra}
\author{Will Troiani}
\date{\today}

\begin{document}
\maketitle
\tableofcontents

\section{Abelian categories}
We hold off for as long as possible from specialising to a particular category.
\begin{defn}
Let $\scr{C}$ be a category and $\lbrace C_i\rbrace_{i = 1}^n$ a finite set of objects. A \textbf{direct sum} of $\lbrace C_i\rbrace_{i=1}^n$ is a tuple $(\bigoplus_{i=1}^nC_i,\lbrace \pi_i: \bigoplus_{i = 1}^nC_i \lto C_i,\rbrace_{i = 1}^n, \lbrace \iota: C_i \lto \bigoplus_{i = 1}^n\rbrace_{i = 1}^n)$ such that $(\bigoplus_{i=1}^nC_i,\lbrace \pi_i\rbrace_{i = 1}^n)$ is a finite product and $(\bigoplus_{i=1}^nC_i,\lbrace \iota_{i = 1}^n\rbrace)$ is a finite coproduct.

The finite product of the empty set is defined by to be a zero object, that is, an object which is both initial and terminal.
\end{defn}
\begin{defn}
Let $f:A \lto B$ be a morphism in a category which admits a $0$ object. A \textbf{kernel for $f$} is a pair $(\operatorname{ker}f,\iota)$ consisting of an object $\operatorname{ker}$ and a monomorphism $\iota: \operatorname{ker}f \lto A$ satisfying:

if $g: C \lto A$ is such that $fg$ factors through $0$ then $g$ factors through $\operatorname{ker}f$. Diagrammatically,  if $g$ is such that the following diagram commutes
\begin{equation}
\begin{tikzcd}
C\arrow[r]\arrow[d,swap,"g"] & 0\arrow[d]\\
A\arrow[r,"f"] & B
\end{tikzcd}
\end{equation}
then there exists a (necessarily unique) morphism $h: C \lto \operatorname{ker}f$ such that the following diagram commutes
\begin{equation}
\begin{tikzcd}
C\arrow[d,swap,"h"]\arrow[dr,"h"]\\
\operatorname{ker}f\arrow[r,rightarrowtail,swap,"\iota"] & A
\end{tikzcd}
\end{equation}
A \textbf{cokernel} for $f$ is defined analogously.
\end{defn}
\begin{defn}
A category $\scr{A}$ is \textbf{abelian} if it satisfies the following properties:
\begin{itemize}
\item for every pair of objects $A,B \in \scr{A}$, the set $\operatorname{hom}(A,B)$ is an abelian group,
\item composition is bilinear,
\item $\scr{A}$ admits all finite direct sums (including the empty direct sum)
\item every morphism admits a kernel and a cokernel,
\item every monorphism $f$ is the kernel of its cokernel,
\item every epimorphism $f$ is the cokernel of its kernel,
\item every morphism factors as a epimorphism followed by a monomorphism.
\end{itemize}
\end{defn}
Throughout, $\scr{A}$ denotes an abelian category.
\begin{defn}
The \textbf{image} of a morphism $f: A \lto B$ in $\scr{A}$ is the kernel of the cokernel of $f$.
\end{defn}
\begin{defn}
A sequence
\begin{equation}
0 \lto A \stackrel{f}{\lto} B \stackrel{g}{\lto}C \lto 0
\end{equation}
in $\scr{A}$ is \textbf{exact} is $f$ is a monomorphism, $g$ is an epimorphism, and $\operatorname{ker}g \cong \operatorname{im}f$.
\end{defn}
\begin{defn}
Let $\scr{B}$ be another abelian category. A functor $F: \scr{A} \lto \scr{B}$ is \textbf{additive} if the induced map for all $A,B \in \scr{A}$
\begin{equation}
\operatorname{hom}(A,B) \lto \operatorname{hom}(FA,FB)
\end{equation}
is a homomorphism.

An additive, covariant functor $F: \scr{A} \lto \scr{B}$ is \textbf{left exact} if given a short exact sequence in $\scr{A}$
\begin{equation}
0 \lto A \stackrel{f}{\lto} B \stackrel{g}{\lto}C \lto 0
\end{equation}
the resulting sequence
\begin{equation}
0 \lto FA \stackrel{Ff}{\lto} FB \stackrel{Fg}{\lto}FC
\end{equation}
is exact.

If $F$ is \emph{contravariant} then it is \textbf{left exact} if the resulting sequence
\begin{equation}
0 \lto FC \stackrel{Fg}{\lto} FB \stackrel{Ff}{\lto} FA
\end{equation}
is exact.

We make the analogous definitions for \textbf{right exact} functors (covariant, or contravariant).

An additive functor $F$ is \textbf{exact} if it is left and right exact.
\end{defn}
\begin{lemma}\label{lem:hom_left_exact}
Let $D \in \scr{A}$. The functor $\operatorname{hom(D,\und{0.2})}$ is left exact.
\end{lemma}
To prove Lemma \ref{lem:hom_left_exact} we first prove it in the special setting where the abelian category $\scr{A}$ is the category of left $R$-modules, where $R$ is a commutative ring with unit.
\begin{proof}[Proof of special case]
Let
\begin{equation}\label{eq:exact_seq_arb}
0 \lto A\stackrel{f}{\lto} B \stackrel{g}{\lto}C \lto 0
\end{equation}
be an exact sequence of $R$-modules.

To prove that $f \circ \und{0.2}$ is injective amounts to proving that $f$ is a monomorphism which is equivalent to $f$ being injective.

That $(g \circ \und{0.2})\circ(f\circ \und{0.2}) = 0$ follows immediately from the fact that $gf = 0$.

We now show that $\operatorname{ker}(g\circ\und{0.2}) = \operatorname{im}(f\circ\und{0.2})$. Let $h: D \lto B$ be such that $gh = 0$. Notice that for every $d \in D$ that $gh(d) = 0 \Rightarrow \exists a_d \in A$ such that $f(a_d) = h(d)$, by exactness of \eqref{eq:exact_seq_arb}.  Moreover, by injectivity of $f$ it follows that this $a_d$ is unique. We define the following homomorphism
\begin{align*}
h': D &\lto A\\
d &\longmapsto a_d
\end{align*}
This is indeed a homomorphism, to see this, notice that if $d_1,d_2 \in D$ are such that $f(a_{d_1}) = f(a_{d_2}) = h(d)$ then
\begin{equation}
a_{d_1 + d_2} - a_{d_1} - a_{d_2} \in \operatorname{ker}f = 0
\end{equation}
similarly, if $d \in D$ with $f(a_d) = h(d)$ and $r \in R$ then $f(a_{rd}) = h(rd) = rh(d) = rf(a_d)$ and so
\begin{equation}
a_{rd} - ra_d \in \operatorname{ker}f = 0
\end{equation}
We now define
\begin{align*}
\Phi: \operatorname{ker}(g \circ \und{0.2}) &\lto \operatorname{im}(f \circ \und{0.2})\\
h &\longmapsto f\circ h'
\end{align*}
which by commutativity of the following diagram:
\begin{equation}
\begin{tikzcd}
D\arrow[d,swap,"{h'}"]\arrow[dr,"{h}"]\\
A\arrow[r,swap,"{f}"] & B
\end{tikzcd}
\end{equation}
is clearly a bijection.
\end{proof}
The heart of the proof is given by the construction of $h'$ given $h$. We now come up with a definition of this map which does not appeal to ``elements of the objects $D,A$" which will generalise the proof to the setting of an arbitrary abelian category.
\begin{proof}[Proof of Lemma \ref{lem:hom_left_exact}]
Let
\begin{equation}\label{eq:exact_seq_arb_arb}
0 \lto A\stackrel{f}{\lto} B \stackrel{g}{\lto}C \lto 0
\end{equation}
be an exact sequence in $\scr{A}$, which in particular means there is an isomorphism $\varphi: \operatorname{ker}g \cong \operatorname{im}f$.Now, say $h: D \lto B \in \operatorname{ker}(g \circ \und{0.2})$, that is, say $gh = 0$.  We thus have that $h$ factors through the kernel of $g$, that is, there exists a morphism $\hat{h}: D \lto \operatorname{ker}g$ such that the following diagram commutes
\begin{equation}
\begin{tikzcd}
\operatorname{ker}g\arrow[dr,rightarrowtail] & A\arrow[d,rightarrowtail,"f"]\\
D\arrow[r,swap,"h"]\arrow[u,dashed,"{\hat{h}}"] & B\arrow[d,twoheadrightarrow,"g"]\\
& C
\end{tikzcd}
\end{equation}
Since $\scr{A}$ is abelian and $f$ is monic, we have that it is the kernel of its cokernel, that is, there exists an isomorphism $A \cong \operatorname{coker}\operatorname{ker}f$ rendering the following diagram commutative
\begin{equation}
\begin{tikzcd}
A\arrow[r,"f"]\arrow[d,swap,"{\cong}"] & B\\
\operatorname{ker}(\operatorname{coker}f)\arrow[ur]
\end{tikzcd}
\end{equation}
Moreover, by exactness of \ref{eq:exact_seq_arb_arb} we have $\operatorname{ker}g \cong \operatorname{im}f$ which implies $\operatorname{ker}(\operatorname{coker}f) \cong \operatorname{ker}g$. Let $\varphi: \operatorname{ker}g \lto A$ denote the resulting isomorphism. We now define the following homomorphism
\begin{align*}
\Phi: \operatorname{ker}(\und{0.2} \circ f) &\lto \operatorname{im}(\und{0.2} \circ g)\\
h &\longmapsto f \varphi \hat{h}
\end{align*}
It remains to show that this is an isomorphism, but this follows easily from the fact that $f$ is a monomorphism and $\varphi$ an isomorphism.
\end{proof}
\begin{lemma}
Let $D \in \scr{A}$. The functor $\operatorname{hom}(\und{0.2},D)$ is left exact.
\end{lemma}
\begin{proof}
Let
\begin{equation}\label{eq:exact_seq_arb_contr}
0 \lto A\stackrel{f}{\lto} B \stackrel{g}{\lto}C \lto 0
\end{equation}
be short exact and consider
\begin{equation}
0 \lto \operatorname{hom}(C,D) \stackrel{\und{0.2} \circ g}{\lto} \operatorname{hom}(B,D)\stackrel{\und{0.2} \circ f}{\lto}\operatorname{hom}(A,D)
\end{equation}
Let $h: D \lto C$ be a morphism such that $h \circ g = 0$. We have $h \circ g = 0 = 0 \circ g$ which implies $h = 0$ by way of $g$ being an epimorphism.

Next we show $\operatorname{ker}(D \circ f) \cong \operatorname{im}(D \circ g)$. Let $h: B \lto D$ be such that $h\circ f =0$. Then consider the following commutative diagram
\begin{equation}
\begin{tikzcd}
A\arrow[d,rightarrowtail,"f"]\\
B\arrow[r,"h"]\arrow[d,twoheadrightarrow,"g"]\arrow[dr] & D\\
C & \operatorname{coker}f\arrow[u,dashed,swap,"\hat{h}"]
\end{tikzcd}
\end{equation}
where the dashed arrow $\hat{h}$ exists by the universal property of $\operatorname{coker}f$. Now, since $\scr{A}$ is an abelian category, $g$ is the cokernel of its kernel, in other words, there is an isomorphism $C \cong \operatorname{coker}(\operatorname{ker}f)$ rendering the following diagram commutative
\begin{equation}
\begin{tikzcd}
B\arrow[r,twoheadrightarrow,"g"]\arrow[dr] & C\arrow[d,"{\cong}"]\\
& \operatorname{coker}(\operatorname{ker}g)
\end{tikzcd}
\end{equation}
Moreover, by exactness of \eqref{eq:exact_seq_arb_contr}, we have $\operatorname{ker}g \cong \operatorname{im}f$ which implies $\operatorname{coker}(\operatorname{ker}g) \cong \operatorname{coker}(f)$. Let $\varphi: C \lto \operatorname{coker}f$ denote the resulting isomorphism. We now define the following function
\begin{align*}
\Phi: \operatorname{ker}(\und{0.2} \circ f) &\lto \operatorname{im}(\und{0.2} \circ g)\\
h & \longmapsto h' \varphi g
\end{align*}
It remains to show that this is an isomorphism, but this follows easily from the fact that $g$ is an epimorphism and $\varphi$ is an isomorphism.
\end{proof}


\section{Resolutions}
Throughout,  $A$ is a commutative ring with unit.
\subsection{Short exact sequences}
\begin{defn}
Given two short exact sequences
\begin{equation}
0 \lto M \lto N \lto P \lto 0
\end{equation}
and
\begin{equation}
0 \lto M' \lto N' \lto P' \lto 0
\end{equation}
which we denote by $S_1,S_2$ respectively,  a \textbf{morphism of short exact sequences} $f: S_1 \lto S_2$ is a triple of module homomorphisms $f_1: M \lto M', f_2: N \lto N', f_3: P \lto P'$ which render the following diagram commutative:
\begin{equation}
\begin{tikzcd}
0\arrow[r] & M\arrow[d,"{f_1}"]\arrow[r] & N\arrow[d,"{f_2}"]\arrow[r] & P\arrow[d,"{f_3}"]\arrow[r] & 0\\
0\arrow[r] & M'\arrow[r] & N'\arrow[r] & P'\arrow[r] & 0
\end{tikzcd}
\end{equation}
\end{defn}
\begin{defn}
A short exact sequence of $A$-modules
\begin{equation}
0 \lto M \stackrel{f}{\lto} N \stackrel{g}{\lto} P \lto 0
\end{equation}
is \textbf{split} (or \textbf{splits}) if it is isomorphic to the short exact sequence
\begin{equation}
0 \lto M \lto M \oplus P \lto P \lto 0
\end{equation}
\end{defn}
\begin{lemma}
Given a short exact sequence
\begin{equation}
0 \lto M \stackrel{f}{\lto} N \stackrel{g}{\lto} P \lto 0
\end{equation}
which we denote by $S$, the following are equivalent:
\begin{enumerate}
\item\label{lem:split} $S$ is split,
\item\label{lem:right_inverse} $g$ admits a right inverse,
\item\label{lem:leftright_inverse} $f$ admits a left inverse.
\end{enumerate}
\end{lemma}
\begin{proof}
First assume $S$ is split. Then we have an isomorphism
\begin{equation}
\begin{tikzcd}
0\arrow[r] & M\arrow[d,"{f_1}"]\arrow[r,"f"] & N\arrow[d,"{f_2}"]\arrow[r,"g"] & P\arrow[d,"{f_3}"]\arrow[r] & 0\\
0\arrow[r] & M\arrow[r,"f'"] & M \oplus P\arrow[r,"g'"] & P\arrow[r] & 0
\end{tikzcd}
\end{equation}
The functions $f',g'$ respectively admit left and right inverses given by $m \longmapsto (m,0)$ and $p \longmapsto(0,p)$. Thus \eqref{lem:split} implies \eqref{lem:leftright_inverse} and \eqref{lem:right_inverse}.

Now say $g$ admits a right inverse, $h: P \lto N$. We have $g h = \operatorname{id}_P$ and so $h$ is injective. We thus have $P \cong \operatorname{im}h$ and similarly, $M \cong \operatorname{im}f$.

Moreover, there is a map $l: N \lto \operatorname{im}f \oplus \operatorname{im}h$ given by $n \longmapsto (n - hg(n),hg(n))$ rendering the following diagram commutative:
\begin{equation}\label{eq:five_diagram}
\begin{tikzcd}
0\arrow[r] & M\arrow[r,"f"]\arrow[d,"{f}"] & N\arrow[r,"g"] \arrow[d,"l"]& P\arrow[d,"{h}"]\arrow[r] & 0\\
0\arrow[r] & \operatorname{im}f\arrow[r] & \operatorname{im}f \oplus \operatorname{im}h\arrow[r] & \operatorname{im}h\arrow[r] & 0
\end{tikzcd}
\end{equation}
it then follows from the five Lemma that $l$ is an isomorphism. The bottom row of \eqref{eq:five_diagram} is clearly isomorphic to
\begin{equation}
0 \lto M \lto M \oplus N \lto N \lto 0
\end{equation}

Lastly,  assume that $f$ admits a right inverse $h: N \lto M$. Since $h f = \operatorname{id}_M$ we have that $h$ is surjective. Thus $N/\operatorname{ker}h \cong M$ and similarly, $N/\operatorname{ker}g \cong P$. Now, there is a map $l: N \lto N/\operatorname{ker}h \oplus N/\operatorname{ker}g$ given by the sum of the respective projection maps which fits into a commutative diagram similar to \eqref{eq:five_diagram}. The result follows similarly to before.
\end{proof}
\subsection{The tensor product}
The tensor product admits the following universal property:
\begin{lemma}\label{lem:tens_univ_prop}
Let $M,N,P$ be modules and denote the set of bilinear transformations $M \times N \lto P$ by $\operatorname{Bil}(M \times N, P)$. There is the following natural isomorphism
\begin{equation}
\operatorname{Bil}(M \oplus N, P) \cong \operatorname{Hom}(M \otimes N, P)
\end{equation}
\end{lemma}
\begin{proof}
Easy.
\end{proof}
The tensor product is distributive, that is:
\begin{lemma}\label{lem:distributivity}
Let $M,N,P$ be modules, then
\begin{equation}
M \otimes (N \oplus P) \cong (M \otimes N) \oplus (M \otimes P)
\end{equation}
\end{lemma}
\begin{proof}
We define an explicit map and an inverse. By Lemma \ref{lem:tens_univ_prop} it suffices to define the following bilinear map:
\begin{align*}
\varphi: M \oplus (N \oplus P) &\lto (M \otimes N) \oplus (M \otimes P)\\
(m,(n,p)) &\longmapsto (m \otimes n, m \otimes p)
\end{align*}
Let $\overline{\varphi}$ map induced by applying Lemma \ref{lem:tens_univ_prop}, we define an explicit inverse to $\overline{\varphi}$. Again, using Lemma \ref{lem:tens_univ_prop} and the universal property of the direct sum, it suffices to define the following two maps
\begin{align*}
\psi_1: M \oplus N &\lto M \otimes (N \oplus P) & \psi_2: M \oplus P &\lto M \otimes (N \oplus P)\\
(m,n) &\longmapsto m \otimes (n,0) & (m,p) &\longmapsto m \otimes (0,p)
\end{align*}
Let $\overline{\psi}: (M \otimes N) \oplus (M \otimes P) \lto M \otimes (N \oplus P)$ denote the induced map. We see:
\begin{align*}
\overline{\psi}\overline{\varphi}(m \otimes (n,p)) &= \overline{\psi}(m \otimes n, m \otimes p)\\
&= m \otimes (n,0) + m \otimes (0,p)\\
&= m \otimes (n,p)
\end{align*}
and
\begin{align*}
\overline{\varphi}\overline{\psi}(m \otimes n, m' \otimes p) &= \overline{\varphi}(m \otimes (n,0) + m' \otimes (0,p))\\
&= (m \otimes n + m' \otimes 0,  m \otimes 0+ m' \otimes p)\\
&= (m \otimes n, m' \otimes p)
\end{align*}
\end{proof}
In fact, the proof of Lemma \ref{lem:distributivity} generalises:
\begin{lemma}\label{lem:tens_sum_commute}
The tensor product commutes with arbitrary direct sum, more precisely, if $M_{i \in I}$ is is a collection of modules and $N$ is also a module, then
\begin{equation}
N \otimes \bigoplus_{i \in I}M_i \cong \bigoplus_{i \in I}(N \otimes M_i)
\end{equation}
\end{lemma}
\begin{proof}
Following the proof of Lemma \ref{lem:distributivity} we define
\begin{align*}
\varphi: N \oplus \bigoplus_{i \in I}M_i &\lto \bigoplus_{i \in I}(N \otimes M_i)\\
(n, (m_i)_{i \in I}) &\longmapsto (n \otimes m_i)_{i \in I}
\end{align*}
which is well defined as since $(m_i)_{i \in I}$ satisfies $m_i = 0$ for all but finitely many $i$, the same can be said of $(n \otimes m_i)_{i \in I}$. We also define an $I$-indexed family of maps
\begin{align*}
\psi_i: N \oplus M_i &\lto N \oplus \bigoplus_{i \in I}M_i\\
(n,m) &\longmapsto (n,\iota_{i}m)
\end{align*}
where
\begin{equation}
\iota_i: M_i \lto \bigoplus_{i \in I}M_i
\end{equation}
is the canonical inclusion map. It is then easy to see that the induced maps $\overline{\varphi}$ and $\overline{\psi}$ are mutual inverse to each other.
\end{proof}

\subsection{Flat modules}
\begin{defn}\label{def:flat}
A module $M$ is flat if given any short exact sequence
\begin{equation}
0 \lto N_1 \lto N_2 \lto N_3 \lto 0
\end{equation}
the induced sequence:
\begin{equation}
0 \lto N_1 \otimes M \lto N_2 \otimes M \lto N_3 \otimes M \lto 0
\end{equation}
is also short exact.
\end{defn}
Below, Lemma \ref{lem:tensor_right_exact} states that in the setting of Definition \ref{def:flat} the sequence
\begin{equation}
N_1 \otimes M \lto N_2 \otimes M \lto N_3 \otimes M \lto 0
\end{equation}
is always short exact.
\begin{defn}\label{def:right_exact}
Let $\underline{\operatorname{Mod}_A}$ denote the category of (left) $A$-modules.

A functor $F: \underline{\operatorname{Mod}_A} \lto \underline{\operatorname{Mod}_A}$ is \textbf{right exact} if given a short exact sequence
\begin{equation}
0 \lto M_1 \lto M_2 \lto M_3 \lto 0
\end{equation}
the induced sequence
\begin{equation}
F(M_1) \lto F(M_2) \lto F(M_3) \lto 0
\end{equation}
is exact.
\end{defn}
Clearly, Definition \ref{def:right_exact} need not be bound to the particular category chosen, but we worth in this restricted setting for now.
\begin{lemma}\label{lem:tensor_right_exact}
For any module $M$, the functor $\und{0.5} \otimes M$ is right exact.
\end{lemma}
\begin{proof}
Let
\begin{equation}
0 \lto N_1 \stackrel{f}{\lto} N_2 \stackrel{g}{\lto} N_3 \lto 0
\end{equation}
be an arbitrary short exact sequence and consider
\begin{equation}
N_1 \otimes M \stackrel{f \otimes \operatorname{id}}{\lto} N_2 \otimes M \stackrel{g \otimes \operatorname{id}}{\lto}N_3 \otimes M \lto 0
\end{equation}
It is clear that $g$ surjective implies $g \otimes \operatorname{id}$ is surjective.  It is also clear that $gf = 0 \Rightarrow (g \otimes \operatorname{id})(f \otimes \operatorname{id}) = 0$. Thus, it remains to show:
\begin{equation}
\operatorname{im}(f \otimes \operatorname{id}) \supseteq\operatorname{ker}(g \otimes \operatorname{id})
\end{equation}
We do this by showing there exists an isomorphism
\begin{equation}
(N_2 \otimes M)/(\operatorname{im}(f \otimes \operatorname{id}))\cong N_3 \otimes M
\end{equation}
The map $g \otimes \operatorname{id}$ induces a homomorphism $\overline{g \otimes \operatorname{id}}: (N_2 \otimes M)/\operatorname{im}(f \otimes \operatorname{id}) \lto N_3 \otimes M$, we construct a right inverse.

Let $h: N_3 \otimes M \lto (N_2 \otimes M)/(\operatorname{im}f \otimes \operatorname{id})$ be such that $h(n \otimes m) = [n' \otimes m]_{\operatorname{im}(f \otimes \operatorname{id})}$. where $n'$ is an arbitrary element of $N_2$ such that $g(n') = n$. This is well defined, as if $n'' \in N_2$ is also such that $g(n'') = n$, then $n' - n'' \in \operatorname{ker}g = \operatorname{im}f$ which means $[n' \otimes m]_{\operatorname{im}(f \otimes \operatorname{id})} = [n'' \otimes m]_{\operatorname{im}(f \otimes \operatorname{id})}$. Notice that $h$ is clearly a right inverse to $\overline{g \otimes \operatorname{id}}$.
\end{proof}
Thus, we have the following definition of a flat module:
\begin{cor}
A module $M$ is flat if and only if it satisfies the following condition:

for any injective morphism $f: N \lto N'$ the induced morphism $f \otimes \operatorname{id}: N \otimes M \lto N' \otimes M$ is injective.
\end{cor}
\begin{example}
A \emph{non-example} of a flat module, ie, a module which is not flat, is given by $\bb{Z}/n\bb{Z}$, for any $n$. Indeed, consider the following short exact sequence
\begin{equation}
0 \lto \bb{Z} \stackrel{\times n}{\lto}\bb{Z} \lto \bb{Z}/n\bb{Z} \lto 0
\end{equation}
which induces the following sequence
\begin{equation}
\bb{Z} \otimes \bb{Z}/n\bb{Z} \lto \bb{Z} \otimes \bb{Z}/n\bb{Z} \lto \bb{Z}/n\bb{Z} \otimes \bb{Z}/n\bb{Z} \lto 0
\end{equation}
which is isomorphic to
\begin{equation}
\bb{Z}/n\bb{Z} \stackrel{\times n}{\lto} \bb{Z}/n\bb{Z} \lto \bb{Z}/n\bb{Z} \otimes \bb{Z}/n\bb{Z} \lto 0
\end{equation}
and the map $\bb{Z}/n\bb{Z} \stackrel{\times n}{\lto} \bb{Z}/n\bb{Z}$ is clearly not injective.
\end{example}
\begin{example}
Free modules are flat. Indeed, if $f: N \lto N'$ is injective, then since the tensor product and direct sum commute (Lemma \ref{lem:tens_sum_commute}) we have the following commuting diagram where the vertical arrows are isomorphisms:
\begin{equation}
\begin{tikzcd}
N \otimes A^I\arrow[r]\arrow[d] & N' \otimes A^I\arrow[d]\\
(N \otimes A)^I\arrow[d]\arrow[r] & (N' \otimes A)^I\arrow[d]\\
N^I\arrow[r] & N'^I
\end{tikzcd}
\end{equation}
\end{example}
\begin{example}
If $M$ is flat and $M \cong N \oplus P$ then both $N$ and $P$ are flat. Indeed, assume $f: O \lto O'$ is injective and denote the inclusion $P \rightarrowtail M$ by $i$. Consider the following commuting diagram
\begin{equation}
\begin{tikzcd}[column sep = huge]
O \otimes P\arrow[r,"{f \otimes \operatorname{id}_P}"]\arrow[d,swap,"{\operatorname{id}_O \otimes i}"] & O' \otimes P\arrow[d,"{\operatorname{id}_{O'} \otimes i}"]\\
O \otimes M\arrow[r,"{f \otimes \operatorname{id}_M}"] & O' \otimes M
\end{tikzcd}
\end{equation}
We have by assumption that $f \otimes \operatorname{id}_M$ is injective, we finish the proof by showing $\operatorname{id}_{O'} \otimes i$ is injective. 

We have the following commutative diagram:
\begin{equation}
\begin{tikzcd}[column sep = huge]
O' \otimes P\arrow[rrd,rightarrowtail]\arrow[d,swap,"{\operatorname{id}_{O'} \otimes i}"]\\
O' \otimes M\arrow[r,"\sim"] & O' \otimes (N \oplus P)\arrow[r,"{\sim}"] & (O' \otimes N) \oplus (O' \otimes P)
\end{tikzcd}
\end{equation}
\end{example}
\subsection{Chain complexes}
Throughout we work in an abelian category $\scr{A}$, first we relax the notion of an exact sequence:
\begin{defn}
A \textbf{chain complex} is a sequence
\begin{equation}
\hdots \stackrel{\partial_{n+1}}{\lto} M_n \stackrel{\partial_{n}}{\lto} M_{n-1} \stackrel{\partial_{n+1}}{\lto} \hdots
\end{equation}
such that for all $n$, $\partial_{n-1}\partial_n = 0$. We denote a chain complex by $(M_\bullet,\partial_\bullet)$ or simply $(M,\partial)$. The element in the $i^{\text{th}}$ position in the complex consists of the \textbf{degree $i$} elements. We sometimes say that the module $M_i$ is in \textbf{position $i$}.
\end{defn}
\begin{remark}
We remark that this notion of degree is consistent with that of a \emph{graded module} (see \cite{ComAlg}).
\end{remark}
\begin{defn}
A \textbf{morphism} of chain complexes $f_\bullet: (M_\bullet,\partial_\bullet) \lto (N_\bullet, \partial_\bullet')$ is a set of morphisms $\lbrace f_n: M_n \lto N_n$ such that for all $n$ the following diagram commutes
\begin{equation}
\begin{tikzcd}
M_n\arrow[r,"{\partial_{n}}"]\arrow[d,swap,"{f_n}"] & M_{n-1}\arrow[d,swap,"{f_n}"]\\
N_n\arrow[r,"{\partial_n'}"] & N_{n-1}
\end{tikzcd}
\end{equation}
\end{defn}
\begin{defn}
Given a chain complex $(M_\bullet, \partial_\bullet)$, the \textbf{$n$-th homology object} $H_n(M_\bullet,\partial_\bullet)$ is given as follows: since $\partial_{n}\partial_{n+1} = 0$ we have that $\partial{n+1}$ factors through $\operatorname{ker}\partial_n$, ie, there is a morphism $j: M_{m+1} \lto \operatorname{ker}\partial_n$ such that the following diagram commutes:
\begin{equation}
\begin{tikzcd}
& & \operatorname{ker}\partial_{m}\arrow[d,rightarrowtail]\arrow[r] & \operatorname{coker}j\\
\hdots\arrow[r] & M_{m+1}\arrow[r,"{\partial_{m+1}}"]\arrow[ur,"{j}"] & M_m\arrow[r,"{\partial_m}"] & M_{m-1}\arrow[r,"{\partial_{m-1}}"] & \hdots
\end{tikzcd}
\end{equation}
we then define $H_n(M_\bullet,\partial_\bullet)$ to be the object $\operatorname{coker}j$.
\end{defn}
\begin{defn}
A \textbf{chain homotopy} between two morphisms $f_\bullet, g_\bullet$ of chain complexes
\begin{equation}
\hdots \stackrel{\partial_{n+1}}{\lto} M_n \stackrel{\partial_{n}}{\lto} M_{n-1} \stackrel{\partial_{n+1}}{\lto} \hdots
\end{equation}
and
\begin{equation}
\hdots \stackrel{\partial_{n+1}'}{\lto} N_n \stackrel{\partial_{n}'}{\lto} N_{n-1} \stackrel{\partial_{n+1}'}{\lto} \hdots
\end{equation}
is a set of maps $\lbrace d_n: M_{n-1} \lto M_n\rbrace_{n \in \bb{Z}}$ such that for all $n$,
\begin{equation}
\partial'_nd_n + d_{n+1}\partial_{n+1} = f - g
\end{equation}
\end{defn}




\begin{remark}
Any two chain homotopy morphisms induce the same map on homology.
\end{remark}
We now specialise to the case where $\scr{A}$ is the category of left $R$-modules, with $R$ a ring. We introduce two operations on chain complexes in $\scr{A}$, the \emph{tensor product} and the \emph{mapping cone}.

\begin{defn}
Given an arbitrary chain complex $(M, \partial)$ we denote by $M(n)$ the chain complex identical to $M$ but with the positions shifted by $n$. More precisely,
\begin{equation}
M(n)_m = M_{n+m}
\end{equation}
If $\partial$ is the differential of $M$, then the differential for $M(n)$, denoted $\partial(n)$ is given by $\partial(n) = (-1)^n\partial$.

Let $R$ be a ring considered as a module over itself. Denote the following chain complex:
\begin{equation}
\hdots \lto 0 \lto 0 \lto R \lto 0 \lto \hdots
\end{equation}
where $R$ occurs in position zero.
\end{defn}
Let $R$ be a ring and $y \in R$ an element of $R$. We construct the following diagram:
\begin{equation}\label{eq:trivial}
\begin{tikzcd}
R(-1) & \hdots\arrow[r] & 0\arrow[r]\arrow[d] & R\arrow[r]\arrow[d,"{\operatorname{id}_R}"] & 0\\
K(y) & \hdots\arrow[r] & R\arrow[r,"y"]\arrow[d,"{\operatorname{id}_R}"] & R\arrow[r]\arrow[d] & 0\\
R & \hdots\arrow[r] & R\arrow[r] & 0\arrow[r] & 0
\end{tikzcd}
\end{equation}
where $K(y)$ denotes the Koszul complex corresponding to $y$ (see \cite{ComAlg}).

Now say we had an arbitrary complex $\scr{G}$:
\begin{equation}
\scr{G}\qquad \hdots \lto G_i \stackrel{\psi_i}{\lto} G_{i+1} \stackrel{\psi_{i+1}}{\lto} G_{i+2} \stackrel{\psi_{i+2}}{\lto} \hdots
\end{equation}
We tensor \eqref{eq:trivial} with $\scr{G}$ to obtain:
\begin{equation}\label{eq:tensored_trivial}
\begin{tikzcd}
\scr{G}(-1) & \hdots\arrow[r] & G_{i-1}\arrow[r]\arrow[d] & G_{i}\arrow[r]\arrow[d] & G_{i+1}\arrow[r]\arrow[d] & \hdots\\
K(y) \otimes \scr{G} & \hdots\arrow[r] & G_{i-1} \oplus G_i\arrow[r]\arrow[d] & G_{i} \oplus G_{i+1}\arrow[r]\arrow[d] & G_{i+1} \oplus G_{i+2}\arrow[r]\arrow[d] & \hdots\\
\scr{G} & \hdots\arrow[r] & G_{i}\arrow[r] & G_{i+1}\arrow[r] & G_{i+2}\arrow[r] & \hdots
\end{tikzcd}
\end{equation}
where the vertical maps are:
\begin{align*}
G_{i-1} &\lto G_{i-1} \oplus G_i & G_{i-1} \oplus G_i &\lto G_{i}\\
g &\longmapsto (g,0) & (g_{i-1},g_i) &\longmapsto g_i
\end{align*}





From Diagram \eqref{eq:tensored_trivial} we then construct a long exact sequence on homology:
\begin{equation}
\hdots \lto H^i(\scr{G}) \stackrel{\delta}{\lto} H^i(\scr{G}) \lto H^{i+1}(K(y) \otimes \scr{G}) \lto H^{i+1}(\scr{G}) \stackrel{\delta}{\lto} H^{i+1}(\scr{G}) \lto \hdots
\end{equation}
we now show that the connecting morphism $\delta_i$ is multiplication by $(-1)^i y$.

Let $g \in G_{i}$ by such that $\partial_{\scr{G}}(g) = 0$ (where $\partial$ is the differential of $\scr{G}$). The map $G_{i+1} \oplus G_i \lto G_i$ is surjective, and so we may pick a lift, an obvious choice is $(0,g)$. Now we calculate $\partial_{K(y) \otimes \scr{G}}(0,g)$. We have:
\begin{equation}
\begin{tikzcd}[column sep = tiny]
G_{i-1} \oplus G_i\arrow[rrr]\arrow[ddd] & & & G_i \oplus G_{i+1}\\
& (0,g)\arrow[d,mapsto] & ((-1)^{i}g y,0)\\
& (0, g \otimes 1)\arrow[r,mapsto] & \big((-1)^{i}g \otimes y,\partial_{\scr{G}}(g)\big) = \big((-1)^{i}g \otimes y,0\big)\arrow[u,mapsto]\\
(G_{i-1} \otimes R) \oplus (G_i \otimes R)\arrow[rrr]  & & & (G_i \otimes R) \oplus (G_{i+1} \otimes R)\arrow[uuu]
\end{tikzcd}
\end{equation}
where we have used the fact that $\partial_{\scr{G}}(g) = 0$.

Thus, the appropriate lift along $G_{i+1} \lto G_{i+1} \oplus G_{i+2}$ to take is $(-1)^i gy$, as claimed.






\subsection{Injective modules}
\begin{defn}
An $A$-module $I$ is \textbf{injective} if every short exact sequence
\begin{equation}
0 \lto I \lto M \lto N \lto 0
\end{equation}
splits.
\end{defn}
\begin{example}
All finite dimensional vector spaces are injective. To see this, let $k$ be a field and $V$ a $k$-vector space, denote by $v_1,...,v_n$ a basis for $V$. Consider a short exact sequence
\begin{equation}
0 \lto V \lto M \lto N \lto 0
\end{equation}
then the images of $v_1,,,.v_n$ under $V \lto N$ can be extended to a basis of $M$. These extra vectors span a subspace $M' \subseteq M$ such that $M \cong V \oplus M'$.
\end{example}
\begin{lemma}\label{lem:inj_mod_defs}
Let $I$ be an $A$-module, then the following are equivalent:
\begin{enumerate}
\item\label{lem:inj_mod_defs_def} $I$ is injective.
\item\label{lem:inj_mod_defs_internal} if $I$ is a submodule of some other $R$-module $M$, then there exists another submodule $M' \subseteq M$ such that $M = I + M'$ and $I \cap M' = \lbrace 0 \rbrace$ (in other words, $M$ is the internal direct sum of $I$ and $M'$),
\item\label{lem:inj_mod_defs_morphism} given an injective homomorphism $f: M \lto N$ and $g: M \lto I$ an arbitrary morphims, then there exists a morphism $h: N \lto I$ such that the following diagram commutes
\begin{equation}
\begin{tikzcd}
M\arrow[d,"g"]\arrow[r,rightarrowtail,"f"] & N\arrow[dl,dashed]\\
I
\end{tikzcd}
\end{equation}
\end{enumerate}
\end{lemma}
\begin{proof}
(\ref{lem:inj_mod_defs_def} $\Rightarrow$ \ref{lem:inj_mod_defs_internal}): Construct the following short exact sequence
\begin{equation}
0 \lto I \lto M \lto \operatorname{coker}M \lto 0
\end{equation}
which necessarily splits. We thus have $M = I + {\operatorname{coker}M}$.

That \ref{lem:inj_mod_defs_internal} $\Rightarrow$ \ref{lem:inj_mod_defs_def} is obvious.

(\ref{lem:inj_mod_defs_def} $\Rightarrow$ \ref{lem:inj_mod_defs_morphism}): Consider the pushout $I \sqcup_M N$ and appeal to the fact that the resulting short exact sequence splits. Conversely, let a short exact sequence be given:
\begin{equation}
0 \lto I \stackrel{f}{\lto} M \lto N \lto 0
\end{equation}
and consider the diagram
\begin{equation}
\begin{tikzcd}
I\arrow[r,rightarrowtail,"f"]\arrow[d,"{\operatorname{id}}"]& M\\
I
\end{tikzcd}
\end{equation}
from which we obtain a left inverse to $f$.
\end{proof}
\begin{remark}
Notice that condition \ref{lem:inj_mod_defs_morphism} of Lemma \ref{lem:inj_mod_defs} is equivalent the following:

if $f: M \rightarrowtail N$ is injective then $(\und{0.2} \circ f): \operatorname{hom}(N, I) \lto \operatorname{hom}(M,I)$ is surjective.

Thus, a module $I$ is injective if and only if $\operatorname{hom}(\und{0.2},I)$ is exact.
\end{remark}
For the next definition, recall that a subobject of a an object $A$ in some category is a monomorphism $A' \rightarrowtail A$.
\begin{defn}
If the abelian category $\scr{A}$ is such that every object is isomorphic to a subobject of an injective object, then $\scr{A}$ \textbf{has enough injectives}.
\end{defn}
If $\scr{A}$ has enough injectives then every object admits an injective resolution. To see this, let $A \in \scr{A}$ and let $I_0$ be such that $i_0: A \rightarrowtail I_0$ is a subobject. By considering the morphism $I_0 \lto \operatorname{coker}i_0$ composed with $i'_0: \operatorname{coker}i \rightarrowtail I_1$, where $I_1$ is again injective, we obtain after repeating this process an injective resolution for $A$:
\begin{equation}
0 \lto A \stackrel{i_0}{\lto} I_0 \stackrel{i'_1i_1}{\lto} I_1 \stackrel{i'_2i_2}{\lto}\hdots
\end{equation}
\begin{remark}
Note that it is true but non-trivial that the category of $R$-modules ($R$ a commutative ring) has enough injectives. See for instance \cite{chung}
\end{remark}


\subsection{Projective modules}\label{Sec:proj_mod}
\begin{defn}
An $A$-module $P$ is \textbf{projective} if every short exact sequence
\begin{equation}
0 \lto M \lto N \lto P \lto 0
\end{equation}
splits.
\end{defn}
\begin{example}\label{ex:free_split}
Free modules are projective. Indeed, consider an arbitrary short exact sequence
\begin{equation}
0 \lto M \lto N \stackrel{f}{\lto} A^S \lto 0
\end{equation}
then we define a right inverse of $f$ by mapping the unit of the $s^{\text{th}}$ copy of $A$ to any lift along $f$ of it. This induces a well defined homomorphism as $A^S$ is free.
\end{example}
\begin{example}\label{ex:summands_proj}
If $P$ is projective and $P \cong N \oplus O$ then both $N$ and $O$ are also projective.
\end{example}
\begin{proof}
Say we have the following short exact sequence
\begin{equation}
0 \lto M_1 \lto M_2 \lto N \lto 0
\end{equation}
then the following sequence is also short exact
\begin{equation}
0 \lto M_1 \oplus O \lto M_2 \oplus O \lto N \oplus O \lto 0
\end{equation}
which is split by hypothesis. There thus exists a right inverse to $M_2 \oplus O \lto N \oplus O$ from which one can derive a right inverse to $M_2 \lto N$.
\end{proof}
\begin{lemma}
Let $P$ be an $A$-module, then the following are equivalent:
\begin{enumerate}
\item\label{lem:projective} $P$ is projective,
\item\label{lem:summand} there exists an $A$-module $M$ such that $M \oplus P$ is free,
\item\label{lem:lift} given a surjective homomorphism $f: N \twoheadrightarrow M$ and an arbitrary homomorphism $g: P \lto M$, there exists a homomorphism $h: P \lto N$ such that the following diagram commutes:
\begin{equation}
\begin{tikzcd}
& P\arrow[dl,swap,dashed,"{h}"]\arrow[d,"{g}"]\\
N\arrow[r,twoheadrightarrow,"{f}"] & M
\end{tikzcd}
\end{equation}
\end{enumerate}
\end{lemma}
\begin{proof}
Say $P$ is projective and let $S$ be a set of generators for $P$. and let $\varphi: A^S \lto P$ be such that $e_s \longmapsto s$ where $e_s$ is the unit of the $s^{\text{th}}$ copy of $A$. Then there is the following short exact sequence
\begin{equation}
0 \lto \operatorname{ker}\varphi \lto A^S \stackrel{\varphi}{\lto} P \lto 0
\end{equation}
which is split as $P$ is projective. Thus $A^S \cong \operatorname{ker}\varphi \oplus P$. Thus \eqref{lem:projective} implies \eqref{lem:summand}.

To see that \eqref{lem:summand} implies \eqref{lem:projective} we observe that free modules are projective (Example \ref{ex:free_split}) and that summands of projective modules are projective \eqref{ex:summands_proj}.

Next we prove \eqref{lem:projective} implies \eqref{lem:lift}.  This is done by considering the fibred product $M \times_N P$. For the converse, say
\begin{equation}
0 \lto M \lto N \stackrel{g}{\lto} P \lto 0
\end{equation}
is exact. Then $N \lto P$ is surjective, so we consider the following diagram:
\begin{equation}
\begin{tikzcd}
& P\arrow[d,"{\operatorname{id}_P}"]\\
N\arrow[r,twoheadrightarrow,"g"] & P
\end{tikzcd}
\end{equation}
from which we obtain a left inverse to $g$.
\end{proof}
\begin{defn}
A \textbf{free resolution} of a module $M$ is an exact sequence
\begin{equation}
\hdots \stackrel{\partial_3}{\lto} M_2 \stackrel{\partial_2}{\lto} M_1 \stackrel{\partial_1}{\lto} M_0 \stackrel{\partial_0}{\lto}M \lto 0
\end{equation}
where each $M_i$ for $i \geq 0$ is free. We denote this $\partial: M_\bullet \lto M$.

One defines similarly a \textbf{projective resolution} (which we also denote by $\partial: M_\bullet \lto M$).

Given two resolutions (free or projective) $\partial: M_{\bullet} \lto M, \partial': N_{\bullet} \lto N$, there is an obvious notion of a \textbf{morphism} of (free or projective, the definition is identical) resolutions which we denote $f: \partial \lto \partial'$.
\end{defn}


We write down some easy to prove facts:
\begin{fact}
Every module admits a free resolution.
\end{fact}
and thus:
\begin{fact}
Every module admits a projective resolution.
\end{fact}
\begin{fact}
If $f: M \lto N$ is a homomorphism and say we have projective resolutions $\partial: M_{\bullet} \lto M$ and $\partial': N_{\bullet} \lto N$, then there exists a morphism of resolutions $\partial \lto \partial'$.
\end{fact}
Let $f: M \twoheadrightarrow N$ be surjective and consider an arbitrary $g: P \lto N$. Since $P$ is projective there exists a morphism $l: P \lto M$ rendering the following diagram commutative
\begin{equation}
\begin{tikzcd}
& P\arrow[dl,swap,"l"]\arrow[d,"g"]\\
M\arrow[r,swap,twoheadrightarrow,"f"] & N
\end{tikzcd}
\end{equation}
In other words, the functor $\operatorname{hom}(P,\und{0.2})$ is exact. We now arrive at the following more general definition of a module being projective:
\begin{defn}
Let $\scr{A}$ be an abelian category. An object $P \in \scr{A}$ is \textbf{projective} if the functor $\operatorname{hom}(P,\und{0.2})$ is exact.
\end{defn}
\begin{defn}
An abelian category $\scr{A}$ \textbf{has enough projectives} if for every object $A \in \scr{A}$ there exists a projective objects $P \in \scr{A}$ and an epimorphism $P \twoheadrightarrow A$.
\end{defn}
If $\scr{A}$ has enough projectives then every object admits a projective resolution. To see this, let $A \in \scr{A}$ and let $P_0$ be projective and $p_0: P_0 \twoheadrightarrow A$ an epimorphism. By considering the morphism $p'_0: \operatorname{ker}p_0 \rightarrowtail P_0$ pre composed with $P_1 \stackrel{p_1}{\twoheadrightarrow} \operatorname{ker}p_0$, where $P_1$ again is projective, we obtain after repeating this process a projective resolution for $A$:
\begin{equation}
\hdots \stackrel{p_1'p_2}{\lto} P_1 \stackrel{p_0'p_1}{\lto}P_0 \stackrel{p_0}{\lto} A \lto 0
\end{equation}
\begin{defn}
A \textbf{chain homotopy} between two resolutions
\begin{equation}
\hdots \lto M_2 \stackrel{i_2}{\lto} M_1 \stackrel{i_1}{\lto} M_0 \stackrel{i_0}{\lto} 0
\end{equation}
and
\begin{equation}
\hdots \lto N_2 \stackrel{j_2}{\lto} N_1 \stackrel{j_1}{\lto} N_0 \stackrel{j_0}{\lto} 0
\end{equation}
say is a set of maps $\lbrace d_n: M_d$
\end{defn}



\begin{proposition}\label{prop:induced_chain_map}
Let $f: M \lto N$ be a morphism in an abelian category with enough projectives. Let $P_\bullet, P_\bullet'$ be projective resolutions respectively for $M,N$. Then there is a chain map $f_\bullet: (P_\bullet \lto M) \lto (P_\bullet' \lto M)$. Moreover, the following sequences
\begin{equation}\label{eq:chain_one}
\hdots P_{2} \lto P_1 \lto P_0 \lto 0
\end{equation}
and
\begin{equation}\label{eq:chain_two}
\hdots P_2' \lto P_1' \lto P_0' \lto 0
\end{equation}
may not be exact sequences (so may not be resolutions), however they are chain maps. Given two such chain maps $f_\bullet,g_\bullet$, there is a chain homotopy between \eqref{eq:chain_one} and \eqref{eq:chain_two}.
\end{proposition}
\begin{proof}
We construct $f_\bullet$ inductively. Consider the diagram of solid arrows
\begin{equation}
\begin{tikzcd}
P_0\arrow[r,twoheadrightarrow,"{i_0}"]\arrow[d,dashed,swap,"{f_0}"] & M\arrow[d,swap,"f"]\arrow[r] & 0\\
P_0'\arrow[r,swap,twoheadrightarrow,"{i_0'}"] & N\arrow[r] & 0
\end{tikzcd}
\end{equation}
Then since $P_0$ is projective, there exists a morphism $f_0: P_0 \lto P_0'$ rending the diagram commutative. Now consider the following diagram
\begin{equation}
\begin{tikzcd}
P_{j+1}\arrow[rr,"{i_{j+1}'}"]\arrow[ddr,dotted]\arrow[d,dashed] & & P_j\arrow[rr,"{i_j}"]\arrow[d,swap,"{f_j}"] & & P_{j-1}\\
P_{j+1}'\arrow[dr,twoheadrightarrow]\arrow[rr,"{i_{j+1}'}"] & & P_j'\arrow[rr,"{i_j'}"] & & P_{j-1}'\\
 & \operatorname{ker}i_{j}\arrow[ur,rightarrowtail]
\end{tikzcd}
\end{equation}
the dotted arrow exists by commutativity and that $i_j' i_{j+1}' = 0$, and the dashed line exists by projectivity of $P_{j+1}$.
\end{proof}
\begin{remark}\label{rmk:chain_homotopy}
It is in Proposition \ref{prop:induced_chain_map} that we see why projective modules are a natural consideration. They are the minimal requirement to obtain this Proposition. Free modules also satisfy this, but we can ask for less.

For the second claim, consider the following Diagram:
\begin{equation}
\begin{tikzcd}[column sep = huge, row sep = huge]
\hdots\arrow[r] & P_{2}\arrow[r,"{i_{2}'}"]\arrow[d,shift left,"{f_{2}}"]\arrow[d,shift right,swap,"{g_{2}}"] & P_1\arrow[r,"{i_1}"]\arrow[d,shift left,"{f_1}"]\arrow[d,shift right,swap,"{g_1}"] & P_{0}\arrow[d,shift left, "{g_0}"]\arrow[d,shift right,swap, "{f_0}"]\arrow[r] & 0\\
\hdots\arrow[r] & P_{2}'\arrow[r,"{i_{2}'}"] & P_1'\arrow[r,"{i_1'}"] & P_{0}'\arrow[r] & 0
\end{tikzcd}
\end{equation}
We have that $i_0'(f_0 - g_0) = f - f = 0$ and so $f_0 - g_0$ factors through $\operatorname{ker}i_0'$. There is also a surjective map $P_1' \twoheadrightarrow \operatorname{ker}i_0'$ and so there exists a map $d_1: P_0 \lto P_1'$ such that the following diagram commutes:
\begin{equation}
\begin{tikzcd}[column sep = huge, row sep = huge]
& P_{0}\arrow[d,"{f_0 - g_0}"]\arrow[r]\arrow[dl,swap,"{d_1}"] & 0\\
P_1'\arrow[r,"{i_1'}"] & P_{0}'\arrow[r] & 0
\end{tikzcd}
\end{equation}
By setting $d_0: 0 \lto P_0'$ and $j: P_0 \lto 0$ when then have that $f_0 - g_0 = d_0 j + i_1'd_1$.

We then notice that
\begin{equation}
i_1'(f_1 - g_1) = i'_1f_1 - i_1'g_1 = f_0i_1 - g_0i_1 = (d_0i_0 + i_1'd_1)i_1 = i_1'd_1i_1
\end{equation}
and so $f_1 - g_1 - d_1i_1$ factors through $\operatorname{ker}i_1'$. There is also a surjective map $P_2' \twoheadrightarrow \operatorname{ker}i_1'$ so there we obtain the necessary map $d_1$. Continuing in this way we construct $d_n$.

Notice that we did not need that $P_n'$ were projective.
\end{remark}
\begin{proposition}
In an abelian category with enough projectives, any two projective resolutions of the same object give rise to naturally isomorphic homology.
\end{proposition}
\begin{proof}




First we construct the following diagram:
\begin{equation}
\begin{tikzcd}
& & & \operatorname{ker}i_0\arrow[rr,twoheadrightarrow]\arrow[dr,rightarrowtail,"{\iota_0}"]\arrow[dd,swap,near end, "{f_0^c}"]\arrow[rr,"{l_0}"] & & \operatorname{coker}j_0\arrow[dd,swap,near end, "{f_0^{\text{ck}}}"]\\
\hdots\arrow[rr] & & P_1\arrow[ur,"{j_0}"]\arrow[rr,near end,"{i_1}"] & & P_0\arrow[rr] & & 0\\
& & & \operatorname{ker}i_0'\arrow[rr,twoheadrightarrow]\arrow[dr,rightarrowtail,"{\iota_0'}"]\arrow[dd,swap,near end, "{g_0^c}"] & & \operatorname{coker}j_0'\arrow[dd,swap,near end, "{g_0^{\text{ck}}}"]\\
\hdots\arrow[rr] & & P_1'\arrow[ur,"{j_0}"]\arrow[rr,near end,"{i_1'}"] & & P_0'\arrow[rr]  & & 0\\
& & & \operatorname{ker}i_0\arrow[rr,twoheadrightarrow]\arrow[dr,rightarrowtail,"{\iota_0}"] & & \operatorname{coker}j_0\\
\hdots\arrow[rr] & & P_1\arrow[ur,"{j_0}"]\arrow[rr,"{i_1}"] & & P_0\arrow[rr] & & 0
\end{tikzcd}
\end{equation}
We know from Remark \ref{rmk:chain_homotopy} that there exists a chain homotopy $\lbrace d_n: P_{n-1} \lto P_n\rbrace$ (where $P_{-1} = 0$) between $g_\bullet f_\bullet$ and $\operatorname{id}_{P_\bullet}$. We calculate:
\begin{align*}
\iota_0g_0^cf_0^c &= g_0f_0\iota_0\\
&= (d_0 0 + i_1 d_1 + \operatorname{id}_{P_0})\iota_0\\
&= 0 + i_1d_1\iota_0 + \iota_0\\
&= \iota_0
\end{align*}
so since $\iota_0$ is monic, we have $g_0^cf_0^c = \operatorname{id}_{\operatorname{ker}P_0}$. It then follows that $g_0^{\text{ck}}f_0^{\text{ck}} l_0= l_0$ which since $l_0$ is epic implies $g_0^{\text{ck}}f_0^{\text{ck}} = \operatorname{id}_{\operatorname{coker}j_0}$.

Next, we consider the following diagram:
\begin{equation}
\begin{tikzcd}
& & & \operatorname{ker}i_2\arrow[rr,twoheadrightarrow]\arrow[dr,rightarrowtail,"{\iota_1}"]\arrow[dd,swap,near end, "{f_1^c}"]\arrow[rr,"{l_2}"] & & \operatorname{coker}j_1\arrow[dd,swap,near end, "{f_1^{\text{ck}}}"]\\
\hdots\arrow[rr] & & P_1\arrow[ur,"{j_1}"]\arrow[rr,near end,"{i_2}"] & & P_1\arrow[rr] & & P_0\\
& & & \operatorname{ker}i_2'\arrow[rr,twoheadrightarrow]\arrow[dr,rightarrowtail,"{\iota_1'}"]\arrow[dd,swap,near end, "{g_1^c}"] & & \operatorname{coker}j_1'\arrow[dd,swap,near end, "{g_1^{\text{ck}}}"]\\
\hdots\arrow[rr] & & P_1'\arrow[ur,"{j_1}"]\arrow[rr,near end,"{i_2'}"] & & P_1'\arrow[rr]  & & P_0\\
& & & \operatorname{ker}i_2\arrow[rr,twoheadrightarrow]\arrow[dr,rightarrowtail,"{\iota_1}"] & & \operatorname{coker}j_1\\
\hdots\arrow[rr] & & P_1\arrow[ur,"{j_1}"]\arrow[rr,"{i_2}"] & & P_1\arrow[rr] & & P_0
\end{tikzcd}
\end{equation}
We calculate, in the following, $\gamma: P_1 \lto \operatorname{ker}i_1$ is the map induced by the fact that $P_1 \lto P_1' \lto P_1 \lto P_0$ is $0$:
\begin{align*}
\iota_1 g_1^cf_1^c &= g_1 f_1 \iota_1\\
&= (d_1 i_1 + i_2 d_2 + \operatorname{id}_{P_1})\iota_1\\
&= d_1 i_1 \iota_1 + i_2 d_2 \iota_1 + \iota_1\\
&= i_2 d_2 \iota_1 + \iota_1\\
&= \iota_1 \gamma \iota_1 + \iota_1
\end{align*}
and so $g_1^c f_1^c = \gamma \iota_1 + \operatorname{id}_{\operatorname{ker}i_1}$.

We now claim $g_1^{\text{ck}}f_1^{\text{ck}} = \operatorname{id}_{\operatorname{coker}j_1}$, since $l_1$ is epic, it suffices to show
\begin{equation}
g_1^{\text{ck}} f_1^{\text{ck}} l_1 = l_1
\end{equation}
We calculate:
\begin{align*}
g_1^{\text{ck}} f_1^{\text{ck}} l_1 &= l_1 g_1^cf_1^c\\
&= l_1(\gamma \iota_1 + \operatorname{id}_{\operatorname{ker}i_1})\\
&= l_1 \gamma \iota_1 + l_1\\
&= l_1
\end{align*}
Following in this way, we have $g_\bullet^{\text{ck}} f_\bullet^{\text{ck}} = \operatorname{id}_\bullet$. That $ f_\bullet^{\text{ck}}g_\bullet^{\text{ck}} = \operatorname{id}_\bullet$ follows via a symmetric argument.
\end{proof}

\subsection{Derived functors}



















\begin{thebibliography}{9}
\bibitem{chung} \url{https://sites.math.washington.edu/~mitchell/Algh/chung.pdf}

\bibitem{ComAlg} \emph{W. Troiani}, Notes on commutative algebra.
\end{thebibliography}
\end{document}




































































