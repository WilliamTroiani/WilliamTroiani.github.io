\documentclass[12pt]{article}

\usepackage{amsthm}
\usepackage{amsmath}
\usepackage{amsfonts}
\usepackage{mathrsfs}
\usepackage{amssymb}
\usepackage{units}
\usepackage{graphicx}
\usepackage{tikz-cd}
\usepackage{nicefrac}
\usepackage{hyperref}
\usepackage{bbm}
\usepackage{color}
\usepackage{tensor}
\usepackage{tipa}
\usepackage{bussproofs}
\usepackage{ stmaryrd }
\usepackage{ textcomp }
\usepackage{leftidx}
\usepackage{afterpage}
\usepackage{varwidth}
\usepackage{physics}

\newcommand\blankpage{
	\null
	\thispagestyle{empty}
	\addtocounter{page}{-1}
	\newpage
}

\graphicspath{ {images/} }

\theoremstyle{plain}
\newtheorem{thm}{Theorem}[subsection] % reset theorem numbering for each chapter
\newtheorem{proposition}[thm]{Proposition}
\newtheorem{lemma}[thm]{Lemma}
\newtheorem{fact}[thm]{Fact}
\newtheorem{cor}[thm]{Corollary}
\newtheorem{post}[thm]{Postulate}

\theoremstyle{definition}
\newtheorem{defn}[thm]{Definition} % definition numbers are dependent on theorem numbers
\newtheorem{exmp}[thm]{Example} % same for example numbers
\newtheorem{notation}[thm]{Notation}
\newtheorem{remark}[thm]{Remark}
\newtheorem{condition}[thm]{Condition}
\newtheorem{question}[thm]{Question}
\newtheorem{construction}[thm]{Construction}
\newtheorem{exercise}[thm]{Exercise}
\newtheorem{example}[thm]{Example}
\newtheorem{observation}[thm]{Observation}
\newtheorem{algorithm}[thm]{Algorithm}

\newcommand{\bb}[1]{\mathbb{#1}}
\newcommand{\scr}[1]{\mathscr{#1}}
\newcommand{\call}[1]{\mathcal{#1}}
\newcommand{\psheaf}{\text{\underline{Set}}^{\scr{C}^{\text{op}}}}
\newcommand{\und}[1]{\underline{\hspace{#1 cm}}}
\newcommand{\adj}[1]{\text{\textopencorner}{#1}\text{\textcorner}}
\newcommand{\comment}[1]{}
\newcommand{\lto}{\longrightarrow}

\title{Continuous quantum computing}
\author{Will Troiani}
\date{August 2020}

\begin{document}
	\maketitle

\begin{defn}\label{def:cont_time_evolution}
	A \textbf{continuous time evolution} of $\bb{H}$ is a Hermitian operator $H$ on $\bb{H}$, this is the \textbf{Hamiltonian}.
\end{defn}

\begin{remark}
	One may be tempted to psychologically project mathematical depth onto: ``as \emph{discrete} is to \emph{continuous}, unitary is to Hermitian". This would be pareidolia though. What is suppressed in these notes is that the \textbf{evolution} of a continuous time evolution is a vectorial differential equation (Schr\"{o}dinger's equation)
	\begin{equation}\label{eq:schrodinger}
		i\hbar \frac{d\ket{\psi}}{dt} = H\ket{\psi}
	\end{equation}
	Single step time evolution can be modelled via continuous time evolution, this involves solving the differnetial equation \eqref{eq:schrodinger}, which is too far abroad from the targetted focus of these notes.
	
	What is important, is the mathematical \emph{definition} \ref{def:cont_time_evolution}. We will not need any continuous analogue to the second half of Definition \ref{def:time_evolution}.
\end{remark}

\end{document}