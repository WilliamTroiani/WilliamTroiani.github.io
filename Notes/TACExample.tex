\documentclass{tac}

% Prepared using tac.cls and diagxy (if you do not have diagxy, compiling this will require commenting lines 165-168)
% PLEASE READ comments in the text below
% PLEASE NOTE: source files for submission to TAC require a comment like the following
%              giving style, packages used, TeX implementation
% TAC style, 2 pp, Xy-pic ver 3.7, MikTeX version 3.1


% NOTE: packages used should be the first part of the preamble

\usepackage{xy}
% diagxy loads xy, so the line preceding is redundant

\input diagxy

\def\xypic{\hbox{\rm\Xy-pic}}

% The TAC hyperref setup should be reloaded after other packages

\usepackage[colorlinks=true]{hyperref}
\hypersetup{allcolors=[rgb]{0.1,0.1,0.4}}

% NOTE: TAC preamble macros come next...

\author{Michael Barr, Geoff Cruttwell, and Robert Rosebrugh}

% NOTE: that \thanks is outside the \author macro, unlike article style...

\thanks{We would like to thank Donald Knuth, Leslie Lamport, Kris
Rose, and all the others who make high quality math typesetting
possible.}

\address{Department of Mathematics and Statistics, McGill University\\
 805 Sherbrooke St. W, Montreal, QC, Canada H3A 2K6\\[5pt]
 Department of Mathematics and Computer Science, Mt. Allison
University\\67 York St., Sackville, NB, Canada E4L 1E6\\
}


\title{A \textsl{TAC} sampler}

% NOTE: this is required...

\copyrightyear{2020}

% NOTE: the next three are optional in this style (but are required for TAC publication)

\keywords{TAC, diagrams}
\amsclass{00A00}

% NOTE: that \CR here provides a vertical listing
\eaddress{barr@math.mcgill.ca\CR rrosebrugh@mta.ca}


% NOTE: author macros  BEGIN here
%       (they are all actually used in the article!!)
% *PLEASE* note the begin and end of author macros in the source file

\def\xypic{\hbox{\rm\Xy-pic}}

% examples of proclamation macros follow - these define environments like \begin{thm} ... \end{thm}
% by default they are italic, but roman proclamations for remarks, examples are also available
% with \newtheoremrm{}{}

\newtheorem{thm}{Theorem}
% note that \newtheorem{theorem}{Theorem} and several others are already in tac.sty
\newtheorem{theorem}{Theorem} 
\newtheorem{lem}{Lemma}

% note rm This is useful for remarks and examples
\newtheoremrm{rem}{Remark}

% TAC has a \proof environment, with abbreviations following; note also \mathrmdef{} and \mathbfdef{}

\let\pf\proof
\let\epf\endproof
\mathrmdef{Hom}
\mathbfdef{Set}

% author macros END here

\begin{document}

\maketitle
\begin{abstract}
 This is distributed as a sampler to illustrate good \textsl{TAC} style.
\end{abstract}

% NOTE: it is good practice to \label all headings (and proclamations) immediately

\section{Introduction}\label{sec-Introduction}

This note includes samples of what we consider good \textsl{TAC} style.  
There are no explicit skips nor any other explicit formatting
instructions in the \LaTeX\ code; these should be left to the journal style.  For the same
reason, the \LaTeX\ code has no explicit numbering of headings or proclamations of
Theorems and so forth. However, they \emph{are} labelled to allow
logical references in the code, such as to Theorem \ref{thm-easy-style}. 
Moreover, TAC style uses the hyperref package to create links to citations like 
\cite{LUG} and to internal references like Lemma \ref{lem-latex-2e}. 
Reference links work \emph{only} if \verb.\label.'s are used. All links are coloured a dark blue without boxing. 
Though permitted, external links are strongly deprecated because of their impermanence.

\emph{Please note that there are additional comments in the source file \texttt{sample.tex}
for this sampler that you are urged to consult. You should \textbf{also} consult
the on-line author instructions on the {\sl TAC} web site.}

\section{Main results}\label{sec-Main-results}

%NOTE: more verbosely, the next line could use \begin{lem} ... \end{lem}

\lem\label{lem-latex-2e} All papers must be in \LaTeX, version 2e.\endlem

\pf Otherwise the editors would have to do a lot of work to prepare the
paper for publication.
\epf


\thm\label{thm-easy-style} The {\sl TAC} style is easy to use.\endthm

\pf
Sectioning is the same as in \LaTeX\ article style;
proclamations such as definitions and theorems are easily specified
by macros such as \verb.\newtheorem{thm}{Theorem}.;
it is easy to use \verb.\mathrmdef{Hom}. to define a macro \verb.\Hom. that produces
roman $\Hom$ when used in math mode.  Similarly \verb.\mathbfdef{Set}. gives $\Set$
in bold.
\epf

% NOTE: If you use \cite at the beginning of a theorem-like environment,
% it must come immediately after the theorem and before any label.

\thm[\cite{LUG}]\label{thm-Lamport-1986}  The following are equivalent
\begin{enumerate}
\item Lists are best done with listing macros such as enumerate;

\item you will never have to renumber anything if you use automatic
numbering of lists and other things.
\end{enumerate}

\epf\endthm

\begin{rem}\label{rem-about-roman}
For proclaimed matter that should be set in Roman we use the {\sl TAC} macros \verb.\newtheoremrm{}{}..
\end{rem}

\section{Further comments}\label{sec-Further-comments}


\begin{list}{}{}
\item[$\bullet$] Source files should include \emph{all} and
\emph{only} author macros that are actually used. \item[$\bullet$]
Be sure to use macros for multicharacter identifiers, such as
\verb.\Hom. above. 
\item[$\bullet$] Be sure to distinguish between
$<$ and $\langle$ and similarly between $>$ and $\rangle$.  Not
only does the former character look wrong as a tuple delimiter,
but the spacing is completely wrong. 
\item[$\bullet$] For diagrams, use \xypic\ or diagxy.% , or Paul Taylor's diagram package.
They can be used together since diagxy is built on top of
\xypic. A syntax example from the diagxy manual is
\begin{verbatim}
$$\bfig
  \square/>>`>`>` >->/[A`B`C`D;e`f`g`m]
  \morphism(500,500)|m|/.>/<-500,-500>[B`C;h]
\efig$$
\end{verbatim}
which makes a familiar diagram:
% NOTE: If you do not have XY-pic and diagxy installed you will have to comment out
%       the next four lines.

$$\bfig
 \square/>>`>`>` >->/[A`B`C`D;e`f`g`m]
 \morphism(500,500)|m|/.>/<-500,-500>[B`C;h]
\efig$$
\item[$\bullet$] Although we accept most reasonable bibliographical styles, the
 following is the one we most strongly recommend.  It results in an 
 [\emph{author, year}] entry in the paper, rather than uninformative numbers in
 brackets. It allows use of e.g. \verb.\cite{LUG}. and the code for this article is:
\begin{verbatim}
\refs

\bibitem [Lamport, 1986]{LUG} L. Lamport, Latex User's Guide \&
Reference Manual. Addison-Wesley (fifth edition), 1986.

\endrefs
\end{verbatim}
\end{list}

 % Note: Although we accept most reasonable bibliographical styles, the
 % following is the one we most strongly recommend.  It results in the
 % author, year entry in the paper, rather than uninformative numbers in
 % brackets.

\refs

\bibitem [Lamport, 1986]{LUG} L. Lamport, Latex User's Guide \&
Reference Manual. Addison-Wesley (fifth edition), 1986.

\endrefs



\end{document}
