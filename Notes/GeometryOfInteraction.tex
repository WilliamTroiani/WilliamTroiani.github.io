\documentclass[12pt]{article}

\usepackage{amsthm}
\usepackage{amsmath}
\usepackage{amsfonts}
\usepackage{mathrsfs}
\usepackage{amssymb}
\usepackage{array}
\usepackage{units}
\usepackage{graphicx}
\usepackage{tikz-cd}
\usepackage{nicefrac}
\usepackage{hyperref}
\usepackage{bbm}
\usepackage{color}
\usepackage{tensor}
\usepackage{tipa}
\usepackage{bussproofs}
\usepackage{ stmaryrd }
\usepackage{ textcomp }
\usepackage{leftidx}
\usepackage{afterpage}
\usepackage{varwidth}
\usepackage{tasks}
\usepackage{ cmll }

\DeclareRobustCommand{\diamondtimes}{%
  \mathbin{\text{\rotatebox[origin=c]{45}{$\boxplus$}}}%
}

\newcommand\blankpage{
    \null
    \thispagestyle{empty}
    \addtocounter{page}{-1}
    \newpage
    }

\graphicspath{ {images/} }

\theoremstyle{plain}
\newtheorem{thm}{Theorem}[subsection] % reset theorem numbering for each chapter
\newtheorem{proposition}[thm]{Proposition}
\newtheorem{lemma}[thm]{Lemma}
\newtheorem{fact}[thm]{Fact}
\newtheorem{cor}[thm]{Corollary}

\theoremstyle{definition}
\newtheorem{defn}[thm]{Definition} % definition numbers are dependent on theorem numbers
\newtheorem{exmp}[thm]{Example} % same for example numbers
\newtheorem{notation}[thm]{Notation}
\newtheorem{remark}[thm]{Remark}
\newtheorem{condition}[thm]{Condition}
\newtheorem{question}[thm]{Question}
\newtheorem{construction}[thm]{Construction}
\newtheorem{exercise}[thm]{Exercise}
\newtheorem{example}[thm]{Example}
\newtheorem{aside}[thm]{Aside}

\def\doubleunderline#1{\underline{\underline{#1}}}
\newcommand{\bb}[1]{\mathbb{#1}}
\newcommand{\scr}[1]{\mathscr{#1}}
\newcommand{\call}[1]{\mathcal{#1}}
\newcommand{\psheaf}{\text{\underline{Set}}^{\scr{C}^{\text{op}}}}
\newcommand{\und}[1]{\underline{\hspace{#1 cm}}}
\newcommand{\adj}[1]{\text{\textopencorner}{#1}\text{\textcorner}}
\newcommand{\comment}[1]{}
\newcommand{\lto}{\longrightarrow}
\newcommand{\rone}{(\operatorname{R}\bold{1})}
\newcommand{\lone}{(\operatorname{L}\bold{1})}
\newcommand{\rimp}{(\operatorname{R} \multimap)}
\newcommand{\limp}{(\operatorname{L} \multimap)}
\newcommand{\rtensor}{(\operatorname{R}\otimes)}
\newcommand{\ltensor}{(\operatorname{L}\otimes)}
\newcommand{\rtrue}{(\operatorname{R}\top)}
\newcommand{\rwith}{(\operatorname{R}\&)}
\newcommand{\lwithleft}{(\operatorname{L}\&)_{\operatorname{left}}}
\newcommand{\lwithright}{(\operatorname{L}\&)_{\operatorname{right}}}
\newcommand{\rplusleft}{(\operatorname{R}\oplus)_{\operatorname{left}}}
\newcommand{\rplusright}{(\operatorname{R}\oplus)_{\operatorname{right}}}
\newcommand{\lplus}{(\operatorname{L}\oplus)}
\newcommand{\prom}{(\operatorname{prom})}
\newcommand{\ctr}{(\operatorname{ctr})}
\newcommand{\der}{(\operatorname{der})}
\newcommand{\weak}{(\operatorname{weak})}
\newcommand{\exi}{(\operatorname{exists})}
\newcommand{\fa}{(\operatorname{for\text{ }all})}
\newcommand{\ex}{(\operatorname{ex})}
\newcommand{\cut}{(\operatorname{cut})}
\newcommand{\ax}{(\operatorname{ax})}
\newcommand{\negation}{\sim}
\newcommand{\true}{\top}
\newcommand{\false}{\bot}
\newcommand{\tagarray}{\mbox{}\refstepcounter{equation}$(\theequation)$}
\newcommand{\startproof}[1]{
\AxiomC{#1}
\noLine
\UnaryInfC{$\vdots$}
}
\newenvironment{scprooftree}[1]%
  {\gdef\scalefactor{#1}\begin{center}\proofSkipAmount \leavevmode}%
  {\scalebox{\scalefactor}{\DisplayProof}\proofSkipAmount \end{center} }

\usepackage[margin=1cm]{geometry}

\title{Geometry of Interaction}
\author{Will Troiani}
\date{January 2021}

\begin{document}

\maketitle
\tableofcontents


%
%
%
%
%
%
%
%
%
%
%
%
\section{Geometry of Interaction}
The product and coproduct of finitely many copies of the Hilbert space $\bb{H} = \ell^2$ are both given by the direct sum. So, any morphism $f: \bb{H}^n \lto \bb{H}^m$ can be decomposed according to the following commutative diagram:
\begin{equation}
    \begin{tikzcd}
    \bigoplus_{i = 1}^n\bb{H}\arrow[r,"f"] & \bigoplus_{j = 1}^m\bb{H}\\
    \bb{H}\arrow[u]\arrow[r]\arrow[r,swap,"{f_{i,j}}"] & \bb{H}\arrow[u]
    \end{tikzcd}
\end{equation}
Thus, the data of such a morphism $f$ is equivalent to the data of a set of morphisms $\lbrace f_{i,j}: \bb{H} \lto \bb{H}\rbrace$ which, if $\operatorname{End}\bb{H}$ denotes the space of endomorphisms on $\bb{H}$, can be written as an element of $M_{n,m}(\operatorname{End}\bb{H})$.

Fix a set of bijections
\begin{equation}
    \scr{B} := \lbrace \alpha_i: \bb{N} \lto \bb{N}^i \mid i > 0\rbrace
\end{equation}
and let $\scr{I}$ denote the corresponding set of isometric isomorphisms
\begin{equation}
    \scr{I} := \lbrace \hat{\alpha}_i = \bigoplus_{j = 1}^ip_{ij}: \bb{H} \lto \bb{H}^i \mid \alpha_i \in \scr{B}\rbrace
\end{equation}
\textcolor{red}{Notation a bit confusing}.
Using the isomorphisms $\hat{\alpha}_i$ we can associate to each multiplicative proof-net \cite{proof-nets} an isometric isomorphism. Recall \cite{proof-nets} that we denote the set of proofs in MLL by $\Sigma$.
\begin{defn}
We let
\begin{equation}
    \llbracket \cdot \rrbracket: \Sigma \lto \operatorname{End}\bb{H}
\end{equation}
denote the function defined inductively by associating to each multiplicative, linear logic deduction rule \cite{proof-nets} an element of $\operatorname{End}\bb{H}$:
\begin{itemize}
    \item Axiom:
    \begin{center}
    \begin{tabular}{ >{\centering}m{7cm} >{\centering}m{7cm} }
        \begin{prooftree}
            \AxiomC{}
            \RightLabel{$\ax$}
            \UnaryInfC{$\vdash A, \negation A$}
        \end{prooftree}
    &
    \begin{align*}
        \bb{H} \stackrel{\hat{\alpha}_2}{\lto} \bb{H}^2 \stackrel{M}{\lto} \bb{H}^2 \stackrel{\hat{\alpha}^{-1}}{\lto} \bb{H}
    \end{align*}
    where $M =
    \begin{pmatrix}
    0 & 1\\
    1 & 0
    \end{pmatrix}$
    \end{tabular}
\end{center}
\item Cut:

\end{itemize}

\begin{example}\label{ex:axiom_GoI}
Consider the proof-net $\pi_1$, for clarity sakes we explicitly put in the occurrence names
\begin{equation}
\begin{tikzcd}
(A,1)\arrow[r,bend left, dash] & (\negation A,2)\arrow[r,bend right, dash] & (A,3)\arrow[r, bend left, dash] & (\negation A,4)
\end{tikzcd}
\end{equation}
which is equivalent under cut-elimination to the following which we denote $\pi_2$:
\begin{equation}
\begin{tikzcd}
(A,1)\arrow[r, bend left, dash] & (\negation A,4)
\end{tikzcd}
\end{equation}
We have
\begin{equation}
\llbracket \pi_1 \rrbracket = 
\begin{matrix}
& (A,3) & (\negation A,2) & (A,1) & (\negation A,4)\\
(A,3) & 0&&&1\\
(\negation A,2) &&0&1&\\
(A,1)&&1&0&\\
(\negation A,4)&1&&&0
\end{matrix}
\end{equation}
We then perform matrix multiplication:
\begin{equation}
\llbracket \pi_1 \rrbracket \sigma \llbracket \pi_1 \rrbracket = 
\begin{bmatrix}
0&&&1\\
&0&1&\\
&1&0&&\\
1&&&0
\end{bmatrix}
\begin{bmatrix}
0&1&&\\
1&0&&\\
&&&&\\
&&&&
\end{bmatrix}
\begin{bmatrix}
0&&&1\\
&0&1&\\
&1&0&&\\
1&&&0
\end{bmatrix}
=
\begin{bmatrix}
&&&&\\
&&&&\\
&&0&1\\
&&1&0
\end{bmatrix}
\cong \llbracket \pi_2\rrbracket
\end{equation}
\end{example}
\begin{example}

\end{example}
\textcolor{red}{Warning: have not done cut-elimination for proof-structures yet.}
The following crucial Lemma will be used to prove both GoI 0 and multiplicative GoI 1.

We now use this to prove GoI 0:
\begin{defn}
Let $\pi$ be a proof-structure
\end{defn}




\begin{thm}[Geometry of Interaction 0]
Let $\pi$ be a proof-structure 
\end{thm}


\end{defn}










\begin{thebibliography}{9}
\bibitem{linearlogic} \emph{Linear Logic}, J.Y. Girard

\bibitem{girard} \emph{Geometry of Interaction I}, J.Y. Girard.

\bibitem{ILSC} \emph{Intuitionistic, linear sequent calculus} W. Troiani.

\bibitem{proof-nets} \emph{Proof-nets}, W Troiani

\end{thebibliography}

\end{document}
