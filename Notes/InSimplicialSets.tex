\documentclass[12pt]{article}

\usepackage{amsthm}
\usepackage{amsmath}
\usepackage{amsfonts}
\usepackage{mathrsfs}
\usepackage{amssymb}
\usepackage{units}
\usepackage{graphicx}
\usepackage{tikz-cd}
\usepackage{nicefrac}
\usepackage{hyperref}
\usepackage{bbm}
\usepackage{color}
\usepackage{tensor}
\usepackage{tipa}
\usepackage{bussproofs}
\usepackage{ stmaryrd }
\usepackage{ textcomp }
\usepackage{leftidx}
\usepackage{afterpage}
\usepackage{appendix}
\usepackage{ tipa }
\usepackage{quiver}

\newcommand\blankpage{
	\null
	\thispagestyle{empty}
	\addtocounter{page}{-1}
	\newpage
}

\graphicspath{ {images/} }

\theoremstyle{plain}
\newtheorem{thm}{Theorem}[subsection] % reset theorem numbering for each chapter
\newtheorem{proposition}[thm]{Proposition}
\newtheorem{lemma}[thm]{Lemma}
\newtheorem{fact}[thm]{Fact}
\newtheorem{cor}[thm]{Corollary}

\theoremstyle{definition}
\newtheorem{defn}[thm]{Definition} % definition numbers are dependent on theorem numbers
\newtheorem{exmp}[thm]{Example} % same for example numbers
\newtheorem{notation}[thm]{Notation}
\newtheorem{remark}[thm]{Remark}
\newtheorem{condition}[thm]{Condition}
\newtheorem{question}[thm]{Question}
\newtheorem{construction}[thm]{Construction}
\newtheorem{exercise}[thm]{Exercise}
\newtheorem{example}[thm]{Example}
\newtheorem{observation}[thm]{Observation}

\newcommand{\bb}[1]{\mathbb{#1}}
\newcommand{\scr}[1]{\mathscr{#1}}
\newcommand{\call}[1]{\mathcal{#1}}
\newcommand{\psheaf}{\text{\underline{Set}}^{\scr{C}^{\text{op}}}}
\newcommand{\und}[1]{\underline{\hspace{#1 cm}}}
\newcommand{\adj}[1]{\text{\textopencorner}{#1}\text{\textcorner}}
\def\doubleunderline#1{\underline{\underline{#1}}}
\newcommand{\comment}[1]{}
\newcommand{\lto}{\longrightarrow}
\newcommand{\rone}{(\operatorname{R}\bold{1})}
\newcommand{\lone}{(\operatorname{L}\bold{1})}
\newcommand{\rimp}{(\operatorname{R} \multimap)}
\newcommand{\limp}{(\operatorname{L} \multimap)}
\newcommand{\rtensor}{(\operatorname{R}\otimes)}
\newcommand{\ltensor}{(\operatorname{L}\otimes)}
\newcommand{\rtrue}{(\operatorname{R}\top)}
\newcommand{\rwith}{(\operatorname{R}\&)}
\newcommand{\lwithleft}{(\operatorname{L}\&)_{\operatorname{left}}}
\newcommand{\lwithright}{(\operatorname{L}\&)_{\operatorname{right}}}
\newcommand{\rplusleft}{(\operatorname{R}\oplus)_{\operatorname{left}}}
\newcommand{\rplusright}{(\operatorname{R}\oplus)_{\operatorname{right}}}
\newcommand{\lplus}{(\operatorname{L}\oplus)}
\newcommand{\prom}{(\operatorname{prom})}
\newcommand{\ctr}{(\operatorname{ctr})}
\newcommand{\der}{(\operatorname{der})}
\newcommand{\weak}{(\operatorname{weak})}
\newcommand{\exi}{(\operatorname{exists})}
\newcommand{\fa}{(\operatorname{for\text{ }all})}
\newcommand{\ex}{(\operatorname{ex})}
\newcommand{\cut}{(\operatorname{cut})}
\newcommand{\ax}{(\operatorname{ax})}
\newcommand{\negation}{\sim}
\newcommand{\true}{\top}
\newcommand{\false}{\bot}
\DeclareRobustCommand{\diamondtimes}{%
	\mathbin{\text{\rotatebox[origin=c]{45}{$\boxplus$}}}%
}
\newcommand{\tagarray}{\mbox{}\refstepcounter{equation}$(\theequation)$}
\newcommand{\startproof}[1]{
	\AxiomC{#1}
	\noLine
	\UnaryInfC{$\vdots$}
}
\newenvironment{scprooftree}[1]%
{\gdef\scalefactor{#1}\begin{center}\proofSkipAmount \leavevmode}%
	{\scalebox{\scalefactor}{\DisplayProof}\proofSkipAmount \end{center} }

\usepackage[margin=1cm]{geometry}

\bibliographystyle{dinat}

\title{In the category of simplicial sets}
\author{Will Troiani}

\begin{document}
	
	\maketitle
\section{In the topos of simplicial sets}
As a hands on example of the methodology presented in Section \cite[\S 5]{FiniteColimits} we consider the particular topos $\underline{\operatorname{sSet}}$ of simplicial sets (Definition \ref{def:simplicialset} below). Recall that associated to every simplicial set $S$ is its \emph{Geometric Realisation} $|S|$ \cite{Milnor}. Although this notion will not be required for this Section, awareness of it will help with guiding intuition.

\begin{defn}
	\label{simplexcategory}
	The \textbf{simplex category} $\Delta$ is the category whose objects are sets of the form $\lbrace 0,1,...,n\rbrace$ for some $n$, these will be denoted $[n]$. The morphisms of this category are order preserving functions. For any positive integer $k$, let $\Delta_{\leq k}$ be the full subcategory of $\Delta$ with objects $\lbrace [0],...,[k]\rbrace$.
\end{defn}
%
There is a canonical way of factorising morphisms in the simplex category:
%
\begin{defn}
	Define
	\begin{align*}
		\epsilon_n^i: [n - 1] &\to [n]\\
		j &\mapsto 
		\begin{cases}
			j&j < i\\
			j+1&j \geq i
		\end{cases}
	\end{align*}
	and
	\begin{align*}
		\eta_n^i: [n + 1] &\to [n]\\
		j &\mapsto 
		\begin{cases}
			j& j \leq i\\
			j-1& j > i
		\end{cases}
	\end{align*}
\end{defn}
%
\begin{thm}
	\label{facedegen}
	Any morphism $[n] \to [m]$ in $\Delta$ can be written uniquely as
	\[\epsilon_{m}^{i_1}\epsilon_{m-1}^{i_2}...\epsilon_{m - k + 1}^{i_l}\eta_{m - k}^{j_1}\eta_{m - k + 1}^{j_2}...\eta_{m-1}^{j_{k-1}}\eta_m^{j_k}\]
	with $m \geq i_1 \geq i_2 \geq ... \geq i_l \geq 0$, and $0 \leq j_1 \leq j_2 \leq ... \leq j_{k} \leq n$.
\end{thm}
\begin{proof}
	See \cite[\S VIII.5.1]{MacLane}.
\end{proof}
%
\begin{defn}
	\label{def:simplicialset}
	A \textbf{simplicial set} is a functor $\Delta^\text{op} \to \text{\underline{Set}}$, where \underline{Set} is the category of sets. The collection of these, along with the collection of natural transformations between them, forms a category \text{\underline{sSet}}, the category of simplicial sets.
\end{defn}

\begin{example}\label{ex:interval}
	Consider the simplicial set $S$ given by the colimit of the following diagram, the geometric realisation of which is the interval.
	% https://q.uiver.app/?q=WzAsMyxbMCwwLCJbMF0iXSxbMCwxLCJbMV0iXSxbMCwyLCJbMF0iXSxbMCwxLCJcXGVwc2lsb25fMF4xIl0sWzIsMSwiXFxlcHNpbG9uXzFeMSIsMl1d
	\[\begin{tikzcd}
		{[0]} \\
		{[1]} \\
		{[0]}
		\arrow["{\epsilon_0^1}", from=1-1, to=2-1]
		\arrow["{\epsilon_1^1}"', from=3-1, to=2-1]
	\end{tikzcd}\]
	We construct the diagram \eqref{eq:colimit_complete} in this setting. There are 5 morphisms in this diagram, including the identity morphisms. These induce two morphisms:
	\begin{equation}
		\begin{tikzcd}
			{\Omega^{[0]} \coprod \Omega^{[0]} \coprod \Omega^{[0]} \coprod \Omega^{[0]} \coprod \Omega^{[1]}}\arrow[r,shift left, "{s_0}"]\arrow[r,shift right,swap,"{s_1}"] & {\Omega^{[0]} \coprod \Omega^{[1]} \coprod \Omega^{[0]}}
		\end{tikzcd}
	\end{equation}
	We now give the description of these morphisms $s_1,s_2$ as terms $t_1,t_2$. Let $z_1, z_2, z_3, z_4, z_5 : \Omega^{[0]}$, and let $z_6 : \Omega^{[1]}$. Here, the idea is that $z_1$ corresponds to $[0]^{\epsilon_0^1}$, $z_2$ corresponds to $[1]^{\operatorname{id}}$, $z_3$ to $[0]^{\operatorname{id}}$, $z_4$ to $[0]^{\epsilon_1^1}$, and $z_5$ to $[1]^{\operatorname{id}}$.
	\begin{align}
		t_0 &=\langle \langle z_1 \cup z_2, z_3 \cup z_4 \rangle , z_5 \rangle \\
		t_1 &= \langle \langle z_2, z_3\rangle, \epsilon_0^1(z_1) \cup \epsilon_1^1(z_5) \cup z_6 \rangle
	\end{align}
	The interesting term here is $t_1$. Reading $t_1$ from left to right, we first read $\langle z_2, z_3\rangle$, indicating that the two copies of $[0]$ are not glued together, and the next component $\epsilon_0^1(z_1) \cup \epsilon_1^1(z_5) \cup z_6$ which describes how the 0 dimensional components are glued to the 1 dimensional component. This fits our intuition of how the geometric realisation of the simplicial set $S$ is constructed.
\end{example}
A more complicated example is given by the following.
\begin{example}
	Let $S$ be the simplicial set given by the colimit of the following diagram, the geometric realisation of which is a triangle. We have artificially added labellings to the copies of objects in this diagram for clarity.
	% https://q.uiver.app/?q=WzAsNyxbMiwwLCJbMF0iXSxbMSwxLCJbMV0iXSxbMCwyLCJbMF0iXSxbMiwxLCJbMl0iXSxbMiwyLCJbMV0iXSxbNCwyLCJbMF0iXSxbMywxLCJbMV0iXSxbMCwxLCJcXGVwc2lsb25fMV4wIiwyXSxbMiwxLCJcXGVwc2lsb25fMV4wIl0sWzIsNCwiXFxlcHNpbG9uXzFeMCIsMl0sWzUsNCwiXFxlcHNpbG9uXzFeMSJdLFs1LDYsIlxcZXBzaWxvbl8xXjAiLDJdLFswLDYsIlxcZXBzaWxvbl8xXjEiXSxbNiwzLCJcXGVwc2lsb25fMl4yIl0sWzEsMywiXFxlcHNpbG9uXzJeMSIsMl0sWzQsMywiXFxlcHNpbG9uXzBeMCJdXQ==
	\[\begin{tikzcd}
		&& {[0]_1} \\
		& {[1]_4} & {[2]_7} & {[1]_5} \\
		{[0]_2} && {[1]_6} && {[0]_3}
		\arrow["{\epsilon_1^0}"', from=1-3, to=2-2]
		\arrow["{\epsilon_1^0}", from=3-1, to=2-2]
		\arrow["{\epsilon_1^0}"', from=3-1, to=3-3]
		\arrow["{\epsilon_1^1}", from=3-5, to=3-3]
		\arrow["{\epsilon_1^0}"', from=3-5, to=2-4]
		\arrow["{\epsilon_1^1}", from=1-3, to=2-4]
		\arrow["{\epsilon_2^2}", from=2-4, to=2-3]
		\arrow["{\epsilon_2^1}"', from=2-2, to=2-3]
		\arrow["{\epsilon_2^0}", from=3-3, to=2-3]
	\end{tikzcd}\]
	We define the following variables.
	\begin{center}
		\begin{tabular}{c c c}
			$z^1_1,z^1_2: \Omega^{[0]_1}$ &$z^2_1,z^2_2: \Omega^{[0]_2}$ & $z^3_1,z^3_2: \Omega^{[0]_3}$\\
			$x_4 : \Omega^{[1]_4}$ & $x_5: \Omega^{[1]_5}$ & $x_6: \Omega^{[1]_6}$\\
			$y_6: \Omega^{[2]_7}$
		\end{tabular}
	\end{center}
	The term of interest is the following, we ignore the bracketting.
	\begin{equation}
		t_1 = \langle z_1^1 \cup z_2^1, z_1^2 \cup z_2^2, z_1^3 \cup z_2^3, \epsilon_1^0(z_2^1) \cup \epsilon_1^0(z_1^2), \epsilon_1^1(z_2^1) \cup \epsilon_1^0(z^3_1), \epsilon_1^0(z^2_2) \cup \epsilon_1^1(z^3_1), \epsilon_2^0(x_6) \cup \epsilon_2^1(x_4) \cup \epsilon_2^2(x_5), y_7\rangle
	\end{equation}
	Which, as in Example \ref{ex:interval}, agrees with the glueing instructions corresponding to the geometric realisation of $S$.
\end{example}
	
	\begin{thebibliography}{99}
		\bibitem{FiniteColimits} W. Troiani, \emph{Finite Colimits in the Internal Language of  Topos}
		\bibitem{May} J. May, \emph{A Crash Course in Algebraic Topology}, University of Chicago Press, Chicago, 1999.
		
	\end{thebibliography}
\end{document}