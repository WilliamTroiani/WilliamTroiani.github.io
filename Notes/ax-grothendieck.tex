\documentclass[12pt]{article}

\usepackage{amsthm}
\usepackage{amsmath}
\usepackage{amsfonts}
\usepackage{mathrsfs}
\usepackage{array}
\usepackage{amssymb}
\usepackage{units}
\usepackage{graphicx}
\usepackage{tikz-cd}
\usepackage{nicefrac}
\usepackage{hyperref}
\usepackage{bbm}
\usepackage{color}
\usepackage{tensor}
\usepackage{tipa}
\usepackage{bussproofs}
\usepackage{ stmaryrd }
\usepackage{ textcomp }
\usepackage{leftidx}
\usepackage{afterpage}
\usepackage{varwidth}
\usepackage{tasks}
\usepackage{ cmll }
\usepackage{quiver}
\usepackage{adjustbox}

\newcommand\blankpage{
	\null
	\thispagestyle{empty}
	\addtocounter{page}{-1}
	\newpage
}

\graphicspath{ {images/} }

\theoremstyle{plain}
\newtheorem{thm}{Theorem}[subsection] % reset theorem numbering for each chapter
\newtheorem{proposition}[thm]{Proposition}
\newtheorem{lemma}[thm]{Lemma}
\newtheorem{fact}[thm]{Fact}
\newtheorem{cor}[thm]{Corollary}

\theoremstyle{definition}
\newtheorem{defn}[thm]{Definition} % definition numbers are dependent on theorem numbers
\newtheorem{exmp}[thm]{Example} % same for example numbers
\newtheorem{notation}[thm]{Notation}
\newtheorem{remark}[thm]{Remark}
\newtheorem{condition}[thm]{Condition}
\newtheorem{question}[thm]{Question}
\newtheorem{construction}[thm]{Construction}
\newtheorem{exercise}[thm]{Exercise}
\newtheorem{example}[thm]{Example}
\newtheorem{aside}[thm]{Aside}

\def\doubleunderline#1{\underline{\underline{#1}}}
\newcommand{\bb}[1]{\mathbb{#1}}
\newcommand{\scr}[1]{\mathscr{#1}}
\newcommand{\call}[1]{\mathcal{#1}}
\newcommand{\psheaf}{\text{\underline{Set}}^{\scr{C}^{\text{op}}}}
\newcommand{\und}[1]{\underline{\hspace{#1 cm}}}
\newcommand{\adj}[1]{\text{\textopencorner}{#1}\text{\textcorner}}
\newcommand{\comment}[1]{}
\newcommand{\lto}{\longrightarrow}
\newcommand{\rone}{(\operatorname{R}\bold{1})}
\newcommand{\lone}{(\operatorname{L}\bold{1})}
\newcommand{\rimp}{(\operatorname{R} \multimap)}
\newcommand{\limp}{(\operatorname{L} \multimap)}
\newcommand{\rtensor}{(\operatorname{R}\otimes)}
\newcommand{\ltensor}{(\operatorname{L}\otimes)}
\newcommand{\rtrue}{(\operatorname{R}\top)}
\newcommand{\rwith}{(\operatorname{R}\&)}
\newcommand{\lwithleft}{(\operatorname{L}\&)_{\operatorname{left}}}
\newcommand{\lwithright}{(\operatorname{L}\&)_{\operatorname{right}}}
\newcommand{\rplusleft}{(\operatorname{R}\oplus)_{\operatorname{left}}}
\newcommand{\rplusright}{(\operatorname{R}\oplus)_{\operatorname{right}}}
\newcommand{\lplus}{(\operatorname{L}\oplus)}
\newcommand{\prom}{(\operatorname{prom})}
\newcommand{\ctr}{(\operatorname{ctr})}
\newcommand{\der}{(\operatorname{der})}
\newcommand{\weak}{(\operatorname{weak})}
\newcommand{\exi}{(\operatorname{exists})}
\newcommand{\fa}{(\operatorname{for\text{ }all})}
\newcommand{\ex}{(\operatorname{ex})}
\newcommand{\cut}{(\operatorname{cut})}
\newcommand{\ax}{(\operatorname{ax})}
\newcommand{\negation}{\sim}
\newcommand{\true}{\top}
\newcommand{\false}{\bot}
\DeclareRobustCommand{\diamondtimes}{%
	\mathbin{\text{\rotatebox[origin=c]{45}{$\boxplus$}}}%
}
\newcommand{\tagarray}{\mbox{}\refstepcounter{equation}$(\theequation)$}
\newcommand{\startproof}[1]{
	\AxiomC{#1}
	\noLine
	\UnaryInfC{$\vdots$}
}
\newenvironment{scprooftree}[1]%
{\gdef\scalefactor{#1}\begin{center}\proofSkipAmount \leavevmode}%
	{\scalebox{\scalefactor}{\DisplayProof}\proofSkipAmount \end{center} }


\title{Ax-Grothendieck via model theory}

\begin{document}
	\maketitle
	
	\begin{defn}\label{def:FoT_fields}
		We define $\call{F}$, the first order theory of fields, beginning with the first order language of fields. Let $\Sigma$ be a signature consisting of a single sort $A$. We introduce 5 function symbols.
		\begin{itemize}
			\item $0,1: A$,
			\item $-: A \lto A$,
			\item $+, \cdot: A \times A \lto A$.
			\end{itemize}
		The first order language of fields has no relation symbols.
		
		The axioms are given as follows.
		\begin{align}
			&(x + y) + z = x + (y + z)\\
			&x + y = y + x\\
			&x + 0 = x\\
			&(x\cdot y)\cdot z = x \cdot (y \cdot z)\\
			&x \cdot 1 = 1 \cdot x = x\\
			&x\cdot (y + z) = x\cdot y + x \cdot z\\
			&x + (-x) = 0\\
			&x\neq 0 \Rightarrow \exists y, xy = 1
			\end{align}
		This set of formulas forms the axioms of $\call{F}$.
		\end{defn}
	
	\begin{defn}
		For each $d \geq 1$ define the following formula.
		\begin{equation}
			P_d := \forall a_0 \ldots \forall a_d \exists x, a_d \neq 0 \wedge a_0 + a_1 x + \ldots + a_{d-1}x^{d-1} + a_dx^d = 0
			\end{equation}
		For each prime number $p$ define the following formula.
		\begin{equation}\label{eq:characteristic}
			S_d := 1 + \ldots + 1 = 0
			\end{equation}
		where there are $d$ instances of $1$ in \eqref{eq:characteristic}.
		\end{defn}
	
	\begin{defn}
		Let $\call{ACF}$ denote the \textbf{first order theory of algebrically closed fields} which is over the same language as $\call{F}$ and consists of all the axioms of Definition \ref{def:FoT_fields} along with $P_d$ for each $d \geq 1$.
		
		The \textbf{first order theory of algebraically closed fields of characteristic $p$} is denoted $\call{ACF}_p$ and consists of all the axioms of $\call{ACF}$ along with $S_p$.
		
		Lastly, the \textbf{first order theory of algebraically closed fields of characteristic $0$} is denoted $\call{ACF}_0$ and consists of all the axioms of $\call{ACF}$ along with the formula $\neg S_p$ for each prime number $p$.
		\end{defn}
	
	
	
	
	
	
	
	
	
	
	
	
	
	
	
	
	
	
	
	
	
	
	\begin{lemma}\label{lem:finite_characteristic_ax}
		Let $f: \overline{\bb{F}_p}^n \lto \overline{\bb{F}_p}^n$ be a polynomial. If $f$ is injective then it is surjective.
		\end{lemma}
	\begin{proof}
		Let $\underline{y} = (y_1,\ldots, y_n) \in \overline{\bb{F}_p}^n$ be arbitrary. Consider the field extension $K \supseteq \bb{F}_p$ generated by $y_1,\ldots, y_n$ as well as the coefficients of $f$. Since every element of $\overline{\bb{F}_p}$ is algebraic over $\bb{F}_p$ (by the definition of an algebraic closure) we have $K$ is an algebraic extension and thus finite of $\bb{F}_p$. Since $\bb{F}_p$ is finite, this implies $K$ is finite. Lastly, we notice that fields are closed under polynomial expressions, and so $f(K^n) \subseteq K^n$, which by injectivity and finiteness implies surjectivity.
		\end{proof}
	
	\begin{cor}
		Let $k$ be an algebraically closed field and $f: k^n \lto k^n$ a polynomial. If $f$ is injective then it is surjective.
		\end{cor}
	\begin{proof}
		There is a small slieght of hand involved, we need to turn the statement of the corollary into a first order formula, but we cannot do that if try to work with a polynomial of arbitrary degree. So instead we will consider the statement ``If $f$ is injective and has degree at most $d$ then it is surjective". The idea is to write out the following statement
		\begin{align}
			\forall a_0\ldots \forall a_{d}&(\forall x \forall y, f(x) = f(y) \Rightarrow x = y)\\
			&\Longrightarrow \forall y \exists x, y = f(x)
		\end{align}
	however we need to write out $f$ explicitly. This is where we use the fact that $f$ is a polynomial. Our first order statement is:
		\begin{align*}
			\forall a_0\ldots \forall a_{d}&(\forall x \forall y, a_0 + a_1 x + \ldots + a_{d-1}x^{d-1} + a_dx^d\\
			&= a_0 + a_1 y + \ldots + a_{d-1}y^{d-1} + a_dy^d \Rightarrow x = y)\\
			&\Longrightarrow \forall y \exists x, y = a_0 + a_1 x + \ldots + a_{d-1}x^{d-1} + a_dx^d
			\end{align*}
		Denote this formula $B_d$.
		
		Since $\call{ACF}_0$ is complete, this statement is either proveable or its negation is proveable. That is, either $\call{ACF}_0 \vdash B_d$ or $\call{ACF}_0 \vdash \neg B_d$. Suppose $\call{ACF}_0 \vdash \neg B_d$ and let $\pi$ be such a proof. Since $\pi$ is finite, only finitely many axioms of $\call{ACF}_0$ appear amongst its premises. This, there exists some prime $q$ such that $\neg S_q$ does \emph{not} appear amongst the premises of $\pi$. That is, $\pi$ is a valid proof in $\call{ACF}_q$! By soundness of $\call{ACF}_q$ we derive a contradiction to Lemma \ref{lem:finite_characteristic_ax}, and so we must have $\call{ACF}_0 \vdash B_d$. The result then follows by soundness of $\call{ACF}_0$.
		\end{proof}
	
	\begin{lemma}
		\label{algebraicclosure}
		Every field $F$ can be embedded into an algebraically closed field $\bar{F}$.
	\end{lemma}
	\begin{proof}
		Let $\Lambda$ be the collection of monic, irreducible polynomials with coefficients in $F$. For each $f \in F$, let $u_{f,0},...,u_{f,d}$ be formal indeterminants, where $d$ is the degree of $f$. Let $F[\lbrace U\rbrace]$ be the polynomial ring over $F$ where $U$ is the collection of all $u_{f,i}$. Write
		\[f - \prod_{i = 0}^d(x - u_{f,i}) = \sum_{i = 0}^{d-1}\alpha_{f,i}x^i \in F[\lbrace U \rbrace][x]\]
		Let $I$ be the ideal generated by $\alpha_{f,i}$. $I$ is not all of $F[\lbrace U \rbrace]$ so there exists a maximal ideal $M$ containing $I$. Let $F_1 = F[\lbrace U \rbrace]/M$. Repeat this process to define $f_i$ for all $i > 0$. Then $\cup_{i = 1}^\infty F_i$ is algebraically closed which $F$ embeds into, and moreover is an algebraic extension of $F$.
	\end{proof}

\begin{cor}
	If $F$ is infinite, then the cardinality of $F$ is equal to the cardinality of $\overline{F}$.
	
	If $F$ is finite, then the cardinality of $\overline{F}$ is countably infinite.
	\end{cor}
\begin{proof}
	Using the notation of the proof of Lemma \ref{algebraicclosure}, we first observe that the $|\{ U \}| = | F |$
	\end{proof}

	
	\begin{lemma}
		Let $p$ be either a prime number or $0$ and let $\kappa \geq \aleph_1$ be an uncountable cardinal. There exists an algebraically closed field of characteristic $p$ whose cardinality is $\kappa$. Moreover, this field is unique up to isomorphism.
		\end{lemma}
	\begin{proof}
		Define $F$ to be
		\begin{equation}
			p = \begin{cases}
				\bb{Q},& p = 0\\
				\bb{F}_p, & p \neq 0
				\end{cases}
			\end{equation}
		Let $X$ be any set of cardinality $\kappa$ (eg, $X = \bb{R}$) and consider the polynomial ring $F[\{ X \}]$. The ideal $I \subseteq F[ \{ X \} ]$ generated by $X$ is not all of $F[\{ X \}]$ and so is contained in some maximal ideal $\frak{m}$ (using Zorn's Lemma). The field $F[\{ X \}]/\frak{m}$ has cardinality $\aleph_1$, as there are countably many polynomials over a single indeterminant, and we claim that the algebraic closure $\overline{F[\{ X \}]/\frak{m}}$ of $F[\{ X \}]/\frak{m}$ also has cardinality $\kappa$.
		
		The argument above show that in the notation of Lemma \ref{algebraicclosure}, that $F_i$ has cardinality $\kappa$ for all $i \geq 0$. Since $\overline{F[\{ X \}]/\frak{m}}$ is the countable union of all of these fields, it follows that the cardinality of $\overline{F[\{ X \}]/\frak{m}}$ is $\kappa$.
		
		The uniqueness claim follows easily by considering a transcendental basis of $\overline{F[\{ X \}]/\frak{m}}$ and observing that this basis has cardinality $\kappa$. The rest follows from the universal property of the algebraic closure.
		\end{proof}
	
	\begin{cor}
		There is only one model (up to isomorphism) of $\call{ALG}_0$ and of $\call{ALG}_p$ for each cardinal $\kappa \geq \aleph_1$.
		\end{cor}
	
	\end{document}






























